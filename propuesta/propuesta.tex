\documentclass[a4paper,11pt]{article}
\usepackage[T1]{fontenc}
\usepackage[utf8]{inputenc}
\usepackage{lmodern}

\title{Propuesta de Tesis}
\author{Juan Manuel Pérez}

\begin{document}

\maketitle

\begin{abstract}

El \emph{entrainment} (mimetización) es un fenómeno inconsciente que se manifiesta a través de la adaptación de posturas, forma de hablar, gestos faciales y otros comportamientos entre dos o más interactores. A su vez, la ocurrencia de esta mimetización está fuertemente emparentada con el sentimiento de empatía y compenetración entre los participantes.
 
En esta tesis, nos proponemos explorar una técnica algorítmica para detectar el entrainment entre variables prosódicas de dos personas. Esta técnica nos permitirá determinar si existe o no convergencia para ciertos parámetros, y ver como está ésto correlacionado con variables sociales tales como empatía (escribir más acá!)


\end{abstract}


\section{Introducción}

Los sistemas de diálogo humano-computadora son cada vez más frecuentes, y sus aplicaciones comprenden una 
amplia gama de rubros: desde aplicaciones móviles, motores de búsqueda, juegos, o tecnologías de asistencia para 
ancianos y discapacitados.

Si bien es cierto que estos sistemas logran captar la dimensión lingüística de la comunicación humana, tienen un déficit
importante a la hora de procesar y transmitir el aspecto superestructural de la comunicación, que radica en el intercambio
de afecto, emociones, actitudes y otras intenciones de los participantes. La habilidad de los participantes de poder 
expresar, comprender, y reaccionar de acuerdo a estas señales sociales es necesario para el entendimiento mutuo y una
comunicación exitosa.
 
Un aspecto particular de la comunicación es el fenómeno de \emph{entrainment}(arrastre, mimetización, efecto camaleón), que
comprende la adaptación inconsciente de las variables acústicas/prosódicas(a/p) (por ejemplo, el tono de la voz, la velocidad del habla, etc) de manera dinámica en el transcurso de una o varias interacciones. Este fenómeno ha sido largamente estudiado, y se ha visto que la convergencia de los participantes en estas variables ocurre en conjunto con un mayor sentimiento de simpatía por sus interlocutores, 

