\documentclass[a4paper,11pt]{article}
\usepackage[T1]{fontenc}
\usepackage[utf8]{inputenc}
\usepackage{lmodern}

\title{Propuesta de Tesis}
\author{Juan Manuel Pérez}

\begin{document}

\maketitle

\begin{abstract}

El entrainment (mimetización) es un fenómeno inconsciente que se manifiesta a través de la adaptación de posturas, forma de hablar, gestos faciales y otros comportamientos entre dos o más interactores. A su vez, la ocurrencia de esta mimetización está fuertemente emparentada con el sentimiento de empatía y compenetración entre los participantes.
 
En esta tesis, nos proponemos explorar una técnica algorítmica para detectar el entrainment entre variables prosódicas de dos personas. Esta técnica nos permitirá determinar si existe o no convergencia para ciertos parámetros, y ver como está ésto correlacionado con variables sociales tales como empatía (escribir más acá!)


\end{abstract}


\end{document}
