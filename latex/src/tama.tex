\section{Descripción TAMA}

Para construir la serie de tiempo de cada interlocutor, dividimos primero el diálogo en ventanas solapadas de igual tamaño \cite{KOU2008}. A la diferencia entre ventana y ventana llamaremos \emph{frame step}, y al tamaño de ventana \emph{frame length}.

\begin{figure}
\centering
\includegraphics[width=10cm]{images/tama.png}

\end{figure}

Nuestro corpus está anotado de manera que tenemos separadas los intervalos donde los interlocutores hablan (llamaremos a cada uno de éstos locuciones o \emph{utterances}). Para cada una de los frames, calcularemos la media

\begin{align}
    \mu &= \sum\limits_{i=1}^N f_i dr_i \label{eq:tama_mean}\\
    dr_i &= \frac{d_i}{\sum\limits_{i=1}^N d_i} \\
\end{align}

donde $i$ itera sobre las locuciones dentro del \emph{frame}, $d_i$ es la duración de la locución y $f_i$ es el valor de la \emph{feature} que estamos midiendo.

Como se ve en \ref{eq:tama_mean}, el $\mu$ que calculamos es una media ponderada por la duración de las locuciones. Así, por ejemplo, al calcular una serie de tiempo sobre el \emph{pitch}, la contribución de interjecciones (usualmente de alto valor) estará disminuída por su breve duración.

La serie de tiempo constará entonces de la secuencia de medias calculadas con \ref{eq:tama_mean} para cada uno de los frames.


\section{Selección de Ventana}

En \cite{KOU2009} se menciona una elección de \emph{frame step} y \emph{frame length} de 10s y 20s respectivamente. En el caso de nuestro corpus, quisimos buscar los parámetros que mejor se ajustaban a éste, manteniendo la superposición del 50\% entre ventanas sucesivas. Con lo que nos queda que $FL = 2 * FS$

¿Qué queremos optimizar? La métrica que elegimos para ésto es encontrar un balance entre un frame no tan grande (para no suavizar en exceso la curva) y que nos reduzca considerablemente la cantidad de indefiniciones; es decir, aquellas ventanas que tomamos en un interlocutor que no tienen ninguna interacción de su parte. Para ver ésto, graficamos la cantidad de indefiniciones en función del step tomado.

\begin{figure}
\centering
\includegraphics[width=10cm]{images/window_selection.png}
\end{figure}



Dentro del rango de $FS \in \{5'',6'', \ldots ,15'' \}$, graficamos para cada sesión, tarea y cada interlocutor las curvas de indefiniciones. A su vez, para mayor claridad, graficamos una curva que promedie todas las tareas de una sesión.


Para tener una visión general de lo que ocurría en todas las sesiones, graficamos una curva promedio de todas las sesiones. En ésta puede observarse que hasta $8''-10''$ hay un fuerte descenso de las indefiniciones, que luego se atenúa. Dado que en general tenemos tareas cortas, preferimos tomar $8''$ como step, y $16''$ como largo de ventana.

OBS: podríamos cambiar ésto a un boxplot!

\begin{figure}
\centering
\includegraphics[height=5cm]{images/window_selection_for_session.png}
\end{figure}
