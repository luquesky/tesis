\section{Modelo de Efectos Fijos}

\newcommand{\slopeestim}[1] { $\estslope \sim #1$ }

El modelo de efectos fijos agrega el concepto de heterogeneidad permitiendo que cada sujeto tenga su propio valor de ordenada al origen. En el caso concreto de nuestro corpus, dicha heterogeneidad puede deberse a multiplicidad de factores no medidos en él. Por ejemplo, la personalidad de los sujetos es un factor no medido y que puede influír en la dinámica de la interacción entre éstos.

Como el entrainment que medimos mediante el proceso TAMA es un proceso direccional (es decir, medimos tanto la influencia de un interlocutor sobre el otro y viceversa), definimos los ``grupos'' como las observaciones de entrainment y sus variables sociales de una sesión y el interlocutor sobre el cual consideramos la direccionalidad, tal y cual lo definimos en una sección anterior. Ésto nos arroja la cantidad de 24 grupos de observaciones



\section{Definición Formal de Efectos Fijos}

NOTA: Escribir ésto

\section{Resultados}

Utilizando como variable explicativa el \entrainment, los resultados no son son significativos. En (NOTA: agregar referencia a las tablas) podemos observar la tabla de coeficientes de esta regresión de efectos fijos.

Por otro lado, este modelo utilizando como variable independiente al valor absoluto del \entrainment dio valores sustancialmente apreciables. Las variables a/p \ENGMAX, \FOMEAN y \NOISETOHARMONICS poseen valores altamente significativos ( p-valor menor a 0.05) para al menos 2 variables sociales. En la tabla \ref{regresion_efectos_fijos_tabla} podemos ver la tabla del test de coeficientes con las variables sociales significativas resaltadas. Una versión simplificada tabla la podemos ver en \ref{sign_table} que grafica mediante tabla de doble entrada aquellos pares de variables a/p y variables sociales con coeficientes significativos y su signo.

Con respecto a las variables sociales, podemos observar que:

\begin{itemize}
  \item \svcontributes se relaciona positivamente con el \absentrainment cuando la variable a/p medida es \FOMEAN o bien \NOISETOHARMONICS. Esto significa que, cuando sube el valor absoluto del \entrainment, esta variable positiva también lo hace con buena probabilidad. Esto es un efecto esperable: cuando hay mimetización, hay colaboración para el éxito en el juego.
  \item \svclear, otra variable que refleja una visión positiva del juego, también se relaciona positivamente con el \absentrainment para las variables \FOMEAN, \NOISETOHARMONICS, \ENGMAX como a su vez para \PHONAVG y para \SYLCOUNT
  \item \svengaged, de la misma manera que las dos anteriores, relaciona positivamente pero sólo con \FOMEAN
  \item \svplanning y \svencourages, otras variables positivas, no presentan valores significativos.
  \item \svdifficult, una variable que representa una característica negativa de la conversación, se relaciona de igual con el \absentrainment cuando la variable acústico prosódica es \ENGMAX. Ésto contiene sentido, ya que a mayor mimetización de los interlocutores, la dificultad de éstos para hablar debería disminuir.
  \item La variable \svbored se comporta de idéntica manera, sólo que con \FOMEAN.
  \item \svdislikes no presenta valores significativos
\end{itemize}

\begin{figure}[p]
\centering
\adjustbox{max width=\textwidth}{
\begin{tabular}{rrrrr}
  \hline
\ENGMAX & $\estslope$ & Std. Error & t value & Significance \\ 
  \hline
contributes\_to\_successful\_completion & 0.0720 & 0.4258 & 1.689631E-01 & 0.8660 \\ 
  \stronghl making\_self\_clear & 1.6914 & 0.3820 & 4.427376E+00 & 0.0000 \\ 
  engaged\_in\_game & 0.3456 & 0.2528 & 1.367266E+00 & 0.1732 \\ 
  planning\_what\_to\_say & 0.5655 & 0.5208 & 1.085851E+00 & 0.2790 \\ 
  gives\_encouragement & 0.4739 & 0.3744 & 1.265523E+00 & 0.2073 \\ 
  \stronghl difficult\_for\_partner\_to\_speak & -0.6925 & 0.2863 & -2.418510E+00 & 0.0166 \\ 
  bored\_with\_game & 0.2110 & 0.2543 & 8.298495E-01 & 0.4077 \\ 
  dislikes\_partner & -0.4254 & 0.3438 & -1.237312E+00 & 0.2175 \\ 

  \hline
\ENGMEAN & $\estslope$ & Std. Error & t value & Significance \\ 
  \hline
  \softhl contributes\_to\_successful\_completion & 0.6552 & 0.3610 & 1.814712E+00 & 0.0712 \\ 
  making\_self\_clear & 0.9470 & 0.6080 & 1.557502E+00 & 0.1211 \\ 
  \softhl engaged\_in\_game & 0.7091 & 0.3847 & 1.843187E+00 & 0.0669 \\ 
  planning\_what\_to\_say & 0.3636 & 0.5756 & 6.316937E-01 & 0.5284 \\ 
  gives\_encouragement & 0.4051 & 0.3482 & 1.163506E+00 & 0.2461 \\ 
  \hl difficult\_for\_partner\_to\_speak & 0.5287 & 0.2515 & 2.101960E+00 & 0.0369 \\ 
  bored\_with\_game & -0.0036 & 0.4106 & -8.663987E-03 & 0.9931 \\ 
  dislikes\_partner & 0.5307 & 0.3889 & 1.364514E+00 & 0.1741 \\ 

  \hline
\FOMEAN & $\estslope$ & Std. Error & t value & Significance \\ 
  \hline
  \stronghl contributes\_to\_successful\_completion & 0.9752 & 0.3058 & 3.188448E+00 & 0.0017 \\ 
  \softhl making\_self\_clear & 0.6998 & 0.3907 & 1.791239E+00 & 0.0749 \\ 
  \stronghl engaged\_in\_game & 0.8538 & 0.2773 & 3.078945E+00 & 0.0024 \\ 
  planning\_what\_to\_say & 0.6430 & 0.5363 & 1.198966E+00 & 0.2321 \\ 
  gives\_encouragement & 0.0006 & 0.3885 & 1.577445E-03 & 0.9987 \\ 
  difficult\_for\_partner\_to\_speak & -0.5323 & 0.3835 & -1.388190E+00 & 0.1667 \\ 
  \stronghl bored\_with\_game & -0.7663 & 0.2582 & -2.968508E+00 & 0.0034 \\ 
  dislikes\_partner & 0.0688 & 0.3808 & 1.806265E-01 & 0.8569 \\ 

\FOMAX & $\estslope$ & Std. Error & t value & Significance \\ 
  \hline
  \softhl contributes\_to\_successful\_completion & 0.7628 & 0.4381 & 1.741129E+00 & 0.0833 \\ 
  making\_self\_clear & 0.6718 & 0.4129 & 1.626984E+00 & 0.1054 \\ 
  engaged\_in\_game & 0.5308 & 0.3776 & 1.405582E+00 & 0.1615 \\ 
  planning\_what\_to\_say & 0.0489 & 0.4210 & 1.161167E-01 & 0.9077 \\ 
  gives\_encouragement & 0.4724 & 0.5464 & 8.647145E-01 & 0.3883 \\ 
  difficult\_for\_partner\_to\_speak & -0.3208 & 0.2821 & -1.136927E+00 & 0.2570 \\ 
  bored\_with\_game & -0.2584 & 0.3764 & -6.865032E-01 & 0.4933 \\ 
  dislikes\_partner & 0.1249 & 0.3884 & 3.216226E-01 & 0.7481 \\ 
\end{tabular}}

\caption{Tablas con los resultados de la regresión de efectos fijos sobre el va,or absoluto de \entrainment para \ENGMAX, \ENGMEAN, \FOMEAN y \FOMAX. En la segunda columna se cita el valor de $\estslope$, la desviación estándar calculada, el t-valor obtenido y la significancia. Las columnas resaltadas corresponden a aquellas significantes, con diferentes matices de gris según $p < 0.10$, $p < 0.5$, o $p < 0.01$}
\label{fig:efectos_fijos_tabla1}

\end{figure}




\begin{figure}[pt!]
\centering
\adjustbox{max width=\textwidth}{
\begin{tabular}{rrrrr}
  \hline
  \hline
\NOISETOHARMONICS & $\estslope$ & Std. Error & t value & Significance \\ 
  \hline
  \hl contributes\_to\_successful\_completion & 0.7271 & 0.3439 & 2.114275E+00 & 0.0358 \\ 
  \stronghl making\_self\_clear & 1.3576 & 0.3613 & 3.758007E+00 & 0.0002 \\ 
  engaged\_in\_game & 0.1270 & 0.3431 & 3.702043E-01 & 0.7117 \\ 
  planning\_what\_to\_say & -0.1625 & 0.4264 & -3.811856E-01 & 0.7035 \\ 
  gives\_encouragement & 0.7665 & 0.4860 & 1.577201E+00 & 0.1165 \\ 
  difficult\_for\_partner\_to\_speak & -0.1683 & 0.3400 & -4.951813E-01 & 0.6211 \\ 
  \softhl bored\_with\_game & 0.5527 & 0.3084 & 1.792251E+00 & 0.0747 \\ 
  dislikes\_partner & 0.3457 & 0.3279 & 1.054410E+00 & 0.2931 \\ 
   \hline

  \hline
\PHONAVG & $\estslope$ & Std. Error & t value & Significance \\ 
  \hline
contributes\_to\_successful\_completion & 0.5557 & 0.3577 & 1.553747E+00 & 0.1220 \\ 
  making\_self\_clear & 0.7598 & 0.5085 & 1.494093E+00 & 0.1369 \\ 
  engaged\_in\_game & 0.2440 & 0.2586 & 9.438356E-01 & 0.3465 \\ 
  planning\_what\_to\_say & 0.3614 & 0.5174 & 6.984626E-01 & 0.4858 \\ 
  gives\_encouragement & 0.0604 & 0.3829 & 1.576928E-01 & 0.8749 \\ 
  \softhl difficult\_for\_partner\_to\_speak & -0.6264 & 0.3374 & -1.856257E+00 & 0.0650 \\ 
  bored\_with\_game & -0.0158 & 0.3204 & -4.921947E-02 & 0.9608 \\ 
  dislikes\_partner & 0.0975 & 0.3137 & 3.108070E-01 & 0.7563 \\ 
   \hline

\SYLAVG & $\estslope$ & Std. Error & t value & Significance \\ 
  \hline
  contributes\_to\_successful\_completion & 0.2451 & 0.3663 & 6.692398E-01 & 0.5042 \\ 
  \softhl making\_self\_clear & 0.7934 & 0.4094 & 1.937743E+00 & 0.0542 \\ 
  engaged\_in\_game & 0.4956 & 0.3642 & 1.360687E+00 & 0.1753 \\ 
  planning\_what\_to\_say & 0.4429 & 0.5189 & 8.535430E-01 & 0.3945 \\ 
  gives\_encouragement & 0.2363 & 0.4192 & 5.637211E-01 & 0.5736 \\ 
  difficult\_for\_partner\_to\_speak & 0.1856 & 0.3481 & 5.332959E-01 & 0.5945 \\ 
  bored\_with\_game & -0.2909 & 0.3606 & -8.067536E-01 & 0.4208 \\ 
  dislikes\_partner & 0.1768 & 0.3452 & 5.120454E-01 & 0.6092 \\ 
   \hline

  \hline
\LOCALJITTER & $\estslope$ & Std. Error & t value & Significance \\ 
  \hline
contributes\_to\_successful\_completion & 0.5770 & 0.3759 & 1.534821E+00 & 0.1265 \\ 
  making\_self\_clear & 0.5057 & 0.4881 & 1.036143E+00 & 0.3015 \\ 
  \hl engaged\_in\_game & 0.4972 & 0.2515 & 1.977130E+00 & 0.0495 \\ 
  planning\_what\_to\_say & -0.0417 & 0.4628 & -9.000210E-02 & 0.9284 \\ 
  gives\_encouragement & -0.0160 & 0.3502 & -4.554031E-02 & 0.9637 \\ 
  difficult\_for\_partner\_to\_speak & -0.2788 & 0.3126 & -8.917922E-01 & 0.3737 \\ 
  bored\_with\_game & 0.1233 & 0.3155 & 3.906725E-01 & 0.6965 \\ 
  dislikes\_partner & -0.1171 & 0.2788 & -4.198582E-01 & 0.6751 \\ 
  \hline
\LOCALSHIMMER & $\estslope$ & Std. Error & t value & Significance \\ 
  \hline
contributes\_to\_successful\_completion & 0.3745 & 0.2754 & 1.359709E+00 & 0.1756 \\ 
  making\_self\_clear & -0.0097 & 0.3821 & -2.544762E-02 & 0.9797 \\ 
  engaged\_in\_game & 0.2434 & 0.2881 & 8.449092E-01 & 0.3993 \\ 
  planning\_what\_to\_say & -0.6040 & 0.4735 & -1.275476E+00 & 0.2037 \\ 
  \softhl gives\_encouragement & 0.3638 & 0.2057 & 1.768094E+00 & 0.0787 \\ 
  difficult\_for\_partner\_to\_speak & 0.2707 & 0.2720 & 9.952034E-01 & 0.3209 \\ 
  bored\_with\_game & -0.3635 & 0.2772 & -1.311203E+00 & 0.1914 \\ 
  dislikes\_partner & -0.1895 & 0.2667 & -7.105564E-01 & 0.4783 \\ 
\end{tabular}}

\caption{Tablas con los resultados de la regresión de efectos fijos para \NOISETOHARMONICS, \SYLAVG, \PHONAVG, \LOCALSHIMMER y \LOCALJITTER. En la segunda columna se cita el valor de $\estslope$, la desviación estándar calculada, el t-valor obtenido y la significancia. Las columnas resaltadas corresponden a aquellas significantes, con diferentes matices de gris según $p < 0.10$, $p < 0.5$, o $p < 0.01$}

\label{fig:efectos_fijos_tabla2}
\end{figure}

\begin{figure}[ht]
\centering
% psl is "Positive Slope"
\newcommand{\psl} { $+$ }
% nsl stands for "Negative SLope"
\newcommand{\nsl} { $-$ }


\begin{tabular}{| c | c | c | c | c | c |}
  \hline
 & ENG\_MAX & ENG\_MEAN & F0\_MEAN & F0\_MAX & NOISERATIO  \\
  \hline
contributes  &      &  & \psl &  & \psl \\ \hline
  clear     & \psl &  & \psl &  & \psl \\ \hline
  engaged    &      &  & \psl &  &      \\ \hline
  planning   &      &  &      &  &      \\ \hline
  encourages &      &  &      &  &      \\ \hline
  difficult  & \nsl &  &      &  &      \\ \hline
  bored      &      &  & \nsl &  &      \\ \hline
  dislikes   &      &  &      &  &      \\ \hline
   \hline
\end{tabular}

\adjustbox{max width=\textwidth}{
\begin{tabular}{| c | c | c | c | c | c | c |}
  \hline
& PHON\_AVG & PHON\_COUNT & SHIMMER & SYL\_AVG & SYL\_COUNT & VCD2TOT \\
  \hline
contributes  &      &  &  &  &      &  \\ \hline
  clear     & \psl &  &  &  & \psl &  \\ \hline
  engaged    &      &  &  &  &      &  \\ \hline
  planning   &      &  &  &  &      &  \\ \hline
  encourages &      &  &  &  &      &  \\ \hline
  difficult  &      &  &  &  &      &  \\ \hline
  bored      &      &  &  &  &      &  \\ \hline
  dislikes   &      &  &  &  &      &  \\ \hline
  \hline
\end{tabular}
}

\caption{Tabla que representa los resultados significantes del experimento. En una de las entradas, tenemos los nombres abreviados de las variables sociales, y en la otra las variables a/p. El símbolo \psl representa valor significante y positivo de la pendiente de la regresión de efectos fijos, mientras que \nsl representa significante y negativo }

\label{sign_table}

\end{figure}

