\section{Modelo de Efectos Fijos}

\newcommand{\slopeestim}[1] { $\estslope \sim #1$ }

El modelo de efectos fijos agrega el concepto de heterogeneidad permitiendo que cada sujeto tenga su propio valor de ordenada al origen. En el caso concreto de nuestro corpus, dicha heterogeneidad puede deberse a multiplicidad de factores no medidos en él. Por ejemplo, la personalidad de los sujetos es un factor no medido y que puede influír en la dinámica de la interacción entre éstos.

Como el entrainment que medimos mediante el proceso TAMA es un proceso direccional (es decir, medimos tanto la influencia de un interlocutor sobre el otro y viceversa), definimos los ``grupos'' como las observaciones de entrainment y sus variables sociales de una sesión y el interlocutor sobre el cual consideramos la direccionalidad, tal y cual lo definimos en una sección anterior. Ésto nos arroja la cantidad de 24 grupos de observaciones



\section{Definición Formal de Efectos Fijos}

NOTA: Escribir ésto

\section{Resultados}

El modelo de efectos fijos sobre el valor absoluto del \emph{entrainment} dio valores sustancialmente apreciables. Las variables a/p \ENGMAX, \FOMEAN y \NOISETOHARMONICS poseen valores altamente significativos ( p-valor menor a 0.05) para la regresión con efectos fijos para al menos 2 variables sociales. En la tabla \ref{regresion_efectos_fijos_tabla} podemos ver la tabla del test de coeficientes con las variables sociales significativas resaltadas. En \ref{sign_table} podemos observar una tabla de doble entrada que simplifica la interpretación de las tablas de coeficientes, marcando aquellos pares de variables a/p y variables sociales con coeficientes significativos y su signo.




\begin{tabular}{rrrrr}
  \hline
 \ENGMAX & Estimate & Std. Error & t value & Pr($>$$|$t$|$) \\
  \hline
contributes\_to\_successful\_completion & 0.0497 & 0.4262 & 0.1165 & 0.9074 \\
  \myhighlight making\_self\_clear & 1.6581 & 0.3864 & 4.2909 & 0.0001 \\
  engaged\_in\_game & 0.3307 & 0.2576 & 1.2840 & 0.2008 \\
  planning\_what\_to\_say & 0.5005 & 0.5327 & 0.9395 & 0.3487 \\
  gives\_encouragement & 0.4264 & 0.3792 & 1.1246 & 0.2622 \\
  \myhighlight difficult\_for\_partner\_to\_speak & -0.7200 & 0.2858 & -2.5190 & 0.0126 \\
  bored\_with\_game & 0.2163 & 0.2560 & 0.8450 & 0.3992 \\
  dislikes\_partner & -0.4318 & 0.3443 & -1.2541 & 0.2114 \\
   \hline
\end{tabular}

\begin{tabular}{rrrrr}
  \hline
 \FOMEAN & Estimate & Std. Error & t value & Pr($>$$|$t$|$) \\
  \hline
 \myhighlight contributes\_to\_successful\_completion & 1.0274 & 0.3025 & 3.3962 & 0.0008 \\
  \myhighlight making\_self\_clear & 0.8307 & 0.3934 & 2.1115 & 0.0361 \\
  \myhighlight engaged\_in\_game & 0.8850 & 0.2750 & 3.2182 & 0.0015 \\
  planning\_what\_to\_say & 0.7167 & 0.5400 & 1.3273 & 0.1860 \\
  gives\_encouragement & 0.0075 & 0.3941 & 0.0190 & 0.9848 \\
  difficult\_for\_partner\_to\_speak & -0.5975 & 0.3928 & -1.5209 & 0.1300 \\
  \myhighlight bored\_with\_game & -0.7586 & 0.2481 & -3.0572 & 0.0026 \\
  dislikes\_partner & 0.0371 & 0.3800 & 0.0977 & 0.9223 \\
   \hline
\end{tabular}

\begin{tabular}{rrrrr}
  \hline
 \NOISETOHARMONICS & Estimate & Std. Error & t value & Pr($>$$|$t$|$) \\
  \hline
\myhighlight contributes\_to\_successful\_completion & 0.7041 & 0.3404 & 2.0686 & 0.0400 \\
  \myhighlight making\_self\_clear & 1.3344 & 0.3537 & 3.7725 & 0.0002 \\
  engaged\_in\_game & 0.0954 & 0.3462 & 0.2756 & 0.7832 \\
  planning\_what\_to\_say & -0.1874 & 0.4177 & -0.4485 & 0.6543 \\
  gives\_encouragement & 0.7234 & 0.4782 & 1.5127 & 0.1321 \\
  difficult\_for\_partner\_to\_speak & -0.1941 & 0.3436 & -0.5648 & 0.5729 \\
  \myhighlight bored\_with\_game & 0.5876 & 0.3028 & 1.9404 & 0.0539 \\
  dislikes\_partner & 0.3582 & 0.3330 & 1.0755 & 0.2835 \\
   \hline
\end{tabular}

\begin{figure}[ht]
\centering
% psl is "Positive Slope"
\newcommand{\psl} { $+$ }
\newcommand{\ppsl} { $++$ }
\newcommand{\pppsl} { $+++$ }

% nsl stands for "Negative SLope"
\newcommand{\nsl} { $-$ }
\newcommand{\nnsl} { $--$ }
\newcommand{\nsl} { $---$ }


\begin{tabular}{| c | c | c | c | c | c |}
  \hline
               &\ENGMAX  & \ENGMEAN  & \FOMEAN  & \FOMAX  & NOISERATIO  \\
  \hline
  contributes  &         &           & \psl     &         & \psl        \\ \hline
  clear        & \pppsl  &           & \psl     &         & \psl        \\ \hline
  engaged      &         &           & \psl     &         &             \\ \hline
  planning     &         &           &          &         &             \\ \hline
  encourages   &         &           &          &         &             \\ \hline
  difficult    & \nsl    &           &          &         &             \\ \hline
  bored        &         &           & \nsl     &         &             \\ \hline
  dislikes     &         &           &          &         &             \\ \hline
  \hline
& PHON\_AVG & PHON\_COUNT & SHIMMER & SYL\_AVG & SYL\_COUNT \\
  \hline
contributes  &      &  &  &  &        \\ \hline
  clear      & \psl &  &  &  & \psl   \\ \hline
  engaged    &      &  &  &  &        \\ \hline
  planning   &      &  &  &  &        \\ \hline
  encourages &      &  &  &  &        \\ \hline
  difficult  &      &  &  &  &        \\ \hline
  bored      &      &  &  &  &        \\ \hline
  dislikes   &      &  &  &  &        \\ \hline
  \hline
\end{tabular}


\caption{Tabla que representa los resultados significantes del experimento. En una de las entradas, tenemos los nombres abreviados de las variables sociales, y en la otra las variables a/p. El símbolo \psl representa valor significante y positivo de la pendiente de la regresión de efectos fijos, mientras que \nsl representa significante y negativo }

\label{sign_table}

\end{figure}

