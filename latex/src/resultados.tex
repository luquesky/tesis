\section{Modelo clásico}

\begin{figure}[t!]
\includegraphics[width=15cm]{images/regression_F0_MEAN_1.pdf}
\caption{Gráfico de los pares entrainment-variable a/p, junto a la regresión lineal obtenida \label{regresion_clasica} para \emph{F0\_MEAN}}
\end{figure}

En el modelo clásico no dio resultados apreciables, principalmente por la baja significancia de sus resultados. En \ref{regresion_clasica} puede verse el gráfico de \emph{F0\_MEAN} y 4 variables sociales y en \ref{regresion_clasica_tabla} pueden verse los valores de las estimaciones de $\widehat{\beta_2}$ junto a sus p-valores.

\begin{figure}[l]
% "ENG_MEAN"
% latex table generated in R 3.2.2 by xtable 1.8-0 package
% Thu Jan  7 03:01:56 2016
\begin{tabular}{rrrrr}
  \hline
 & Estimate & Std. Error & t value & Pr($>$$|$t$|$) \\
  \hline
bored\_with\_game & 10.65 & -0.44 & 0.00 & 0.66 \\
  difficult\_for\_partner\_to\_speak & 10.56 & -0.54 & 0.00 & 0.59 \\
  contributes\_to\_successful\_completion & 59.62 & -1.15 & 0.00 & 0.25 \\
  engaged\_in\_game & 73.14 & 0.55 & 0.00 & 0.59 \\
  gives\_encouragement & 47.49 & -0.04 & 0.00 & 0.97 \\
  making\_self\_clear & 52.97 & -0.99 & 0.00 & 0.33 \\
  planning\_what\_to\_say & 32.02 & -1.81 & 0.00 & 0.07 \\
  dislikes\_partner & 9.61 & -0.94 & 0.00 & 0.35 \\
   \hline
\end{tabular}

% "F0_MEAN"
% latex table generated in R 3.2.2 by xtable 1.8-0 package
% Thu Jan  7 03:01:56 2016
\begin{tabular}{rrrrr}
  \hline
 & Estimate & Std. Error & t value & Pr($>$$|$t$|$) \\
  \hline
bored\_with\_game & 10.63 & -0.18 & 0.00 & 0.86 \\
  difficult\_for\_partner\_to\_speak & 10.68 & -0.97 & 0.00 & 0.33 \\
  contributes\_to\_successful\_completion & 59.38 & 0.89 & 0.00 & 0.37 \\
  engaged\_in\_game & 73.41 & 0.77 & 0.00 & 0.44 \\
  gives\_encouragement & 47.59 & 1.09 & 0.00 & 0.28 \\
  making\_self\_clear & 52.91 & -0.32 & 0.00 & 0.75 \\
  planning\_what\_to\_say & 31.49 & 0.39 & 0.00 & 0.70 \\
  dislikes\_partner & 9.81 & -1.74 & 0.00 & 0.08 \\
   \hline
\end{tabular}

% "F0_MAX"
% latex table generated in R 3.2.2 by xtable 1.8-0 package
% Thu Jan  7 03:01:56 2016
\begin{tabular}{rrrrr}
  \hline
 & Estimate & Std. Error & t value & Pr($>$$|$t$|$) \\
  \hline
bored\_with\_game & 11.08 & 2.46 & 0.00 & 0.01 \\
  difficult\_for\_partner\_to\_speak & 10.65 & 0.52 & 0.00 & 0.60 \\
  contributes\_to\_successful\_completion & 60.08 & -0.98 & 0.00 & 0.33 \\
  engaged\_in\_game & 74.20 & -0.57 & 0.00 & 0.57 \\
  gives\_encouragement & 48.17 & -1.04 & 0.00 & 0.30 \\
  making\_self\_clear & 53.86 & -2.32 & 0.00 & 0.02 \\
  planning\_what\_to\_say & 31.86 & -1.05 & 0.00 & 0.30 \\
  dislikes\_partner & 9.65 & 0.97 & 0.00 & 0.33 \\
   \hline
\end{tabular}

% "NOISE_TO_HARMONICS_RATIO"
% latex table generated in R 3.2.2 by xtable 1.8-0 package
% Thu Jan  7 03:01:56 2016
\begin{tabular}{rrrrr}
  \hline
 & Estimate & Std. Error & t value & Pr($>$$|$t$|$) \\
  \hline
bored\_with\_game & 10.75 & -0.44 & 0.00 & 0.66 \\
  difficult\_for\_partner\_to\_speak & 10.64 & 0.16 & 0.00 & 0.87 \\
  contributes\_to\_successful\_completion & 60.34 & -0.75 & 0.00 & 0.45 \\
  engaged\_in\_game & 74.49 & -0.16 & 0.00 & 0.87 \\
  gives\_encouragement & 48.39 & -0.79 & 0.00 & 0.43 \\
  making\_self\_clear & 53.60 & -0.11 & 0.00 & 0.91 \\
  planning\_what\_to\_say & 31.99 & -0.02 & 0.00 & 0.99 \\
  dislikes\_partner & 9.62 & -1.39 & 0.00 & 0.17 \\
   \hline
\end{tabular}




\caption{Tablas con los resultados de la regresión clásica para ENG\_MEAN, ENG\_MAX, F0\_MEAN y F0\_MAX. En la segunda columna se cita el valor de $\widehat{\beta_2}$, la desviación estándar calculada, el t-valor obtenido y la significancia}\label{regresion_clasica_tabla}
\end{figure}



\section{Modelo de Efectos Fijos}

El modelo de efectos fijos sobre el valor absoluto del \emph{entrainment} dio valores sustancialmente más apreciables. \ENGMAX, \ENGMEAN, \FOMEAN y \NOISETOHARMONICS poseen valores altamente significativos ( p-valor menor a 0.1) para la regresión con efectos fijos para al menos 3 variables sociales.

\begin{figure}

\begin{tabular}{rrrrr}
  \hline
 \ENGMAX & Estimate & Std. Error & t value & Pr($>$$|$t$|$) \\
  \hline
contributes\_to\_successful\_completion & 0.0497 & 0.4262 & 0.1165 & 0.9074 \\
  \myhighlight making\_self\_clear & 1.6581 & 0.3864 & 4.2909 & 0.0001 \\
  engaged\_in\_game & 0.3307 & 0.2576 & 1.2840 & 0.2008 \\
  planning\_what\_to\_say & 0.5005 & 0.5327 & 0.9395 & 0.3487 \\
  gives\_encouragement & 0.4264 & 0.3792 & 1.1246 & 0.2622 \\
  \myhighlight difficult\_for\_partner\_to\_speak & -0.7200 & 0.2858 & -2.5190 & 0.0126 \\
  bored\_with\_game & 0.2163 & 0.2560 & 0.8450 & 0.3992 \\
  dislikes\_partner & -0.4318 & 0.3443 & -1.2541 & 0.2114 \\
   \hline
\end{tabular}

\begin{tabular}{rrrrr}
  \hline
 \FOMEAN & Estimate & Std. Error & t value & Pr($>$$|$t$|$) \\
  \hline
 \myhighlight contributes\_to\_successful\_completion & 1.0274 & 0.3025 & 3.3962 & 0.0008 \\
  \myhighlight making\_self\_clear & 0.8307 & 0.3934 & 2.1115 & 0.0361 \\
  \myhighlight engaged\_in\_game & 0.8850 & 0.2750 & 3.2182 & 0.0015 \\
  planning\_what\_to\_say & 0.7167 & 0.5400 & 1.3273 & 0.1860 \\
  gives\_encouragement & 0.0075 & 0.3941 & 0.0190 & 0.9848 \\
  difficult\_for\_partner\_to\_speak & -0.5975 & 0.3928 & -1.5209 & 0.1300 \\
  \myhighlight bored\_with\_game & -0.7586 & 0.2481 & -3.0572 & 0.0026 \\
  dislikes\_partner & 0.0371 & 0.3800 & 0.0977 & 0.9223 \\
   \hline
\end{tabular}

\begin{tabular}{rrrrr}
  \hline
 \NOISETOHARMONICS & Estimate & Std. Error & t value & Pr($>$$|$t$|$) \\
  \hline
\myhighlight contributes\_to\_successful\_completion & 0.7041 & 0.3404 & 2.0686 & 0.0400 \\
  \myhighlight making\_self\_clear & 1.3344 & 0.3537 & 3.7725 & 0.0002 \\
  engaged\_in\_game & 0.0954 & 0.3462 & 0.2756 & 0.7832 \\
  planning\_what\_to\_say & -0.1874 & 0.4177 & -0.4485 & 0.6543 \\
  gives\_encouragement & 0.7234 & 0.4782 & 1.5127 & 0.1321 \\
  difficult\_for\_partner\_to\_speak & -0.1941 & 0.3436 & -0.5648 & 0.5729 \\
  \myhighlight bored\_with\_game & 0.5876 & 0.3028 & 1.9404 & 0.0539 \\
  dislikes\_partner & 0.3582 & 0.3330 & 1.0755 & 0.2835 \\
   \hline
\end{tabular}

\caption{Tablas con los resultados de la regresión clásica para ENG\_MEAN, ENG\_MAX, F0\_MEAN y F0\_MAX. En la segunda columna se cita el valor de $\widehat{\beta_2}$, la desviación estándar calculada, el t-valor obtenido y la significancia}\label{regresion_clasica_tabla}

\end{figure}
