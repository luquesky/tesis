\section{Armado de Tabla}

Para condensar todos nuestros datos, armamos una tabla por cada variable a/p. Esta tabla contiene información definida para cada interlocutor, tarea y sesión de nuestro corpus.

\begin{enumerate}
  \item session: número de sesión
  \item speaker: 0 si corresponde al interlocutor A; B en otro caso
  \item task: número de tarea
  \item count: La cantidad de puntos definidos que tiene la serie
  \item entrainment: Si $speaker=0$, es $A\rightarrow B$; $B \rightarrow A$ en otro caso
  \item best\_lag: el lag del cross-correlogram donde se logra el \emph{entrainment}
  \item tama\_mean: el promedio de la variable
\end{enumerate}

Además, agregamos las variables sociales (relativas al interlocutor) para cada fila:

\begin{enumerate}
\item contributes\_to\_successful\_completion
\item making\_self\_clear
\item engaged\_in\_game
\item planning\_what\_to\_say
\item gives\_encouragement
\item difficult\_for\_partner\_to\_speak
\item bored\_with\_game
\item dislikes\_partner
\end{enumerate}

El corpus original cuenta con más variables pero éstas son las únicas que tomaremos en cuenta (citar algo acá!)

En el corpus original, cada variable estaba replicada por cada interlocutor, y por sí o por no, de manera que teníamos:

\begin{enumerate}
  \item $conversation\_awkward\_A_yes$
  \item $conversation\_awkward\_A_no$
  \item $conversation\_awkward\_B_yes$
  \item $conversation\_awkward\_B_no$
\end{enumerate}


Ésto nos da una tabla de 210 filas, y 21 columnas. Para cada sesión y speaker, podemos pensar que tenemos una serie de tiempo donde el tiempo es cada tarea, y los datos son el entrainment y las variables sociales. En la jerga econométrica, llamamos a este tipo de datos \emph{de panel}\cite{gujarati1999}: un conjunto de mediciones temporales sobre un mismo sujeto a lo largo del tiempo. En este caso el sujeto es un interlocutor en una sesión, el tiempo son las tareas, y las mediciones son los entrainments


\begin{figure}
\centering

\section{Panel de datos}
\label{sec:panel_data}

Luego de construir las series de tiempo para cada una de las conversaciones que seleccionamos anteriormente, pasamos a construir una gran tabla que se utilizó en los experimentos de regresión detallados en la siguiente sección. Para condensar todos nuestros datos, armamos una tabla por cada variable acústico-prosódicas, que contiene información definida para cada interlocutor y tarea de nuestro corpus.

Cada columna de esta tabla representa los datos de un interlocutor dentro de una tarea. Éste hecho lo usamos fuertemente a la hora de definir los grupos en nuestro modelo de Efectos Fijos. En la figura \ref{fig:panel_data} se describen las columnas generadas.

\begin{figure}[b!]
\centering
\adjustbox{max width=\textwidth}{
\begin{tabular}{|c|c|}
  \hline
  \myhighlight \emph{Campo} & \emph{Descripción} \\\hline
  session & número de sesión \\\hline
  speaker & 0 si corresponde al interlocutor A; B en otro caso \\\hline
  task & número de tarea \\\hline
  count & La cantidad de puntos definidos que tiene la serie \\\hline
  entrainment & Si $speaker=0$, es $\fwentrainment{AB}$; $\fwentrainment{BA}$ en otro caso \\\hline
  best\_lag & el lag del cross-correlogram donde se logra el \emph{entrainment} \\\hline

  engaged\_in\_game & ¿el sujeto parece comprometido con el juego? \\\hline
  difficult\_for\_partner\_to\_speak & ¿el sujeto dirige la conversación? \\\hline
  contributes\_to\_successful\_completion & ¿el sujeto contribuye para el éxito del equipo? \\\hline
  gives\_encouragement & ¿el sujeto alienta a su compañero?\\\hline
  making\_self\_clear & ¿el sujeto se expresa correctamente?\\\hline

  planning\_what\_to\_say & \\\hline
  bored\_with\_game & ¿el sujeto se muestra aburrido? \\\hline
  dislikes\_partner &  ¿al sujeto no le agrada su compañero? \\\hline
\end{tabular}}
\label{fig:panel_data}
\caption{Columnas de la tabla generada para ser utilizada en los experimentos de regresión lineal}
\end{figure}



La tabla generada tuvo una dimensión de 210 x 21, siendo 210 la cantidad de tareas (contadas dos veces por cada hablante) y 21 las columnas mencionadas en la figura \ref{fig:panel_data}. Una forma de ver ésta tabla es que, para cada sesión y hablante, tenemos una serie de tiempo sobre las tareas   siendo los datos el entrainment y las variables sociales. En la jerga econométrica, llamamos a este tipo de datos \emph{de panel}\cite{gujarati1999}: un conjunto de mediciones temporales sobre un mismo sujeto a lo largo del tiempo. En este caso el sujeto es un interlocutor en una sesión, el tiempo son las tareas, y las mediciones son los entrainments y las diferentes variables sociales.

En la figura \ref{fig:panel_data_example} tenemos una sección de la tabla. Los sujetos que tenemos en éste ejemplo son 3: $speaker = 0$ y $session=1$, $speaker = 1$ y $session=1$, y $speaker = 0$ y $session=2$. También tenemos cinco series de tiempo para cada sujeto: \entrainment, \emph{bored}, \emph{engaged}, \emph{encourages} y \emph{clear}. Vale la pena remarcar que estas series de tiempo, al igual que las que consideramos en la construcción de TAMA, pueden tener datos faltantes.


\begin{figure}
\centering
\adjustbox{max width=\textwidth}{
\begin{tabular}{lrrrrrrrrr}
\toprule
session &  speaker &  task &  entrainment &  bored\_with\_game &  engaged\_in\_game &  gives\_encouragement &  making\_self\_clear &  planning\_what\_to\_say \\
\midrule
1 &        0 &    10 &     0.581475 &                0 &                5 &                    5 &                  5\\
1 &        0 &    12 &    -0.569677 &                1 &                5 &                    5 &                  5\\
1 &        0 &    13 &     0.533701 &                2 &                4 &                    5 &                  4\\
1 &        1 &    10 &    -0.917101 &                0 &                5 &                    2 &                  3\\
1 &        1 &    12 &     0.467112 &                0 &                5 &                    4 &                  2\\
1 &        1 &    13 &    -0.602364 &                0 &                5 &                    4 &                  3\\
2 &        0 &     3 &     0.520696 &                0 &                4 &                    5 &                  5\\
2 &        0 &     4 &    -0.241060 &                0 &                5 &                    4 &                  4\\
2 &        0 &     7 &     0.743719 &                0 &                5 &                    4 &                  5\\
2 &        0 &     8 &     0.147362 &                0 &                5 &                    4 &                  2\\
\bottomrule
\end{tabular}
}


\caption{Ejemplo de tabla generada para $\FOMEAN$}
\label{fig:panel_data_example}
\end{figure}

\label{panel_data}
\end{figure}
