\section{Armado de Tabla}

Para condensar todos nuestros datos, armamos una tabla por cada variable a/p. Esta table contiene información definida para cada interlocutor, tarea y sesión de nuestro corpus.

\begin{itemize}
  \item session: número de sesión
  \item task: número de tarea
  \item speaker: 0 si corresponde al interlocutor A; B en otro caso
  \item count: La cantidad de puntos definidos que tiene la serie
  \item entrainment: Si $speaker=0$, es $A\rightarrow B$; $B \rightarrow A$ en otro caso
  \item best\_lag: el lag del cross-correlogram donde se logra el \emph{entrainment}
  \item tama\_mean: el promedio de la variable
\end{itemize}

Además, agregamos las variables sociales (relativas al interlocutor) para cada fila:

\begin{itemize}
  \item conversation\_awkward
  \item flow\_naturally
  \item hard\_time\_understanding\_each\_other
  \item contributes\_to\_successful\_completion
  \item making\_self\_clear
  \item engaged\_in\_game
  \item planning\_what\_to\_say
  \item gives\_encouragement
  \item difficult\_for\_partner\_to\_speak
  \item bored\_with\_game
  \item dislikes\_partner
\end{itemize}

En el corpus original, cada variable estaba replicada por cada interlocutor, y por sí o por no, de manera que teníamos:

\begin{enumerate}
  \item $conversation\_awkward\_A_yes$
  \item $conversation\_awkward$
  \item $conversation\_awkward$
  \item $conversation\_awkward$
\end{itemize}



Ésto nos da una tabla de 210 filas, y 21 columnas.
