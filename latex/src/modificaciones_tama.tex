\section{Modificaciones a TAMA}
\label{sec:tama_modifications}
Al método TAMA descripto en \ref{sec:ant_tama} le hemos aplicado algunas variaciones, que pasaremos a detallar.

En primer lugar, \cite{KOU2008.2} discute la disyuntiva de elegir un tamaño de ventana y step para el método: ventanas demasiado chicas pueden causar que no hayan segmentos de habla en ellas, mientras que un tamaño de ventana demasiado grande suavizaría en exceso la serie de tiempo. A colación de esto, menciona dos posibles soluciones para el problema de los puntos faltantes: interpolar (también mencionado en \cite{DEL2013}) o repetir el punto anterior de la serie.

Estos enfoques, sin embargo, pueden dar lugar a valores de \entrainment artificialmente altos por la construcción misma de la serie, ya que nos generaría puntos correlacionados fuertemente entre sí en cada una de las series de los hablantes. Por otro lado, descartar aquellas conversaciones que tengan puntos faltantes puede ser demasiado restrictivo y eliminar de nuestro corpus una gran cantidad de datos valiosos. Teniendo estas cosas en mente, decidimos aceptar series de tiempo con datos faltantes, que pueden ser producto de ventanas sin segmentos de habla o con algunos demasiado pequeños que imposibiliten la medición de las variables \ap: por ejemplo interjecciones o backchanneling (\emph{uh-huh} o \emph{hmmm} en inglés).

\section{Selección de Ventana}
\label{sec:window_selection}
En \cite{KOU2008.2} se menciona una elección de \emph{frame step} y \emph{frame length} de 10s y 20s respectivamente. En el caso de nuestro corpus, quisimos buscar los parámetros que mejor se ajustaban a éste, manteniendo la superposición del 50\% entre ventanas sucesivas. Con lo que nos queda que $FL = 2 * FS$

¿Qué queremos optimizar? La métrica que elegimos para ésto es encontrar un balance entre un frame no tan grande (para no suavizar en exceso la curva) y que nos reduzca considerablemente la cantidad de indefiniciones; es decir, aquellas ventanas que tomamos en un interlocutor que no tienen ninguna interacción de su parte. Para ver ésto, graficamos la cantidad de indefiniciones en función del step tomado.

\begin{figure}
\centering
\includegraphics[width=10cm]{images/window_selection.png}
\end{figure}



Dentro del rango de $FS \in \{5'',6'', \ldots ,15'' \}$, graficamos para cada sesión, tarea y cada interlocutor las curvas de indefiniciones. A su vez, para mayor claridad, graficamos una curva que promedie todas las tareas de una sesión.


Para tener una visión general de lo que ocurría en todas las sesiones, graficamos una curva promedio de todas las sesiones. En ésta puede observarse que hasta $8''-10''$ hay un fuerte descenso de las indefiniciones, que luego se atenúa. Dado que en general tenemos tareas cortas, preferimos tomar $8''$ como step, y $16''$ como largo de ventana.

OBS: podríamos cambiar ésto a un boxplot!

\begin{figure}
\centering
\includegraphics[height=5cm]{images/window_selection_for_session.png}
\end{figure}

