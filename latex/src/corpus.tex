\section{Columbia Game Corpus}

\begin{figure}[t]
\centering
\includegraphics[width=10cm]{images/columbia_games_card_game.png}
\caption{Juego del Columbia Games - Juego de Cartas}
\end{figure}


\newcommand{\cardgame} {\emph{Juego de cartas}}
\newcommand{\objectgame} {\emph{Juego de objetos}}


Nuestro corpus \cite{GRAV2009} consiste en doce conversaciones diádicas (i.e., con dos participantes) entre trece personas angloparlantes distintas. Todos los participantes reportaron hablar Inglés Americano Estándar, y no tener problemas de audición. La edad de los participantes se encuentra en el rango de los 20 a 50 años.

Las grabaciones se hicieron en 44 kHz, 16 bits con un canal separado para cada hablante; luego fueron guardadas en 16 kHz para el presente estudio. Cada sesión duró aproximadamente 45 minutos, totalizando 9 horas de
diálogos, 70.259 palabras (2.037 únicas) para todo el cuerpo de datos.

En cada sesión, se sentó a dos participantes (quienes no se conocían previamente) en una cabina profesional de grabación, cara a cara a ambos lados de una mesa, y con una cortina opaca colgando entre ellos para evitar la comunicación visual. Los participantes contaron con sendas computadoras portátiles conectadas entre sí, en las cuales jugaron una serie de juegos simples que requerían de comunicación verbal. El primero de ellos, un juego de cartas que consta de tres instancias que no consideramos en el presente estudio. Luego de ésto, pasaron al juego que analizamos.

\subsection{Juego de Objetos}

Luego de completar el juego de cartas, los sujetos siguieron interactuando a través del juego de objetos. En éste, la pantalla de cada jugador mostró un tablero con varios objetos, entre 5 y 7, como se ve en la figura \ref{objects_game}. Todos ellos se encuentran cercanos, con excepción de uno, a quien llamamos el Objetivo.

Para uno de los jugadores (el Descriptor) el Objetivo apareció en una posición aleatoria entre otros objetos. Para el otro jugador, a quien llamaremos el Seguidor, el Objetivo apareció en la parte baja de la pantalla. Entonces, al Descriptor se le encargó describir la posición del Objetivo de manera que el Seguidor pudiera mover su representación del objeto a la misma posición en su pantalla. Luego de una negociación entre ambos jugadores para decidir la mejor posición del objeto, se les asignó a los jugadores una puntuación entre 1 y 100 puntos de acuerdo a qué tan acertado fue el posicionamiento del Seguidor.

\begin{figure}[]
\centering
\includegraphics[width=10cm]{images/columbia_games.jpg}
\caption{Juego de objetos del Columbia Games}
\label{objects_game}
\end{figure}


Este juego tomó lugar durante 14 tareas. En las primeras cuatro, uno de los sujetos tomó el papel del Descriptor; en los siguientes cuatro invirtieron roles, y en los finales seis fueron alternando.

\subsection{Anotaciones sobre comportamiento social}

Varios aspectos del comportamiento de los jugadores durante los juegos de objetos fueron anotados mediante la herramienta de crowdsourcing \emph{Mechanical Turk}. Cada anotador escuchó un clip de un juego y tuvo que responder a las siguientes preguntas (para cada uno de los sujetos):

\begin{itemize}
  \item ¿el sujeto parece comprometido con el juego?
  \item ¿el sujeto dirige la conversación?
  \item ¿el sujeto contribuye para el éxito del equipo?
  \item ¿el sujeto alienta a su compañero?
  \item ¿el sujeto se expresa correctamente?
  \item ¿al sujeto no le agrada su compañero?
  \item ¿el sujeto le hace difícil hablar a su compañero?
  \item ¿el sujeto intenta acaparar la conversación?
\end{itemize}

entre otras. Cada uno de estos clips fue puntuado por cinco anotadores, que respondieron por sí o por no. El puntaje que recibe cada una de las preguntas (a las cuales llamaremos a partir de ahora \emph{variables sociales}) consiste en la cantidad de respuestas afirmativas que recibió, teniendo un rango de 0 a 5.

