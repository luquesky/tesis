En el presente trabajo, analizamos como dos métricas del \entrainment derivadas automáticamente de la conversación se relacionan con la percepciones sociales de terceros independientes del juego. Todo este análisis se da en el contexto de un juego orientado a tareas, que comprende interacciones de una naturaleza muy similar a las de una interfaz humano-computadora.

Estas métricas fueron construídas a través del análisis de series de tiempo, y apuntan a cuantificar cuánto se imitan o mimetizan los hablantes en términos de sus variables acústico-prosódicas. En primer lugar, contemplamos una métrica que penaliza el dis-entrainment con valores negativos. Se aplicó análisis de regresión sobre esta métrica, y los resultados no fueron significativos. En segundo lugar, construimos una métrica que valora de igual manera el \entrainment y el dis-\entrainment, de acuerdo a literatura que sugería que el segundo fenómeno puede considerarse en algunas circunstancias como un mecanismo de adaptación cooperativa. Al efectuar el análisis de regresión sobre ésta, los resultados fueron significativos y consistentes a la hipotesis planteada de que el \entrainment se relaciona positivamente con características sociales favorables de la conversación, mientras que lo hace de manera inversa con aquellas negativas.

Respecto a trabajos anteriores donde se construyen medidas del \entrainment prosódico, la métrica construída se comporta de manera consistente, manteniendo las relaciones entre aquellas variables sociales de carácter positivo y negativo. Esta métrica, además, se puede efectuar sin intervención manual a diferencia de aquellas que utilizan ToBI o anotaciones de otro tipo. A su vez, la cuantificación presentada evita el problema del alineamiento de turnos mediante la abstracción de éstos usando series de tiempo.

Una contribución importante de este trabajo es la validación de la métrica introducida en \cite{KOU2008.2}, dando indicios de que ésta guarda relación con la percepción social de la conversación. Igual de importante es el uso del valor absoluto de la correlación cruzada, como medida unificadora del \entrainment y \disentrainment y que remarca la importancia deĺ segundo fenómeno dentro de la comunicación verbal, a la luz de últimos trabajos acerca de la divergencia en el diálogo.

A pesar de que los resultados son prometedores, siguen siendo preliminares y la robustez de estos requieren más validaciones. Como trabajo a futuro, proponemos reproducir estos experimentos sobre otros corpus de habla. Adicionalmente, se debería verificar el impacto del proceso de pre-whitening, ya que un análisis a priori no mostró grandes diferencias entre usar o no este filtro. Otra dirección posible es evaluar un análisis multivariado de las diferentes variables \ap para construir una medida del \entrainment prosódico.
