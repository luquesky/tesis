En el presente trabajo, analizamos cómo dos métricas dinámicas del fenómeno conocido como \entrainment o mimetización en el plano acústico-prosódico se relacionan con las percepciones, por parte de terceros, de aspectos sociales de la interacción entre los participantes. Ambas métricas pueden computarse en forma automática a partir de las grabaciones de las conversaciones, con un hablante por canal, y con transcripciones  alineadas temporalmente al audio. Todo este análisis se da en el contexto de un juego orientado a tareas, que comprende interacciones de una naturaleza muy similar a las de una interfaz humano-computadora.

Estas métricas fueron construidas a través del análisis de series de tiempo, y apuntan a cuantificar cuánto se imitan o mimetizan los hablantes en términos de sus variables acústico-prosódicas. En primer lugar, contemplamos una métrica que penaliza el dis-entrainment con valores negativos. Se aplicó análisis de regresión sobre esta métrica, y los resultados que dio no fueron significativos. En segundo lugar, construimos una métrica que valora de igual manera el \entrainment y el dis-\entrainment, de acuerdo a trabajos previos que sugerían que el segundo fenómeno puede considerarse en algunas circunstancias como un mecanismo de adaptación cooperativa. Al efectuar el análisis de regresión sobre esta métrica, los resultados fueron significativos y consistentes con la hipótesis planteada de que el \entrainment se relaciona positivamente con características sociales favorables de la conversación, mientras que lo hace de manera inversa con aquellas negativas.

Respecto a trabajos anteriores que construyen medidas del \entrainment acústico-prosódico, la métrica usada en esta tesis se comporta de manera consistente preservando las relaciones expuestas en otros trabajos entre el \entrainment y aquellas variables sociales de carácter positivo y negativo. Esta métrica, además, se puede efectuar sin intervención manual, a diferencia de aquellas que utilizan ToBI o anotaciones de otro tipo. A su vez, la cuantificación presentada evita el problema del alineamiento de turnos mediante la abstracción de éstos usando series de tiempo.

Una contribución importante de este trabajo es la validación de la métrica introducida en \cite{KOU2008.2}, dando indicios de que ésta efectivamente captura rasgos relevantes de la interacción, que a su vez guardan relación con la percepción social de la conversación. Igual de importante es el uso del valor absoluto de la correlación cruzada, como medida unificadora del \entrainment y \disentrainment y que remarca la importancia deĺ segundo fenómeno dentro de la comunicación verbal, a la luz de últimos trabajos acerca de la divergencia en el diálogo.

A pesar de que los resultados son prometedores, siguen siendo preliminares y su robustez requiere de más validaciones. Como trabajo futuro, proponemos reproducir estos experimentos sobre otros corpus de habla, como por ejemplo Switchboard \footnote{https://catalog.ldc.upenn.edu/LDC97S62}. Adicionalmente, se debería verificar el impacto del proceso de pre-whitening, ya que un análisis preliminar no mostró grandes diferencias entre usar o no este filtro. Otra dirección posible es utilizar herramientas de análisis multivariado de series de tiempo sobre las diferentes variables \ap y sobre la base de esto construir nuevas métricas del \entrainment prosódico.
