
\section{Introducción}

Los sistemas de diálogo humano-computadora son cada vez más frecuentes, y sus aplicaciones comprenden una amplia gama de rubros: desde aplicaciones móviles, motores de búsqueda, juegos, o tecnologías de asistencia para ancianos y discapacitados.

Si bien es cierto que estos sistemas logran captar la dimensión lingüística de la comunicación humana, tienen un déficit importante a la hora de procesar y transmitir el aspecto superestructural de la comunicación, que radica en el intercambio de afecto, emociones, actitudes y otras intenciones de los participantes. La habilidad de los participantes de poder expresar, comprender, y reaccionar de acuerdo a estas señales sociales es necesaria para el entendimiento mutuo y una comunicación exitosa.

Un aspecto particular de la comunicación es el fenómeno de \emph{entrainment}(arrastre, mimetización, efecto camaleón), que comprende la adaptación inconsciente de las variables acústicas/prosódicas(a/p) (por ejemplo, el tono de la voz, la velocidad del habla, etc) de manera dinámica en el transcurso de una o varias interacciones. Este fenómeno ha sido introducido por \emph{Brennan et al}\cite{BRE1996} en 1996, y se ha observado que la convergencia de los participantes en estas variables ocurre en conjunto con una interacción más fluída y un mayor sentimiento de simpatía por sus interlocutores \cite{CHAR1999}.

Poder medir esta mimetización de los interlocutores no es una tarea fácil, sin embargo. En primer lugar, un diálogo no es una sucesión de turnos, sino que es una serie de tiempo dinámica, llena de interrupciones. Más aún, la mimetización no tiene un carácter instantáneo, sino que se sucede a lo largo de la interacción entre los participantes. Estos factores dificultan ostensiblemente poder modelar este fenómeno.
