\nota{Robar un poco de Pieraccini y Huerta - Where do we go from here}
\section{Sistemas de diálogo}

Los sistemas de diálogo humano-computadora son cada vez más frecuentes, y sus aplicaciones comprenden una amplia gama de rubros: desde aplicaciones móviles, motores de búsqueda, juegos, o tecnologías de asistencia para ancianos y discapacitados. Si bien es cierto que estos sistemas logran captar la dimensión lingüística de la comunicación humana, tienen un déficit importante a la hora de procesar y transmitir el aspecto superestructural de la comunicación oral, que radica en el intercambio de afecto, emociones, actitudes y otras intenciones de los participantes. Éste problema puede verse en cualquier sistema que interactúe sintetizando lenguaje humano: por ejemplo, las aplicaciones telefónicas que atienden automáticamente a sus clientes\cite{pieraccini2005, raux2006}. Stanley Kubrick y Arthur C. Clarke predijeron esto a la perfección, cuando en ``2001: Una Odisea en el Espacio''(1968) dotaron a \emph{HAL} de una voz monótona y robótica, casi lobotomizada. Otro problema grave que sufren estos sistemas humano-computadora es que asumen que sus interacciones de ``a turnos'', cuando las conversaciones entre humanos suelen distar bastante de ese modelo.

Dentro de las cualidades del lenguaje oral, una de las más distintivas es la \emph{prosodia}, qué es la dimensión que capta \emph{cómo} se dicen las cosas, en contraposición a \emph{qué} se está manifestando. Posee varias componentes acústico-prosódicas: por ejemplo, el tono o pitch, la intensidad o volumen, la calidad de la voz, la velocidad del habla y otras. Un manejo adecuado de estas componentes es lo que, hoy día, distingue una voz humana de una artificial. Esta carencia de habilidad sobre la prosodia conlleva cierta dificultad en la interacción con agentes conversaciones, que suelen ser calificados como ``mecánicos'' o ``extraños'' en su forma de comunicarse. \cite{raux2006, ward2005}

En pos de mejorar el entendimiento entre agentes conversacionales y sus usuarios, resulta de vital importancia poder entender y modelar las variaciones prosódicas de la comunicación oral. esto se traduciría tanto en una mejor apreciación de lo que quiere comunicar el usuario, como en una mayor naturalidad de la voz sintetizada por el agente.

\section{Mimetización}

En la literatura de Psicología del Comportamiento se ha observado con frecuencia que, bajo ciertas condiciones, cuando una persona mantiene una conversación, ésta modifica su manera de actuar aproximándola a la de su interlocutor. En una reseña de este tema se describe a este fenómeno como una ``imitación no consciente de posturas, maneras, expresiones faciales y otros comportamientos del compañero interaccional'' \cite[p. 893]{CHAR1999}  y conjeturan que es más fuerte en individuos con empatía disposicional. En otras palabras, personas con predisposición a buscar la aceptación social modifican su comportamiento en forma más marcada para aproximarlo a sus interlocutores

Esta modificación del comportamiento ha sido observada también en la manera de hablar. Por ejemplo, los interlocutores adoptan las mismas formas léxicas para referirse a las cosas, negociando tácitamente descripciones compartidas, en especial para cosas que resulten poco familiares \cite{BRE1996}. Estudios más recientes sugieren que esto también es cierto para el uso de estructuras sintácticas \cite{REI2006}. Este fenómeno subconsciente es conocido como mimetización, alineamiento, adaptación o convergencia, y también con el término inglés \entrainment, y se ha mostrado que juega un rol importante en la coordinación de diálogos, facilitando tanto la producción como la comprensión del habla en los seres humanos\cite{nenkova2008, gravano2015backward}. En nuestro caso, nos interesa principalmente el \entrainment de la prosodia.

\section{Midiendo la mimetización}

Muchos estudios han examinado la mimetización prosódica, listados en \cite{DEL2013}. Un número importante de ellos se han basado en la premisa de la mimetización como un fenómeno lineal, en el cual la convergencia ``va sucediendo'' a lo largo de la conversación \cite{burgoon1995interpersonal}. Estos estudios dividen las conversaciones en varias partes, y verifican que la diferencia absoluta entre los valores medios (de las variables \ap) y sus desviaciones se aproxime en las últimas partes de la interacción. Sin embargo, este enfoque de la mimetización niega su faceta dinámica: los interlocutores pueden estar inactivos y luego hablar, pueden pasar por varias etapas como escuchar, pensar, discutir un punto, etc. En \cite{levitan2011measuring} se reportó que éste es un fenómeno no sólamente lineal, sino también dinámico, donde los interlocutores van coincidiendo en el análisis por turnos.

Un problema común que surge a la hora de calcular estas métricas es el hecho de que las conversaciones no están alineadas en el tiempo, ni se dan en turnos de duración constante. Nos preguntamos entonces qué partes del diálogo de un hablante deberían compararse con qué otras partes de su par. Un enfoque de comparar interlocuciones uno a uno es demasiado simple y no captura situaciones de diálogo reales, mucho más dinámicas y con solapamiento casi constante.

Para atacar estos inconvenientes, utilizamos el método \TAMA(Time Aligned Moving Average) \cite{KOU2008}, que consiste en separar en ventanas de tiempo el diálogo, y promediar los valores de las variables prosódicas dentro de cada una. Este método es muy similar a aplicar un filtro de Promedio Móvil (Moving Average), lo que da el nombre a la técnica. Al separar el diálogo en ventanas de tiempo, podemos construir dos series de tiempo en base a cada interlocutor. Estas abstracciones son mucho más tratables que tener una secuencia de elocuciones de parte de cada hablante, y nos permiten efectuar análisis bien conocidos, uno de los cuáles nos permite construir una medida del \entrainment.

\section{Objetivo del estudio}

En el presente estudio, utilizaremos la técnica de \TAMA sobre un corpus de diálogo entre dos participantes angloparlantes, quienes interactúan mediante un juego a través de computadoras. El corpus ha sido anotado manualmente con variables que describen la percepción social de la conversación; por ejemplo: ¿el sujeto parece comprometido con el juego? ¿al sujeto no le agrada su compañero?.

Utilizaremos la técnica \TAMA para calcular el grado de \entrainment correspondiente en varias variables \ap, veremos si existe alguna relación significante entre los valores medidos y las percepciones sociales sobre las conversaciones. Uno esperaría, por la literatura previa, que valores altos de \entrainment se relacionen con valores altos de variables sociales positivas, tales como mostrarse colaborativo o compenetrado en la tarea. A su vez, compararemos los resultados obtenidos con otras medidas de \entrainment y validaremos la consistencia de la técnica del estudio con éstas.
