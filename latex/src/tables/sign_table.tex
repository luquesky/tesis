\begin{figure}[ht]
\centering
% psl is "Positive Slope"
\newcommand{\psl} { $+$ }
% nsl stands for "Negative SLope"
\newcommand{\nsl} { $-$ }


\begin{tabular}{| c | c | c | c | c | c |}
  \hline
 & ENG\_MAX & ENG\_MEAN & F0\_MEAN & F0\_MAX & NOISERATIO  \\
  \hline
contributes  &      &  & \psl &  & \psl \\ \hline
  clear     & \psl &  & \psl &  & \psl \\ \hline
  engaged    &      &  & \psl &  &      \\ \hline
  planning   &      &  &      &  &      \\ \hline
  encourages &      &  &      &  &      \\ \hline
  difficult  & \nsl &  &      &  &      \\ \hline
  bored      &      &  & \nsl &  &      \\ \hline
  dislikes   &      &  &      &  &      \\ \hline
  \hline
& PHON\_AVG & PHON\_COUNT & SHIMMER & SYL\_AVG & SYL\_COUNT \\
  \hline
contributes  &      &  &  &  &        \\ \hline
  clear      & \psl &  &  &  & \psl   \\ \hline
  engaged    &      &  &  &  &        \\ \hline
  planning   &      &  &  &  &        \\ \hline
  encourages &      &  &  &  &        \\ \hline
  difficult  &      &  &  &  &        \\ \hline
  bored      &      &  &  &  &        \\ \hline
  dislikes   &      &  &  &  &        \\ \hline
  \hline
\end{tabular}


\caption{Tabla que representa los resultados significantes del experimento. En una de las entradas, tenemos los nombres abreviados de las variables sociales, y en la otra las variables a/p. El símbolo \psl representa valor significante y positivo de la pendiente de la regresión de efectos fijos, mientras que \nsl representa significante y negativo }

\label{sign_table}

\end{figure}
