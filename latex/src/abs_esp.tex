%\begin{center}
%\large \bf \runtitulo
%\end{center}
%\vspace{1cm}
\chapter*{\runtitulo}

\noindent El \emph{entrainment} (mimetización) es un fenómeno inconsciente que se manifiesta a través de la adaptación de posturas, forma de hablar, gestos faciales y otros comportamientos entre dos o más interactores. A su vez, la ocurrencia de esta mimetización está fuertemente emparentada con el sentimiento de empatía y compenetración entre los participantes.

En esta tesis, nos proponemos explorar una técnica algorítmica para detectar el entrainment entre variables prosódicas de dos personas. Esta técnica nos permitirá determinar si existe o no convergencia para ciertos parámetros, y ver como está ésto correlacionado con variables sociales tales como la empatía, la compenetración con la tarea, y otras.

\bigskip

\noindent\textbf{Palabras claves:} Guerra, Rebelión, Wookie, Jedi, Fuerza, Imperio (no menos de 5).
