\chapter*{\runtitulo}

Los sistemas de diálogo humano-computadora son cada vez más frecuentes, y sus aplicaciones comprenden una amplia gama de rubros: desde aplicaciones móviles, motores de búsqueda, juegos, hasta tecnologías de asistencia para ancianos y discapacitados. Si bien es cierto que estos sistemas logran captar buena parte de la dimensión lingüística de la comunicación humana, tienen un déficit importante a la hora de procesar y transmitir el aspecto superestructural de la comunicación oral, que radica en el intercambio de afecto, emociones, actitudes y otras intenciones de los participantes.

El \emph{entrainment} (mimetización) es un fenómeno inconsciente que se manifiesta a través de la adaptación de posturas, forma de hablar, gestos faciales y otros comportamientos entre dos o más interactores. A su vez, la ocurrencia de esta mimetización está fuertemente emparentada con el sentimiento de empatía y compenetración entre los participantes. En nuestro caso, nos es de interés el \emph{entrainment} sobre las variables acústico-prosódicas, como el tono, intensidad, y otras.

En el presente trabajo, nos proponemos explorar y refinar una métrica del entrainment acústico-prosódico definida en trabajos previos. Analizamos la relación entre los valores obtenidos y las percepciones sociales que terceros tienen sobre las conversaciones, en un corpus de diálogos orientados a tareas en inglés.

\bigskip

\noindent\textbf{Palabras claves:} Procesamiento del Habla, Series de Tiempo, Entrainment, Regresión Lineal
