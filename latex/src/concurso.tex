
%%%%%%%%%%%%%%%%%%%%%%% file typeinst.tex %%%%%%%%%%%%%%%%%%%%%%%%%
%
% This is the LaTeX source for the instructions to authors using
% the LaTeX document class 'llncs.cls' for contributions to
% the Lecture Notes in Computer Sciences series.
% http://www.springer.com/lncs       Springer Heidelberg 2006/05/04
%
% It may be used as a template for your own input - copy it
% to a new file with a new name and use it as the basis
% for your article.
%
% NB: the document class 'llncs' has its own and detailed documentation, see
% ftp://ftp.springer.de/data/pubftp/pub/tex/latex/llncs/latex2e/llncsdoc.pdf
%
%%%%%%%%%%%%%%%%%%%%%%%%%%%%%%%%%%%%%%%%%%%%%%%%%%%%%%%%%%%%%%%%%%%


\documentclass[runningheads,a4paper]{llncs}


\nonstopmode


\usepackage{amsmath}
\usepackage{amssymb}
\usepackage{booktabs}
\usepackage[tables]{xcolor}
\usepackage{colortbl}
\usepackage{cite}
\usepackage{adjustbox}
\setcounter{tocdepth}{3}
\setcounter{secnumdepth}{3}
\usepackage{graphicx}
\usepackage[utf8]{inputenc}
\usepackage[spanish]{babel}
%\usepackage[numbers]{natbib}
\bibliographystyle{abbrv}
\usepackage{url}
\usepackage{xspace}


\newcommand{\keywords}[1]{\par\addvspace\baselineskip
\noindent\keywordname\enspace\ignorespaces#1}

%%% Macros
\newcommand{\absentrainment} {\emph{unsigned entrainment}\xspace}
\newcommand{\fwentrainment}[1] {\mathcal{E}_{#1}}

\newcommand{\entrainment} {\emph{entrainment}\xspace}
\newcommand{\disentrainment} {\emph{disentrainment}\xspace}
\newcommand{\ap} {acústico-prosódicas\xspace}

\newcommand{\TAMA} {\emph{TAMA}\xspace}
%%% Variables A/P
\newcommand{\apvar}[1]{\emph{#1}\xspace}

\newcommand{\ENGMAX} {\apvar{Int Max}}
\newcommand{\ENGMEAN} {\apvar{Int Mean}}
\newcommand{\FOMAX} {\apvar{F0 Max}}
\newcommand{\FOMEAN} {\apvar{F0 Mean}}
\newcommand{\NOISETOHARMONICS} {\apvar{NHR}}
\newcommand{\SYLAVG} {\apvar{Sílabas/seg}}
\newcommand{\PHONAVG} {\apvar{Fonemas/seg}}
\newcommand{\LOCALSHIMMER} {\apvar{Shimmer}}
\newcommand{\LOCALJITTER} {\apvar{Jitter}}


%%% variables sociales

\newcommand{\socialvariable}[1] {\emph{#1}\xspace}
\newcommand{\svcontributes} {\socialvariable{contributes-to-completion}}
\newcommand{\svclear} {\socialvariable{making-self-clear}}
\newcommand{\svengaged} {\socialvariable{engaged-with-game}}
\newcommand{\svplanning} {\socialvariable{planning-what-to-say}}
\newcommand{\svencourages} {\socialvariable{gives-encouragement}}
\newcommand{\svdifficult} {\socialvariable{difficult-for-partner-to-speak}}
\newcommand{\svbored} {\socialvariable{bored-with-game}}
\newcommand{\svdislikes} {\socialvariable{dislikes-partner}}

%%% Time series

\newcommand{\varnorm}[1] {
    (#1_t - \mu_{#1})
}

%%% Regresión and stuff
\newcommand{\estslope} {\widehat{\beta_2}}
\newcommand{\estintercept} {\widehat{\beta_1}}
\newcommand{\softhl} {\rowcolor[gray]{.88}}
\newcommand{\hl} {\rowcolor[gray]{.80}}
\newcommand{\stronghl} {\rowcolor[gray]{.75}}

% psl is "Positive Slope"
\newcommand{\psl} { $+$ }
\newcommand{\ppsl} { $++$ }
\newcommand{\pppsl} { $+++$ }

% nsl stands for "Negative SLope"
\newcommand{\nsl} { $-$ }
\newcommand{\nnsl} { $--$ }
\newcommand{\nnnsl} { $---$ }




\begin{document}

\mainmatter  % start of an individual contribution

% first the title is needed
\title{Métricas de mimetización acústico-prosódica en hablantes y su relación con rasgos sociales de diálogos}

% a short form should be given in case it is too long for the running head
\titlerunning{Métricas de mimetización acústico-prosódica en hablantes}

% the name(s) of the author(s) follow(s) next
%
% NB: Chinese authors should write their first names(s) in front of
% their surnames. This ensures that the names appear correctly in
% the running heads and the author index.
%
\author{Juan Manuel Pérez\\
\path|jmperez@dc.uba.ar|
}
%
\authorrunning{Juan Manuel Pérez}
% (feature abused for this document to repeat the title also on left hand pages)

% the affiliations are given next; don't give your e-mail address
% unless you accept that it will be published
\institute{Tesis de Licenciatura en Ciencias de la Computación \\
Facultad de Ciencias Exactas y Naturales \\
Universidad de Buenos Aires}

%
% NB: a more complex sample for affiliations and the mapping to the
% corresponding authors can be found in the file "llncs.dem"
% (search for the string "\mainmatter" where a contribution starts).
% "llncs.dem" accompanies the document class "llncs.cls".
%

\toctitle{Métricas de mimetización acústico-prosódica en hablantes}
\tocauthor{Juan Manuel Pérez}
\maketitle


\begin{abstract}
El \emph{entrainment} (mimetización) es un fenómeno inconsciente que se manifiesta a través de la adaptación de posturas, forma de hablar, gestos faciales y otros comportamientos entre dos o más interactores, fuertemente emparentado con el sentimiento de empatía entre los participantes. En el presente trabajo, nos proponemos explorar y refinar una métrica del entrainment acústico-prosódico definida en trabajos previos. Analizamos la relación entre los valores obtenidos y las percepciones sociales que terceros tienen sobre las conversaciones, en un corpus de diálogos orientados a tareas en inglés.
\keywords{Procesamiento del Habla, Series de Tiempo, Entrainment}
\end{abstract}


\section{Introducción}


\begin{frame}
  \frametitle{Sistemas de diálogo Humano-Computadora}

  \begin{enumerate}[<+->]
    \item Sistemas actuales
    \item Bien en la parte lingüística de la comunicación: entender y transmitir mensajes estructuralmente correctos.
    \item Mal en la parte superestructural: intercambio de emociones, actitudes, etc.
    \item El presente trabajo trata de hacer un (pequeño) aporte sobre el análisis de la ``naturalidad'' de las conversaciones.
  \end{enumerate}
\end{frame}


\begin{frame}
  \begin{columns}
    \column{0.5\textwidth}
    \frametitle{Ejemplos de ``falta de naturalidad''}
    \begin{enumerate}
      \item Sistemas de llamadas comerciales
      \item Siri, Google Now
      \item Otros?
    \end{enumerate}
    \column{0.5\textwidth}
    \begin{figure}
      \includegraphics[scale=0.25]{images/hal.jpg}
    \end{figure}
  \end{columns}
\end{frame}


\begin{frame}
  \frametitle{Prosodia}
  \framesubtitle{Definir acá qué es la prosodia aproximadamente}

\end{frame}



\begin{frame}
  \frametitle{Entrainment}

  \begin{enumerate}
    \item Fenómeno ubícuo en la comunicación
    \item Ocurre a varios niveles
    \item Largamente estudiado en psicología de comportamiento (referencias)
    \item (ya está, la próxima conversación que tengan afuera de acá van a chequear ésto)
  \end{enumerate}
\end{frame}

\begin{frame}
  \frametitle{¿Y cómo lo medimos?}
  La definición de entrainment hasta acá vista es heurística! ¿Cómo definimos una medida para esto?

  Vamos a explorar una métrica definida en trabajos anteriores, pulirla un poco, y verificar que efectivamente capture ciertas características del entrainment.

  ¿Cómo? Usando un corpus con anotaciones sociales


\end{frame}


\section{Materiales y Método}
\subsection{Corpus}
\begin{frame}
  \frametitle{Columbia Games Corpus}
  \framesubtitle{Descripción}
  \begin{columns}
    \column{0.35\textwidth}
      \begin{figure}
        \includegraphics[width=\textwidth]{images/columbia_games_color.jpg}
      \end{figure}

    \column{0.65\textwidth}

    \begin{itemize}
      \item Desarrollado por Agustín Gravano para su tesis doctoral.
      \item Corpus de conversaciones diádicas en Inglés Norteamericano
      \item 12 sesiones con 14 tareas/juegos cada una.
      \item En cada sesión, se sentó a dos participantes en una cabina profesional de grabación, y una cortina opaca colgando entre ellos para evitar la comunicación visual.
      \item Los participantes contaron con computadoras a través de las cuales interactuaban mediante juegos.
    \end{itemize}
  \end{columns}
\end{frame}


\begin{frame}
  \frametitle{Columbia Games Corpus}
  \framesubtitle{Anotaciones sociales}

  Cinco anotadores escucharon el audio correspondiente a una tarea del juego y respondieron a varias preguntas sobre los sujetos:


\begin{table}
\adjustbox{max width=0.8\textwidth}{
\begin{tabular}{|l|l|}
  \hline

  \textbf{Nombre} & \textbf{Pregunta} \\
  \hline

  \svcontributes &  ¿el sujeto contribuye para el éxito del equipo?  \\ \hline
  \svengaged &  ¿el sujeto parece comprometido con el juego? \\ \hline
  \svclear &  ¿el sujeto se expresa correctamente? \\ \hline
  \svplanning &  ¿el sujeto piensa lo que va a decir? \\ \hline
  \svencourages &  ¿el sujeto alienta a su compañero? \\ \hline
  \svdifficult &  ¿el sujeto le hace difícil hablar a su compañero? \\ \hline
  \svbored &  ¿el sujeto está aburrido con el juego? \\ \hline
  \svdislikes &  ¿al sujeto no le agrada su compañero? \\ \hline
\end{tabular}
}
\end{table}

De cada una de estas preguntas obtenemos un puntaje de 0 a 5, para cada hablante de cada tarea.
\end{frame}


\begin{frame}
\frametitle{Extracción de features acústico-prosódicas}

Usando el software Praat \footnote{http://www.fon.hum.uva.nl/praat/} se extrajeron las variables acústico-prosódicas para cada segmento de habla

\begin{table}
\adjustbox{max width=0.8\textwidth}{
\centering
\begin{tabular} {|c|c|}
  \hline
  Variable & Descripción \\
  \hline
  \FOMEAN & Valor medio de la frecuencia fundamental \\\hline
  \FOMAX  & Valor máximo de la frecuencia fundamental \\\hline
  \ENGMEAN & Valor medio de la intensidad \\\hline
  \ENGMAX & Valor máximo de la intensidad \\\hline
  \NOISETOHARMONICS & Noise-to-harmonics ratio \\\hline
  \LOCALSHIMMER & Shimmer medido \\\hline
  \LOCALJITTER  & Jitter medido \\\hline
  \SYLAVG & Cantidad de sílabas por segundo \\\hline
  \PHONAVG & Cantidad de fonemas por segundo \\\hline
\end{tabular}
}
\end{table}
\end{frame}



\subsection{Modificaciones a TAMA}

\begin{frame}
\frametitle{TAMA}
\framesubtitle{Nuestras modificaciones}

\begin{itemize}
  \item Usamos un step de 8s y un tamaño de ventana de 16s, manteniendo el solapamiento del 50\%.
  \item A diferencia del trabajo original de Kousidis, utilizamos series con datos faltantes.
  \item Pero sólo nos quedamos con aquellas que tengan 5 o más puntos definidos.
\end{itemize}
\end{frame}


\begin{frame}
\frametitle{TAMA}
\framesubtitle{Nuestra medida de mimetización}

  \begin{columns}
    \column{0.33\textwidth}
    \begin{figure}[t]
      \includegraphics[scale=0.28]{images/time_plot.png}
    \end{figure}

    \begin{figure}[t]
      \includegraphics[scale=0.28]{images/cross_correlogram_2.png}
    \end{figure}

    \column{0.66\textwidth}

    Definimos dos medidas de mimetización

    \begin{align*}
      \fwentrainment{AB}^{(1)} &= r_s \text{ con s maximizando } |r_k|,  k <= 0  \\
      \fwentrainment{BA}^{(1)} &= r_s \text{ con s maximizando } |r_k|,  k >= 0  \\
      \fwentrainment{XY}^{(2)} &= |\fwentrainment{XY}^{(1)}|
    \end{align*}

    Segunda métrica motivada por estudios sobre la antimimetización. Healey et al (2014) sugiere que puede ser una conducta de adaptación cooperativa.

    Levitan et al (2015) da más indicios en esa dirección.
  \end{columns}
\end{frame}

\subsection{Análisis de la relación con variables sociales}

\begin{frame}
\frametitle{Mimetización y relación con variables sociales}

  Para analizar la relación entre las variables sociales ($V$) y nuestras medidas de \emph{mimetización} ($\mathcal{E}$), planteamos un modelo de regresión lineal.

    \begin{equation}
      V_i \sim \beta_1 + \beta_2 \mathcal{E}_i
    \end{equation}

  Nuestra hipótesis es

  \begin{enumerate}
    \item Si $V$ es una variable de carácter positivo, entonces $\beta_2 > 0$
    \item Si $V$ es una variable de carácter negativo, entonces $\beta_2 < 0$
  \end{enumerate}
\end{frame}


\section{Resultados}
\section{Modelo clásico}

\begin{figure}[t!]
\includegraphics[width=15cm]{images/regression_F0_MEAN_1.pdf}
\caption{Gráfico de los pares entrainment-variable a/p, junto a la regresión lineal obtenida \label{regresion_clasica} para \emph{F0\_MEAN}}
\end{figure}

En el modelo clásico dio resultados con baja significancia. En \ref{regresion_clasica} puede verse el gráfico de \emph{F0\_MEAN} y 4 variables sociales y en \ref{regresion_clasica_tabla} pueden verse los valores de las estimaciones de $\estslope$ junto a sus p-valores.

\begin{figure}[l]
% "ENG_MEAN"
% latex table generated in R 3.2.2 by xtable 1.8-0 package
% Thu Jan  7 03:01:56 2016
\begin{tabular}{rrrrr}
  \hline
 & Estimate & Std. Error & t value & Pr($>$$|$t$|$) \\
  \hline
bored\_with\_game & 10.65 & -0.44 & 0.00 & 0.66 \\
  difficult\_for\_partner\_to\_speak & 10.56 & -0.54 & 0.00 & 0.59 \\
  contributes\_to\_successful\_completion & 59.62 & -1.15 & 0.00 & 0.25 \\
  engaged\_in\_game & 73.14 & 0.55 & 0.00 & 0.59 \\
  gives\_encouragement & 47.49 & -0.04 & 0.00 & 0.97 \\
  making\_self\_clear & 52.97 & -0.99 & 0.00 & 0.33 \\
  planning\_what\_to\_say & 32.02 & -1.81 & 0.00 & 0.07 \\
  dislikes\_partner & 9.61 & -0.94 & 0.00 & 0.35 \\
   \hline
\end{tabular}

% "F0_MEAN"
% latex table generated in R 3.2.2 by xtable 1.8-0 package
% Thu Jan  7 03:01:56 2016
\begin{tabular}{rrrrr}
  \hline
 & Estimate & Std. Error & t value & Pr($>$$|$t$|$) \\
  \hline
bored\_with\_game & 10.63 & -0.18 & 0.00 & 0.86 \\
  difficult\_for\_partner\_to\_speak & 10.68 & -0.97 & 0.00 & 0.33 \\
  contributes\_to\_successful\_completion & 59.38 & 0.89 & 0.00 & 0.37 \\
  engaged\_in\_game & 73.41 & 0.77 & 0.00 & 0.44 \\
  gives\_encouragement & 47.59 & 1.09 & 0.00 & 0.28 \\
  making\_self\_clear & 52.91 & -0.32 & 0.00 & 0.75 \\
  planning\_what\_to\_say & 31.49 & 0.39 & 0.00 & 0.70 \\
  dislikes\_partner & 9.81 & -1.74 & 0.00 & 0.08 \\
   \hline
\end{tabular}

% "F0_MAX"
% latex table generated in R 3.2.2 by xtable 1.8-0 package
% Thu Jan  7 03:01:56 2016
\begin{tabular}{rrrrr}
  \hline
 & Estimate & Std. Error & t value & Pr($>$$|$t$|$) \\
  \hline
bored\_with\_game & 11.08 & 2.46 & 0.00 & 0.01 \\
  difficult\_for\_partner\_to\_speak & 10.65 & 0.52 & 0.00 & 0.60 \\
  contributes\_to\_successful\_completion & 60.08 & -0.98 & 0.00 & 0.33 \\
  engaged\_in\_game & 74.20 & -0.57 & 0.00 & 0.57 \\
  gives\_encouragement & 48.17 & -1.04 & 0.00 & 0.30 \\
  making\_self\_clear & 53.86 & -2.32 & 0.00 & 0.02 \\
  planning\_what\_to\_say & 31.86 & -1.05 & 0.00 & 0.30 \\
  dislikes\_partner & 9.65 & 0.97 & 0.00 & 0.33 \\
   \hline
\end{tabular}

% "NOISE_TO_HARMONICS_RATIO"
% latex table generated in R 3.2.2 by xtable 1.8-0 package
% Thu Jan  7 03:01:56 2016
\begin{tabular}{rrrrr}
  \hline
 & Estimate & Std. Error & t value & Pr($>$$|$t$|$) \\
  \hline
bored\_with\_game & 10.75 & -0.44 & 0.00 & 0.66 \\
  difficult\_for\_partner\_to\_speak & 10.64 & 0.16 & 0.00 & 0.87 \\
  contributes\_to\_successful\_completion & 60.34 & -0.75 & 0.00 & 0.45 \\
  engaged\_in\_game & 74.49 & -0.16 & 0.00 & 0.87 \\
  gives\_encouragement & 48.39 & -0.79 & 0.00 & 0.43 \\
  making\_self\_clear & 53.60 & -0.11 & 0.00 & 0.91 \\
  planning\_what\_to\_say & 31.99 & -0.02 & 0.00 & 0.99 \\
  dislikes\_partner & 9.62 & -1.39 & 0.00 & 0.17 \\
   \hline
\end{tabular}




\caption{Tablas con los resultados de la regresión clásica para ENG\_MEAN, ENG\_MAX, F0\_MEAN y F0\_MAX. En la segunda columna se cita el valor de $\estslope$, la desviación estándar calculada, el t-valor obtenido y la significancia}\label{regresion_clasica_tabla}
\end{figure}



\section{Modelo de Efectos Fijos}

\newcommand{\slopeestim}[1] { $\estslope \sim #1$ }

El modelo de efectos fijos sobre el valor absoluto del \emph{entrainment} dio valores sustancialmente más apreciables. \ENGMAX, \FOMEAN y \NOISETOHARMONICS poseen valores altamente significativos ( p-valor menor a 0.05) para la regresión con efectos fijos para al menos 2 variables sociales.

En la tabla \ref{regresion_efectos_fijos_tabla} podemos ver estos valores con las variables sociales significativas resaltadas.


\begin{figure}

\begin{tabular}{rrrrr}
  \hline
 \ENGMAX & Estimate & Std. Error & t value & Pr($>$$|$t$|$) \\
  \hline
contributes\_to\_successful\_completion & 0.0497 & 0.4262 & 0.1165 & 0.9074 \\
  \myhighlight making\_self\_clear & 1.6581 & 0.3864 & 4.2909 & 0.0001 \\
  engaged\_in\_game & 0.3307 & 0.2576 & 1.2840 & 0.2008 \\
  planning\_what\_to\_say & 0.5005 & 0.5327 & 0.9395 & 0.3487 \\
  gives\_encouragement & 0.4264 & 0.3792 & 1.1246 & 0.2622 \\
  \myhighlight difficult\_for\_partner\_to\_speak & -0.7200 & 0.2858 & -2.5190 & 0.0126 \\
  bored\_with\_game & 0.2163 & 0.2560 & 0.8450 & 0.3992 \\
  dislikes\_partner & -0.4318 & 0.3443 & -1.2541 & 0.2114 \\
   \hline
\end{tabular}

\begin{tabular}{rrrrr}
  \hline
 \FOMEAN & Estimate & Std. Error & t value & Pr($>$$|$t$|$) \\
  \hline
 \myhighlight contributes\_to\_successful\_completion & 1.0274 & 0.3025 & 3.3962 & 0.0008 \\
  \myhighlight making\_self\_clear & 0.8307 & 0.3934 & 2.1115 & 0.0361 \\
  \myhighlight engaged\_in\_game & 0.8850 & 0.2750 & 3.2182 & 0.0015 \\
  planning\_what\_to\_say & 0.7167 & 0.5400 & 1.3273 & 0.1860 \\
  gives\_encouragement & 0.0075 & 0.3941 & 0.0190 & 0.9848 \\
  difficult\_for\_partner\_to\_speak & -0.5975 & 0.3928 & -1.5209 & 0.1300 \\
  \myhighlight bored\_with\_game & -0.7586 & 0.2481 & -3.0572 & 0.0026 \\
  dislikes\_partner & 0.0371 & 0.3800 & 0.0977 & 0.9223 \\
   \hline
\end{tabular}

\begin{tabular}{rrrrr}
  \hline
 \NOISETOHARMONICS & Estimate & Std. Error & t value & Pr($>$$|$t$|$) \\
  \hline
\myhighlight contributes\_to\_successful\_completion & 0.7041 & 0.3404 & 2.0686 & 0.0400 \\
  \myhighlight making\_self\_clear & 1.3344 & 0.3537 & 3.7725 & 0.0002 \\
  engaged\_in\_game & 0.0954 & 0.3462 & 0.2756 & 0.7832 \\
  planning\_what\_to\_say & -0.1874 & 0.4177 & -0.4485 & 0.6543 \\
  gives\_encouragement & 0.7234 & 0.4782 & 1.5127 & 0.1321 \\
  difficult\_for\_partner\_to\_speak & -0.1941 & 0.3436 & -0.5648 & 0.5729 \\
  \myhighlight bored\_with\_game & 0.5876 & 0.3028 & 1.9404 & 0.0539 \\
  dislikes\_partner & 0.3582 & 0.3330 & 1.0755 & 0.2835 \\
   \hline
\end{tabular}

\caption{Tablas con los resultados de la regresión de efectos fijos para \ENGMAX, \FOMEAN y \NOISETOHARMONICS. En la segunda columna se cita el valor de $\estslope$, la desviación estándar calculada, el t-valor obtenido y la significancia. Las columnas resaltadas corresponden a aquellas significantes}\label{regresion_efectos_fijos_tabla}
\end{figure}


\section{Conclusiones y trabajo futuro}
\begin{frame}
\frametitle{Conclusiones}

\begin{enumerate}
  \item Desarrollo de métrica de entrainment automática a partir de conversaciones transcritas.
  \item Indicios de validación de la métrica introducida por Kousidis et al en un corpus orientado a tareas.
  \item Más indicios sobre la prevalencia y característica positiva del disentrainment
\end{enumerate}


\end{frame}

\begin{frame}
\frametitle{Trabajo a futuro}

\begin{enumerate}
  \item Reproducir experimentos sobre otros corpus, por ejemplo Switchboard \footnote{https://catalog.ldc.upenn.edu/LDC97S62}
  \item Chequear filtro de prewhitening
  \item Análisis multivariado de las variables \ap para construir nuevas métricas de entrainment
\end{enumerate}
\end{frame}



\bibliography{concurso}


\end{document}
