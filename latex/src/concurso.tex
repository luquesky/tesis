
%%%%%%%%%%%%%%%%%%%%%%% file typeinst.tex %%%%%%%%%%%%%%%%%%%%%%%%%
%
% This is the LaTeX source for the instructions to authors using
% the LaTeX document class 'llncs.cls' for contributions to
% the Lecture Notes in Computer Sciences series.
% http://www.springer.com/lncs       Springer Heidelberg 2006/05/04
%
% It may be used as a template for your own input - copy it
% to a new file with a new name and use it as the basis
% for your article.
%
% NB: the document class 'llncs' has its own and detailed documentation, see
% ftp://ftp.springer.de/data/pubftp/pub/tex/latex/llncs/latex2e/llncsdoc.pdf
%
%%%%%%%%%%%%%%%%%%%%%%%%%%%%%%%%%%%%%%%%%%%%%%%%%%%%%%%%%%%%%%%%%%%


\documentclass[runningheads,a4paper]{llncs}

\usepackage{amssymb}
\setcounter{tocdepth}{3}
\usepackage{graphicx}
\usepackage[utf8]{inputenc}
\usepackage[spanish]{babel}
\usepackage{url}
\usepackage{xspace}


\urldef{\mailsa}\path|{alfred.hofmann, ursula.barth, ingrid.haas, frank.holzwarth,|
\urldef{\mailsb}\path|anna.kramer, leonie.kunz, christine.reiss, nicole.sator,|
\urldef{\mailsc}\path|erika.siebert-cole, peter.strasser, lncs}@springer.com|    
\newcommand{\keywords}[1]{\par\addvspace\baselineskip
\noindent\keywordname\enspace\ignorespaces#1}

%%% Macros
\newcommand{\absentrainment} {\emph{absolute entrainment}\xspace}
\newcommand{\fwentrainment}[1] {\mathcal{E}_{#1}}

\newcommand{\entrainment} {\emph{entrainment}\xspace}
\newcommand{\disentrainment} {\emph{disentrainment}\xspace}

\newcommand{\TAMA} {\emph{TAMA}\xspace}
%%% Variables A/P
\newcommand{\ENGMAX} {ENG\_MAX\xspace}
\newcommand{\ENGMEAN} {ENG\_MEAN\xspace}
\newcommand{\FOMAX} {F0\_MAX\xspace}
\newcommand{\FOMEAN} {F0\_MEAN\xspace}
\newcommand{\TOTFRAMES} {VCD2TOT\_FRAMES\xspace}
\newcommand{\NOISETOHARMONICS} {NOISE\_TO\_HARMONICS\_RATIO\xspace}
\newcommand{\SYLCOUNT} {SYLLABES\_COUNT\xspace}
\newcommand{\SYLAVG} {SYLLABES\_COUNT\xspace}
\newcommand{\LOCALSHIMMER} {SOUND\_VOICED\_LOCAL\_SHIMMER\xspace}
\newcommand{\PHONCOUNT} {PHONEMES\_COUNT\xspace}
\newcommand{\PHONAVG} {PHONEMES\_AVERAGE\xspace}


%%% variables sociales

\newcommand{\socialvariable}[1] {\emph{#1}}
\newcommand{\svcontributes} {\socialvariable{contributes-to-completion}}
\newcommand{\svclear} {\socialvariable{making-self-clear}}
\newcommand{\svengaged} {\socialvariable{engaged-with-game}}
\newcommand{\svplanning} {\socialvariable{planning-what-to-say}}
\newcommand{\svencourages} {\socialvariable{gives-encouragement}}
\newcommand{\svdifficult} {\socialvariable{difficult-for-partner-to-speak}}
\newcommand{\svbored} {\socialvariable{bored-with-game}}
\newcommand{\svdislikes} {\socialvariable{dislikes-partner}}

%%% Regresión and stuff
\newcommand{\estslope} {\widehat{\beta_2}}
\newcommand{\myhighlight} {\rowcolor[gray]{.75}}

%%% Nota
\newcommand{\nota}[1]{\todo{#1}}




\begin{document}

\mainmatter  % start of an individual contribution

% first the title is needed
\title{Métricas de mimetización acústico-prosódica en hablantes y su relación con rasgos sociales de diálogos}

% a short form should be given in case it is too long for the running head
\titlerunning{Lecture Notes in Computer Science: Authors' Instructions}

% the name(s) of the author(s) follow(s) next
%
% NB: Chinese authors should write their first names(s) in front of
% their surnames. This ensures that the names appear correctly in
% the running heads and the author index.
%
\author{Alfred Hofmann%
\thanks{Please note that the LNCS Editorial assumes that all authors have used
the western naming convention, with given names preceding surnames. This determines
the structure of the names in the running heads and the author index.}%
\and Ursula Barth\and Ingrid Haas\and Frank Holzwarth\and\\
Anna Kramer\and Leonie Kunz\and Christine Rei\ss\and\\
Nicole Sator\and Erika Siebert-Cole\and Peter Stra\ss er}
%
\authorrunning{Lecture Notes in Computer Science: Authors' Instructions}
% (feature abused for this document to repeat the title also on left hand pages)

% the affiliations are given next; don't give your e-mail address
% unless you accept that it will be published
\institute{Springer-Verlag, Computer Science Editorial,\\
Tiergartenstr. 17, 69121 Heidelberg, Germany\\
\mailsa\\
\mailsb\\
\mailsc\\
\url{http://www.springer.com/lncs}}

%
% NB: a more complex sample for affiliations and the mapping to the
% corresponding authors can be found in the file "llncs.dem"
% (search for the string "\mainmatter" where a contribution starts).
% "llncs.dem" accompanies the document class "llncs.cls".
%

\toctitle{Lecture Notes in Computer Science}
\tocauthor{Authors' Instructions}
\maketitle


\begin{abstract}
Los sistemas de diálogo humano-computadora son cada vez más frecuentes, y sus aplicaciones comprenden una amplia gama de rubros: desde aplicaciones móviles, motores de búsqueda, juegos o tecnologías de asistencia para ancianos y discapacitados. Si bien es cierto que estos sistemas logran captar buena parte de la dimensión lingüística de la comunicación humana, tienen un déficit importante a la hora de procesar y transmitir el aspecto superestructural de la comunicación oral, que radica en el intercambio de afecto, emociones, actitudes y otras intenciones de los participantes.

El \emph{entrainment} (mimetización) es un fenómeno inconsciente que se manifiesta a través de la adaptación de posturas, forma de hablar, gestos faciales y otros comportamientos entre dos o más interactores. A su vez, la ocurrencia de esta mimetización está fuertemente emparentada con el sentimiento de empatía y compenetración entre los participantes. En nuestro caso, nos es de interés el \emph{entrainment} sobre las variables acústico-prosódicas, como el tono, intensidad, y otras.

En el presente trabajo, nos proponemos explorar y refinar una métrica del entrainment acústico-prosódico definida en trabajos previos. Analizamos la relación entre los valores obtenidos y las percepciones sociales que terceros tienen sobre las conversaciones, en un corpus de diálogos orientados a tareas en inglés.
\keywords{Procesamiento del Habla, Series de Tiempo, Entrainment, Regresión Linea}
\end{abstract}


\section{Introduction}



Los sistemas de diálogo humano-computadora son cada vez más frecuentes, y sus aplicaciones comprenden una amplia gama de rubros: desde aplicaciones móviles, motores de búsqueda, juegos o tecnologías de asistencia para ancianos y discapacitados. Si bien es cierto que estos sistemas logran captar la dimensión lingüística de la comunicación humana, tienen un déficit importante a la hora de procesar y transmitir el aspecto superestructural de la comunicación oral, que radica en el intercambio de afecto, emociones, actitudes y otras intenciones de los participantes. Este problema puede verse en cualquier sistema que interactúe sintetizando lenguaje humano: por ejemplo, las aplicaciones telefónicas que atienden automáticamente a sus clientes \cite{pieraccini2005,raux2006}. Stanley Kubrick y Arthur C. Clarke predijeron esto a la perfección, cuando en ``2001: Una Odisea en el Espacio''(1968) dotaron a \emph{HAL} de una voz monótona y robótica, casi lobotomizada. Otro problema grave que sufren estos sistemas humano-computadora es que asumen que sus interacciones de ``a turnos'', cuando las conversaciones entre humanos suelen distar bastante de ese modelo.

Dentro de las cualidades del lenguaje oral, una de las más distintivas es la \emph{prosodia}, qué es la dimensión que capta \emph{cómo} se dicen las cosas, en contraposición a \emph{qué} se está manifestando. Posee varias componentes acústico-prosódicas: por ejemplo, el tono o pitch, la intensidad o volumen, la calidad de la voz, la velocidad del habla y otras. Un manejo adecuado de estas componentes es lo que, hoy día, distingue una voz humana de una artificial. Esta carencia de habilidad sobre la prosodia conlleva cierta dificultad en la interacción con agentes conversaciones, que suelen ser calificados como ``mecánicos'' o ``extraños'' en su forma de comunicarse. \cite{raux2006,ward2005}

En pos de mejorar el entendimiento entre agentes conversacionales y sus usuarios, resulta de vital importancia poder entender y modelar las variaciones prosódicas de la comunicación oral. Esto se traduciría tanto en una mejor apreciación de lo que quiere comunicar el usuario, como en una mayor naturalidad de la voz sintetizada por el agente.

\subsection{Mimetización}

En la literatura de Psicología del Comportamiento se ha observado con frecuencia que, bajo ciertas condiciones, cuando una persona mantiene una conversación, ésta modifica su manera de actuar aproximándola a la de su interlocutor. En una reseña de este tema se describe a este fenómeno como una ``imitación no consciente de posturas, maneras, expresiones faciales y otros comportamientos del compañero interaccional'' \cite[p. 893]{CHAR1999}  y conjeturan que es más fuerte en individuos con empatía disposicional. En otras palabras, personas con predisposición a buscar la aceptación social modifican su comportamiento en forma más marcada para aproximarlo a sus interlocutores

Esta modificación del comportamiento ha sido observada también en la manera de hablar. Por ejemplo, los interlocutores adoptan las mismas formas léxicas para referirse a las cosas, negociando tácitamente descripciones compartidas, en especial para cosas que resulten poco familiares \cite{BRE1996}. Estudios más recientes sugieren que esto también es cierto para el uso de estructuras sintácticas \cite{REI2006}. Este fenómeno subconsciente es conocido como mimetización, alineamiento, adaptación o convergencia y también con el término inglés \entrainment. Se ha mostrado que juega un rol importante en la coordinación de diálogos, facilitando tanto la producción como la comprensión del habla en los seres humanos\cite{nenkova2008,gravano2015backward}. En nuestro caso, nos interesa principalmente el \entrainment de la prosodia.

\subsection{Midiendo la mimetización}

Muchos estudios han examinado la mimetización prosódica, listados en \cite{DEL2013}. Un número importante de ellos se han basado en la premisa de la mimetización como un fenómeno lineal, en el cual la convergencia ``va sucediendo'' a lo largo de la conversación \cite{burgoon1995interpersonal}. Estos estudios dividen las conversaciones en varias partes, y verifican que la diferencia absoluta entre los valores medios (de las variables \ap) y sus desviaciones se aproxime en las últimas partes de la interacción. Sin embargo, este enfoque de la mimetización niega su faceta dinámica: los interlocutores pueden estar inactivos y luego hablar, pueden pasar por varias etapas como escuchar, pensar, discutir un punto, etc. En \cite{levitan2011measuring} se reportó que éste es un fenómeno no sólamente lineal, sino también dinámico, donde los interlocutores van coincidiendo en el análisis por turnos.

Un problema común que surge a la hora de calcular estas métricas es el hecho de que las conversaciones no están alineadas en el tiempo, ni se dan en turnos de duración constante. Nos preguntamos entonces qué partes del diálogo de un hablante deberían compararse con qué otras partes de su par. Un enfoque de comparar interlocuciones uno a uno es demasiado simple y no captura situaciones de diálogo reales, mucho más dinámicas y con solapamiento casi constante.

Para atacar estos inconvenientes, utilizamos el método \TAMA(Time Aligned Moving Average) \cite{KOU2008}, que consiste en separar en ventanas de tiempo el diálogo, y promediar los valores de las variables prosódicas dentro de cada una. Este método es muy similar a aplicar un filtro de Promedio Móvil (Moving Average), lo que da el nombre a la técnica. Al separar el diálogo en ventanas de tiempo, podemos construir dos series de tiempo en base a cada interlocutor. Estas abstracciones son mucho más tratables que tener una secuencia de elocuciones de parte de cada hablante, y nos permiten efectuar análisis bien conocidos, uno de los cuáles nos permite construir una medida del \entrainment.

\subsection{Objetivo del estudio}

En el presente estudio, aplicamos la técnica de \TAMA para definir dos métricas de \entrainment. Utilizamos un corpus de diálogo entre dos participantes angloparlantes, quienes interactúan mediante un juego a través de computadoras. El corpus ha sido anotado manualmente con variables que describen la percepción social de la conversación; por ejemplo: ¿el sujeto parece comprometido con el juego? ¿al sujeto no le agrada su compañero?

Luego, veremos si existe, para cada una de las variables \ap,  alguna relación significante entre las métricas definidas y las percepciones sociales sobre las conversaciones. Uno esperaría que valores altos de nuestras métricas del \entrainment se relacionen con valores altos de variables sociales positivas, tales como mostrarse colaborativo o compenetrado en la tarea. 



\subsection{Checking the PDF File}

Kindly assure that the Contact Volume Editor is given the name and email
address of the contact author for your paper. The Contact Volume Editor
uses these details to compile a list for our production department at
SPS in India. Once the files have been worked upon, SPS sends a copy of
the final pdf of each paper to its contact author. The contact author is
asked to check through the final pdf to make sure that no errors have
crept in during the transfer or preparation of the files. This should
not be seen as an opportunity to update or copyedit the papers, which is
not possible due to time constraints. Only errors introduced during the
preparation of the files will be corrected.

This round of checking takes place about two weeks after the files have
been sent to the Editorial by the Contact Volume Editor, i.e., roughly
seven weeks before the start of the conference for conference
proceedings, or seven weeks before the volume leaves the printer's, for
post-proceedings. If SPS does not receive a reply from a particular
contact author, within the timeframe given, then it is presumed that the
author has found no errors in the paper. The tight publication schedule
of LNCS does not allow SPS to send reminders or search for alternative
email addresses on the Internet.

In some cases, it is the Contact Volume Editor that checks all the final
pdfs. In such cases, the authors are not involved in the checking phase.

\subsection{Additional Information Required by the Volume Editor}

If you have more than one surname, please make sure that the Volume Editor
knows how you are to be listed in the author index.

\subsection{Copyright Forms}

The copyright form may be downloaded from the ``For Authors"
(Information for LNCS Authors) section of the LNCS Website:
\texttt{www.springer.com/lncs}. Please send your signed copyright form
to the Contact Volume Editor, either as a scanned pdf or by fax or by
courier. One author may sign on behalf of all of the other authors of a
particular paper. Digital signatures are acceptable.

\section{Paper Preparation}

Springer provides you with a complete integrated \LaTeX{} document class
(\texttt{llncs.cls}) for multi-author books such as those in the LNCS
series. Papers not complying with the LNCS style will be reformatted.
This can lead to an increase in the overall number of pages. We would
therefore urge you not to squash your paper.

Please always cancel any superfluous definitions that are
not actually used in your text. If you do not, these may conflict with
the definitions of the macro package, causing changes in the structure
of the text and leading to numerous mistakes in the proofs.

If you wonder what \LaTeX{} is and where it can be obtained, see the
``\textit{LaTeX project site}'' (\url{http://www.latex-project.org})
and especially the webpage ``\textit{How to get it}''
(\url{http://www.latex-project.org/ftp.html}) respectively.

When you use \LaTeX\ together with our document class file,
\texttt{llncs.cls},
your text is typeset automatically in Computer Modern Roman (CM) fonts.
Please do
\emph{not} change the preset fonts. If you have to use fonts other
than the preset fonts, kindly submit these with your files.

Please use the commands \verb+\label+ and \verb+\ref+ for
cross-references and the commands \verb+\bibitem+ and \verb+\cite+ for
references to the bibliography, to enable us to create hyperlinks at
these places.

For preparing your figures electronically and integrating them into
your source file we recommend using the standard \LaTeX{} \verb+graphics+ or
\verb+graphicx+ package. These provide the \verb+\includegraphics+ command.
In general, please refrain from using the \verb+\special+ command.

Remember to submit any further style files and
fonts you have used together with your source files.

\subsubsection{Headings.}

Headings should be capitalized
(i.e., nouns, verbs, and all other words
except articles, prepositions, and conjunctions should be set with an
initial capital) and should,
with the exception of the title, be aligned to the left.
Words joined by a hyphen are subject to a special rule. If the first
word can stand alone, the second word should be capitalized.

Here are some examples of headings: ``Criteria to Disprove
Context-Freeness of Collage Language", ``On Correcting the Intrusion of
Tracing Non-deterministic Programs by Software", ``A User-Friendly and
Extendable Data Distribution System", ``Multi-flip Networks:
Parallelizing GenSAT", ``Self-determinations of Man".

\subsubsection{Lemmas, Propositions, and Theorems.}

The numbers accorded to lemmas, propositions, and theorems, etc. should
appear in consecutive order, starting with Lemma 1, and not, for
example, with Lemma 11.

\subsection{Figures}

For \LaTeX\ users, we recommend using the \emph{graphics} or \emph{graphicx}
package and the \verb+\includegraphics+ command.

Please check that the lines in line drawings are not
interrupted and are of a constant width. Grids and details within the
figures must be clearly legible and may not be written one on top of
the other. Line drawings should have a resolution of at least 800 dpi
(preferably 1200 dpi). The lettering in figures should have a height of
2~mm (10-point type). Figures should be numbered and should have a
caption which should always be positioned \emph{under} the figures, in
contrast to the caption belonging to a table, which should always appear
\emph{above} the table; this is simply achieved as matter of sequence in
your source.

Please center the figures or your tabular material by using the \verb+\centering+
declaration. Short captions are centered by default between the margins
and typeset in 9-point type (Fig.~\ref{fig:example} shows an example).
The distance between text and figure is preset to be about 8~mm, the
distance between figure and caption about 6~mm.

To ensure that the reproduction of your illustrations is of a reasonable
quality, we advise against the use of shading. The contrast should be as
pronounced as possible.

If screenshots are necessary, please make sure that you are happy with
the print quality before you send the files.
\begin{figure}
\centering
%\includegraphics[height=6.2cm]{eijkel2}
\caption{One kernel at $x_s$ (\emph{dotted kernel}) or two kernels at
$x_i$ and $x_j$ (\textit{left and right}) lead to the same summed estimate
at $x_s$. This shows a figure consisting of different types of
lines. Elements of the figure described in the caption should be set in
italics, in parentheses, as shown in this sample caption.}
\label{fig:example}
\end{figure}

Please define figures (and tables) as floating objects. Please avoid
using optional location parameters like ``\verb+[h]+" for ``here".

\paragraph{Remark 1.}

In the printed volumes, illustrations are generally black and white
(halftones), and only in exceptional cases, and if the author is
prepared to cover the extra cost for color reproduction, are colored
pictures accepted. Colored pictures are welcome in the electronic
version free of charge. If you send colored figures that are to be
printed in black and white, please make sure that they really are
legible in black and white. Some colors as well as the contrast of
converted colors show up very poorly when printed in black and white.

\subsection{Formulas}

Displayed equations or formulas are centered and set on a separate
line (with an extra line or halfline space above and below). Displayed
expressions should be numbered for reference. The numbers should be
consecutive within each section or within the contribution,
with numbers enclosed in parentheses and set on the right margin --
which is the default if you use the \emph{equation} environment, e.g.,
\begin{equation}
  \psi (u) = \int_{o}^{T} \left[\frac{1}{2}
  \left(\Lambda_{o}^{-1} u,u\right) + N^{\ast} (-u)\right] dt \;  .
\end{equation}

Equations should be punctuated in the same way as ordinary
text but with a small space before the end punctuation mark.

\subsection{Footnotes}

The superscript numeral used to refer to a footnote appears in the text
either directly after the word to be discussed or -- in relation to a
phrase or a sentence -- following the punctuation sign (comma,
semicolon, or period). Footnotes should appear at the bottom of
the
normal text area, with a line of about 2~cm set
immediately above them.\footnote{The footnote numeral is set flush left
and the text follows with the usual word spacing.}

\subsection{Program Code}

Program listings or program commands in the text are normally set in
typewriter font, e.g., CMTT10 or Courier.

\medskip

\noindent
{\it Example of a Computer Program}
\begin{verbatim}
program Inflation (Output)
  {Assuming annual inflation rates of 7%, 8%, and 10%,...
   years};
   const
     MaxYears = 10;
   var
     Year: 0..MaxYears;
     Factor1, Factor2, Factor3: Real;
   begin
     Year := 0;
     Factor1 := 1.0; Factor2 := 1.0; Factor3 := 1.0;
     WriteLn('Year  7% 8% 10%'); WriteLn;
     repeat
       Year := Year + 1;
       Factor1 := Factor1 * 1.07;
       Factor2 := Factor2 * 1.08;
       Factor3 := Factor3 * 1.10;
       WriteLn(Year:5,Factor1:7:3,Factor2:7:3,Factor3:7:3)
     until Year = MaxYears
end.
\end{verbatim}
%
\noindent
{\small (Example from Jensen K., Wirth N. (1991) Pascal user manual and
report. Springer, New York)}

\subsection{Citations}

For citations in the text please use
square brackets and consecutive numbers: provided automatically
by \LaTeX 's \verb|\cite| \dots\verb|\bibitem| mechanism.

\subsection{Page Numbering and Running Heads}

There is no need to include page numbers. If your paper title is too
long to serve as a running head, it will be shortened. Your suggestion
as to how to shorten it would be most welcome.

\section{LNCS Online}

The online version of the volume will be available in LNCS Online.
Members of institutes subscribing to the Lecture Notes in Computer
Science series have access to all the pdfs of all the online
publications. Non-subscribers can only read as far as the abstracts. If
they try to go beyond this point, they are automatically asked, whether
they would like to order the pdf, and are given instructions as to how
to do so.

Please note that, if your email address is given in your paper,
it will also be included in the meta data of the online version.

\section{BibTeX Entries}

The correct BibTeX entries for the Lecture Notes in Computer Science
volumes can be found at the following Website shortly after the
publication of the book:
\url{http://www.informatik.uni-trier.de/~ley/db/journals/lncs.html}

\subsubsection*{Acknowledgments.} The heading should be treated as a
subsubsection heading and should not be assigned a number.

\section{The References Section}\label{references}

In order to permit cross referencing within LNCS-Online, and eventually
between different publishers and their online databases, LNCS will,
from now on, be standardizing the format of the references. This new
feature will increase the visibility of publications and facilitate
academic research considerably. Please base your references on the
examples below. References that don't adhere to this style will be
reformatted by Springer. You should therefore check your references
thoroughly when you receive the final pdf of your paper.
The reference section must be complete. You may not omit references.
Instructions as to where to find a fuller version of the references are
not permissible.

We only accept references written using the latin alphabet. If the title
of the book you are referring to is in Russian or Chinese, then please write
(in Russian) or (in Chinese) at the end of the transcript or translation
of the title.


%\begin{thebibliography}{4}

\bibliographystyle{alpha}
\bibliography{tesis}

%\end{thebibliography}


\section*{Appendix: Springer-Author Discount}

LNCS authors are entitled to a 33.3\% discount off all Springer
publications. Before placing an order, the author should send an email, 
giving full details of his or her Springer publication,
to \url{orders-HD-individuals@springer.com} to obtain a so-called token. This token is a
number, which must be entered when placing an order via the Internet, in
order to obtain the discount.

\section{Checklist of Items to be Sent to Volume Editors}
Here is a checklist of everything the volume editor requires from you:

\end{document}
