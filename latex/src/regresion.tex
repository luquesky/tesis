
\begin{figure}
\centering
\includegraphics[width=10cm]{images/linear_regression.jpg}
\caption{Ejemplo de Regresión Lineal}
\end{figure}

\section{Análisis de regresión}

Llegado a este punto, dada una variable a/p, nos interesaría evaluar la relación entre el entrainment y las distintas variables sociales. Con esto en mente, planteamos un modelo de regresión lineal donde nuestra variable explicativa será la mimetización, y la variable \emph{dependiente} será la variable social.

En base a ésto, podremos observar cuál es la variación conjunta de ellas. Es esperable que, al aumentar la mimetización, aumenten ciertas variables sociales (por ejemplo, la compenetración en el juego) y que otras desciendan (el aburrimiento).


\subsection{Modelo clásico de Regresión Lineal}

En el modelo clásico de regresión lineal, tenemos un conjunto de valores fijos $X_1, X_2, \ldots, X_n$, que son llamadas variables independientes. Asociado a cada uno de estos valores fijos, tenemos variables aleatorias $Y_1, \ldots, Y_n$. Asumimos, además, que nuestras variables son de la forma

\begin{equation}
  Y_i = E[Y|X_i] + u_i
\end{equation}

donde $u_i$ es la perturbación estocástica de la variable.

Asumiendo que $E[Y|X_i]$ es una función lineal de $X_i$; es decir, que existen $\beta_1, \beta_2 \in \mathbb{R}$ que cumplen

\begin{equation}
  E[Y|X_i] = \beta_1 + \beta_2 X_i
\end{equation}

obtenemos que

\begin{equation}
  Y_i = \beta_1 + \beta_2 X_i + u_i
\end{equation}

Nuestro objetivo es poder entonces conseguir estimadores $\widehat{\beta_1}, \widehat{\beta_2}$ que nos permitan analizar y predecir el comportamiento conjunto de estas variables.

\subsection{Nuestro modelo}

Sea entonces una variable acústica/prosódica (por ejemplo, el pitch o la intensidad), y una variable social de las que acabamos de enumerar en \ref{sec:panel_data}. Sean $E_1, \ldots, E_n$ los valores de entrainment para el set de datos que definimos en \ref{sec:panel_data}, y sean $V_1, V_2, \ldots V_n$

