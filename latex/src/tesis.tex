\documentclass[11pt,a4paper,twoside]{tesis}
% SI NO PENSAS IMPRIMIRLO EN FORMATO LIBRO PODES USAR
%\documentclass[11pt,a4paper]{tesis}

\usepackage{graphicx}
\usepackage[authoryear]{natbib}
\usepackage{adjustbox}
\usepackage{amssymb}
\usepackage{amsmath}
\usepackage{amsthm}
\usepackage{booktabs}
\usepackage[tables]{xcolor}
\usepackage{colortbl}
\usepackage[utf8]{inputenc}
\usepackage[spanish]{babel}
\usepackage[left=3cm,right=3cm,bottom=3.5cm,top=3.5cm]{geometry}

\begin{document}

%%%% CARATULA
% Comentar y descomentar según corresponda
\def\titulo{Licenciado }

\def\autor{Juan Manuel Pérez}
\def\tituloTesis{Mimetización entre interlocutores}
\def\runtitulo{Medición de la mimetización entre interlocutores utilizando series de tiempo}
\def\runtitle{Measuring entrainment between speakers using time series}
\def\director{Agustín Gravano}
\def\codirector{Ramiro Gálvez}
\def\lugar{Buenos Aires, 2015}
\input{caratula}

%%%% ABSTRACTS, AGRADECIMIENTOS Y DEDICATORIA
\frontmatter
\pagestyle{empty}
\chapter*{\runtitulo}

Los sistemas de diálogo humano-computadora son cada vez más frecuentes, y sus aplicaciones comprenden una amplia gama de rubros: desde aplicaciones móviles, motores de búsqueda, juegos, hasta tecnologías de asistencia para ancianos y discapacitados. Si bien es cierto que estos sistemas logran captar buena parte de la dimensión lingüística de la comunicación humana, tienen un déficit importante a la hora de procesar y transmitir el aspecto superestructural de la comunicación oral, que radica en el intercambio de afecto, emociones, actitudes y otras intenciones de los participantes.

El \emph{entrainment} (mimetización) es un fenómeno inconsciente que se manifiesta a través de la adaptación de posturas, forma de hablar, gestos faciales y otros comportamientos entre dos o más interactores. A su vez, la ocurrencia de esta mimetización está fuertemente emparentada con el sentimiento de empatía y compenetración entre los participantes. En nuestro caso, nos es de interés el \emph{entrainment} sobre las variables acústico-prosódicas, como el tono, intensidad, y otras.

En el presente trabajo, nos proponemos explorar y refinar una métrica del entrainment acústico-prosódico definida en trabajos previos. Analizamos la relación entre los valores obtenidos y las percepciones sociales que terceros tienen sobre las conversaciones, en un corpus de diálogos orientados a tareas en inglés.

\bigskip

\noindent\textbf{Palabras claves:} Procesamiento del Habla, Series de Tiempo, Entrainment, Regresión Lineal

\input{abs_en.tex}

\tableofcontents

\mainmatter
\pagestyle{headings}

%%%% ACA VA EL CONTENIDO DE LA TESIS

\newcommand{\ENGMAX} { ENG\_MAX }
\newcommand{\ENGMEAN} { ENG\_MEAN }
\newcommand{\FOMAX} { F0\_MAX }
\newcommand{\FOMEAN} { F0\_MEAN }
\newcommand{\TOTFRAMES} { VCD2TOT\_FRAMES }
\newcommand{\NOISETOHARMONICS} { NOISE\_TO\_HARMONICS\_RATIO }
\newcommand{\SYLCOUNT} { SYLLABES\_COUNT }
\newcommand{\SYLAVG} { SYLLABES\_COUNT }
\newcommand{\LOCALSHIMMER} { SOUND\_VOICED\_LOCAL\_SHIMMER }
\newcommand{\PHONCOUNT} { PHONEMES\_COUNT }
\newcommand{\PHONAVG} { PHONEMES\_AVERAGE }

\newcommand{\estslope} { \widehat{\beta_2} }

\newcommand{\myhighlight} {\rowcolor[gray]{.75}}

\chapter{Introducción}
\begin{frame}
  \frametitle{Sistemas de diálogo Humano-Computadora}

  \begin{enumerate}[<+->]
    \item Sistemas actuales
    \item Bien en la parte lingüística de la comunicación: entender y transmitir mensajes estructuralmente correctos.
    \item Mal en la parte superestructural: intercambio de emociones, actitudes, etc.
    \item El presente trabajo trata de hacer un (pequeño) aporte sobre el análisis de la ``naturalidad'' de las conversaciones.
  \end{enumerate}
\end{frame}


\begin{frame}
  \begin{columns}
    \column{0.5\textwidth}
    \frametitle{Ejemplos de ``falta de naturalidad''}
    \begin{enumerate}
      \item Sistemas de llamadas comerciales
      \item Siri, Google Now
      \item Otros?
    \end{enumerate}
    \column{0.5\textwidth}
    \begin{figure}
      \includegraphics[scale=0.25]{images/hal.jpg}
    \end{figure}
  \end{columns}
\end{frame}


\begin{frame}
  \frametitle{Prosodia}
  \framesubtitle{Definir acá qué es la prosodia aproximadamente}

\end{frame}



\begin{frame}
  \frametitle{Entrainment}

  \begin{enumerate}
    \item Fenómeno ubícuo en la comunicación
    \item Ocurre a varios niveles
    \item Largamente estudiado en psicología de comportamiento (referencias)
    \item (ya está, la próxima conversación que tengan afuera de acá van a chequear ésto)
  \end{enumerate}
\end{frame}

\begin{frame}
  \frametitle{¿Y cómo lo medimos?}
  La definición de entrainment hasta acá vista es heurística! ¿Cómo definimos una medida para esto?

  Vamos a explorar una métrica definida en trabajos anteriores, pulirla un poco, y verificar que efectivamente capture ciertas características del entrainment.

  ¿Cómo? Usando un corpus con anotaciones sociales


\end{frame}



\chapter{Método}

\section{Columbia Games Corpus}

\newcommand{\objectgame} {\emph{Juego de objetos}}


Empleamos el Columbia Games Corpus  \cite{GRAV2009} que consiste de doce conversaciones diádicas (i.e., con dos participantes) entre trece personas angloparlantes distintas. Todos los participantes reportaron hablar inglés americano estándar, y no tener problemas de audición. La edad de los participantes se encuentra en el rango de los 20 a 50 años.

Las grabaciones se hicieron en 44 kHz, 16 bits con un canal separado para cada hablante; luego fueron guardadas en 16 kHz para el presente estudio. Cada sesión duró aproximadamente 45 minutos, totalizando 9 horas de
diálogos, 70.259 palabras (2.037 únicas) para todo el cuerpo de datos. Todas las conversaciones cuentan con transcripciones textuales alineadas temporalmente a la señal de audio, realizadas por personal especialmente entrenado.

En cada sesión, se sentó a dos participantes (quienes no se conocían previamente) en una cabina profesional de grabación, cara a cara a ambos lados de una mesa, y con una cortina opaca colgando entre ellos para evitar la comunicación visual. Los participantes contaron con sendas computadoras portátiles conectadas entre sí, en las cuales jugaron una serie de juegos simples que requerían de comunicación verbal. El primero de ellos es un juego de cartas que no consideramos en el presente estudio por tratarse esencialmente de monólogos o diálogos con poca interacción. Luego de esto, pasaron al juego que analizamos, denominado `juego de objetos'.

\subsection{Juego de Objetos}

En el juego de objetos, la pantalla de cada jugador mostró un tablero con varios objetos, entre 5 y 7, como se ve en la Figura \ref{objects_game}.
Para uno de los jugadores (el Descriptor) el objeto \emph{Objetivo} aparecía en una posición aleatoria entre otros objetos. Para el otro jugador, a quien llamaremos el Seguidor, el objetivo aparecía en la parte baja de la pantalla. Entonces, al Descriptor se le encargaba describir la posición del Objetivo de manera que el Seguidor pudiera mover su representación del objeto a la misma posición en su pantalla. Luego de una negociación entre ambos jugadores para decidir la mejor posición del objeto, se les asignó a los jugadores una puntuación entre 1 y 100 puntos de acuerdo a qué tan acertado fue el posicionamiento del objetivo por parte del Seguidor.

\begin{figure}
\centering
\includegraphics[scale=0.5]{images/columbia_games.jpg}
\caption{Juego de objetos del Columbia Games}
\label{objects_game}
\end{figure}


Cada sesión consistió de 14 tareas como ésta, cambiando los objetos y sus ubicaciones. En las primeras cuatro tareas, uno de los sujetos tomó el papel del Descriptor; en los siguientes cuatro invirtieron roles, y en las finales seis fueron alternando los roles de Descriptor y Seguidor.

\subsection{Anotaciones sobre comportamiento social}

Varios aspectos del comportamiento de los jugadores durante los juegos de objetos fueron anotados mediante la herramienta de crowdsourcing \emph{Amazon Mechanical Turk}\footnote{https://www.mturk.com}. Cada anotador escuchó el audio correspondiente a una tarea del juego y tuvo que responder a varias preguntas sobre cada uno de los sujetos, entre las que se encuentran:

\begin{itemize}
  \item ¿el sujeto contribuye para el éxito del equipo?  (\svcontributes)
  \item ¿el sujeto parece comprometido con el juego? (\svengaged)
  \item ¿el sujeto se expresa correctamente? (\svclear)
  \item ¿el sujeto piensa lo que va a decir? (\svplanning)
  \item ¿el sujeto alienta a su compañero? (\svencourages)
  \item ¿el sujeto le hace difícil hablar a su compañero? (\svdifficult)
  \item ¿el sujeto está aburrido con el juego? (\svbored)
  \item ¿al sujeto no le agrada su compañero? (\svdislikes)
\end{itemize}

\noindent Cada uno de estos audios fue puntuado por cinco anotadores, que respondieron por sí o por no para cada una de las preguntas. El puntaje que recibe cada una de las preguntas (a las cuales llamaremos a partir de ahora \emph{variables sociales}) consiste en la cantidad de respuestas afirmativas que recibió, teniendo un rango de 0 a 5. Por ejemplo, una tarea dada podría tener puntaje 3 para la variable social `el sujeto A se expresa correctamente' o puntaje 5 para la variable `el sujeto B dirige la conversación'.

\subsection{Extracción de variables acústico/prosódicas}

La herramienta \emph{Praat} \footnote{http://www.fon.hum.uva.nl/praat/} fue utilizada para extraer automáticamente las variables \ap del corpus. Las variables que medimos fueron el tono, la intensidad, la proporción de vocalizaciones, jitter, shimmer, cantidad de sílabas, cantidad de fonemas, y la proporción de ruido sobre armónicos. Estos atributos fueron medidos en cada uno de los segmentos de habla del corpus.

Repasemos algunos conceptos que necesitamos para definir las variables acústicas.

\begin{itemize}
  \item \emph{f0} refiere a la frecuencia fundamental de una onda, que es el recíproco del período de ésta. El \emph{tono} o \emph{pitch} es la percepción que tenemos de la frecuencia fundamental, que nos marca cuán agudos o graves son los sonidos.
  \item \emph{Intensity} refiere al volumen o intensidad de la onda. Ésta mide la amplitud de la onda, y es la percepción de cuán fuerte es el sonido.
  \item \emph{jitter y shimmer} se refieren, en un intervalo de tiempo, a los desplazamientos de la onda de la verdadera periodicidad y de la amplitud, respectivamente. Están asociadas con la percepción de la calidad de la voz.
  \item Un \emph{fonema} es la articulación simple de sonidos del habla, tanto de vocales como de consonantes. Ejemplos de fonemas son los sonidos de las letras u, a, s, k en español.
  \item El \emph{noise-to-harmonics ratio} (abreviado NHR) puede considerarse como una medida de calidad de la voz, que cuantifica la proporción de ruido que hay en ésta.
\end{itemize}


En la siguiente tabla resumimos estas features. Recordemos nuevamente que estas features son medidas en un intervalo de tiempo.

\begin{figure}[h!]
\centering

\begin{tabular} {|c|c|}
  \hline
  Variable & Descripción \\
  \hline
  \hline
  \FOMEAN & Valor medio de la frecuencia fundamental \\\hline
  \FOMAX  & Valor máximo de la frecuencia fundamental \\\hline
  \ENGMEAN & Valor medio de la intensidad \\\hline
  \ENGMAX & Valor máximo de la intensidad \\\hline
  \NOISETOHARMONICS & Noise-to-harmonics ratio \\\hline
  \LOCALSHIMMER & Shimmer medido \\\hline
  \LOCALJITTER  & Jitter medido \\\hline
  \SYLAVG & Cantidad de sílabas por segundo \\\hline
  \PHONAVG & Cantidad de fonemas por segundo \\\hline
\end{tabular}
\end{figure}

\section{Series de Tiempo}

En términos informales, una serie de tiempo es un conjunto de datos recolectados a través del tiempo.

\section{Descripción TAMA}

En \cite{KOU2008} se introdujo un método novedoso para el análisis del \entrainment acústico/prosódico. Esta técnica consiste, a grandes rasgos, en armar dos series de tiempo para cada uno de los interlocutores y luego utilizar herramientes de análisis sobre las series construídas. Una serie de tiempo, en términos coloquiales, es una colección cronológica de observaciones, como pueden ser los valores de las acciones de una empresa a lo largo del tiempo, o la cantidad de lluvia medida en \emph{ml} para cada mes de cierto año. En el apéndice \ref{sec:time_series} describimos más en detalle los conceptos básicos sobre series de tiempo.

Un problema que resuelve esta técnica es el del alineamiento: si intentásemos comparar cada segmento del habla (utterance) con otros, ¿cómo los alineamos? Una posibilidad sería uno a uno, aunque ésto es muy simplista y poco representativo de la realidad. Al introducir el concepto de series de tiempo, podemos olvidarnos de los segmentos del habla y simplemente utilizar estas construcciones.

Para construir la serie de tiempo de cada interlocutor debemos, en primer lugar, dividir el diálogo en ventanas solapadas de igual tamaño. A la diferencia entre ventana y ventana llamaremos \emph{frame step}, y al tamaño de ventana \emph{frame length}. Consideraremos sólo los segmentos de habla que se encuentren dentro de cada ventana; aquellos segumentos que atraviesen los límites de las ventanas son cortados para que se mantengan dentro de éste. En la figura \ref{tama} se ilustra el proceso.

Como producto de ésto, nuestro corpus queda dividido en una sucesión ventanas solapadas. En el trabajo original, se usa un step de 10 segundos, y un tamaño de ventana de 20 segundos. Ésto da como resultado un solapamiento del 50\%. En la sección \ref{sec:window_selection}, describimos la elección del tamaño de ventana que hicimos en base al corpus que utilizamos.

\begin{figure}
\centering
\includegraphics[width=10cm]{images/tama.png}
\caption{Gráfico de la separación del diálogo en ventanas}
\label{tama}
\end{figure}

Una vez que la conversación se ha partido en ventanas mediante el proceso descripto, se calculan los valores de la serie de tiempo para cada interlocutores de cada una de ellas. Ésto se hace mediante el siguiente cálculo:

\begin{equation}
    \mu = \sum\limits_{i=1}^N f_i d_i^\prime \label{eq:tama_mean}\\
\end{equation}

donde $i$ itera sobre las elocuciones dentro del \emph{frame}, $d_i^\prime$ es la duración relativa del segmento (respecto del tiempo total hablado) y $f_i$ es el valor de la \emph{feature} que estamos midiendo. $d_i^\prime$ se calcula con la fórmula

\begin{equation}
dr_i = \frac{d_i}{\sum\limits_{i=1}^N d_i}
\end{equation}

donde $d_i$ es la longitud en segundos de los segmentos del habla en el frame.

Como se ve en \ref{eq:tama_mean}, el valor que calculamos es una media ponderada del valor de la feature por la duración de las locuciones. Así, por ejemplo, al calcular una serie de tiempo sobre la intensidad, la contribución de interjecciones (\emph{ah!} por ejemplo), que suelen tener altos valores \emph{volumen}, estará disminuída por sus breves duraciones.

Una vez obtenidas, dado un feature acústico/prosódico y una conversación, dos series de tiempo mediante el cálculo ventana a ventana de \ref{eq:tama_mean}, necesitamos efectuar algún tipo de análisis sobre éstas para obtener una medida del \entrainment.

\nota{Mejorar el dibujo éste y agregarle una descripción}

%\subsection{Corpus}
\begin{frame}
  \frametitle{Columbia Games Corpus}
  \framesubtitle{Descripción}
  \begin{columns}
    \column{0.35\textwidth}
      \begin{figure}
        \includegraphics[width=\textwidth]{images/columbia_games_color.jpg}
      \end{figure}

    \column{0.65\textwidth}

    \begin{itemize}
      \item Desarrollado por Agustín Gravano para su tesis doctoral.
      \item Corpus de conversaciones diádicas en Inglés Norteamericano
      \item 12 sesiones con 14 tareas/juegos cada una.
      \item En cada sesión, se sentó a dos participantes en una cabina profesional de grabación, y una cortina opaca colgando entre ellos para evitar la comunicación visual.
      \item Los participantes contaron con computadoras a través de las cuales interactuaban mediante juegos.
    \end{itemize}
  \end{columns}
\end{frame}


\begin{frame}
  \frametitle{Columbia Games Corpus}
  \framesubtitle{Anotaciones sociales}

  Cinco anotadores escucharon el audio correspondiente a una tarea del juego y respondieron a varias preguntas sobre los sujetos:


\begin{table}
\adjustbox{max width=0.8\textwidth}{
\begin{tabular}{|l|l|}
  \hline

  \textbf{Nombre} & \textbf{Pregunta} \\
  \hline

  \svcontributes &  ¿el sujeto contribuye para el éxito del equipo?  \\ \hline
  \svengaged &  ¿el sujeto parece comprometido con el juego? \\ \hline
  \svclear &  ¿el sujeto se expresa correctamente? \\ \hline
  \svplanning &  ¿el sujeto piensa lo que va a decir? \\ \hline
  \svencourages &  ¿el sujeto alienta a su compañero? \\ \hline
  \svdifficult &  ¿el sujeto le hace difícil hablar a su compañero? \\ \hline
  \svbored &  ¿el sujeto está aburrido con el juego? \\ \hline
  \svdislikes &  ¿al sujeto no le agrada su compañero? \\ \hline
\end{tabular}
}
\end{table}

De cada una de estas preguntas obtenemos un puntaje de 0 a 5, para cada hablante de cada tarea.
\end{frame}


\begin{frame}
\frametitle{Extracción de features acústico-prosódicas}

Usando el software Praat \footnote{http://www.fon.hum.uva.nl/praat/} se extrajeron las variables acústico-prosódicas para cada segmento de habla

\begin{table}
\adjustbox{max width=0.8\textwidth}{
\centering
\begin{tabular} {|c|c|}
  \hline
  Variable & Descripción \\
  \hline
  \FOMEAN & Valor medio de la frecuencia fundamental \\\hline
  \FOMAX  & Valor máximo de la frecuencia fundamental \\\hline
  \ENGMEAN & Valor medio de la intensidad \\\hline
  \ENGMAX & Valor máximo de la intensidad \\\hline
  \NOISETOHARMONICS & Noise-to-harmonics ratio \\\hline
  \LOCALSHIMMER & Shimmer medido \\\hline
  \LOCALJITTER  & Jitter medido \\\hline
  \SYLAVG & Cantidad de sílabas por segundo \\\hline
  \PHONAVG & Cantidad de fonemas por segundo \\\hline
\end{tabular}
}
\end{table}
\end{frame}



\subsection{Modificaciones a TAMA}

\begin{frame}
\frametitle{TAMA}
\framesubtitle{Nuestras modificaciones}

\begin{itemize}
  \item Usamos un step de 8s y un tamaño de ventana de 16s, manteniendo el solapamiento del 50\%.
  \item A diferencia del trabajo original de Kousidis, utilizamos series con datos faltantes.
  \item Pero sólo nos quedamos con aquellas que tengan 5 o más puntos definidos.
\end{itemize}
\end{frame}


\begin{frame}
\frametitle{TAMA}
\framesubtitle{Nuestra medida de mimetización}

  \begin{columns}
    \column{0.33\textwidth}
    \begin{figure}[t]
      \includegraphics[scale=0.28]{images/time_plot.png}
    \end{figure}

    \begin{figure}[t]
      \includegraphics[scale=0.28]{images/cross_correlogram_2.png}
    \end{figure}

    \column{0.66\textwidth}

    Definimos dos medidas de mimetización

    \begin{align*}
      \fwentrainment{AB}^{(1)} &= r_s \text{ con s maximizando } |r_k|,  k <= 0  \\
      \fwentrainment{BA}^{(1)} &= r_s \text{ con s maximizando } |r_k|,  k >= 0  \\
      \fwentrainment{XY}^{(2)} &= |\fwentrainment{XY}^{(1)}|
    \end{align*}

    Segunda métrica motivada por estudios sobre la antimimetización. Healey et al (2014) sugiere que puede ser una conducta de adaptación cooperativa.

    Levitan et al (2015) da más indicios en esa dirección.
  \end{columns}
\end{frame}

\subsection{Análisis de la relación con variables sociales}

\begin{frame}
\frametitle{Mimetización y relación con variables sociales}

  Para analizar la relación entre las variables sociales ($V$) y nuestras medidas de \emph{mimetización} ($\mathcal{E}$), planteamos un modelo de regresión lineal.

    \begin{equation}
      V_i \sim \beta_1 + \beta_2 \mathcal{E}_i
    \end{equation}

  Nuestra hipótesis es

  \begin{enumerate}
    \item Si $V$ es una variable de carácter positivo, entonces $\beta_2 > 0$
    \item Si $V$ es una variable de carácter negativo, entonces $\beta_2 < 0$
  \end{enumerate}
\end{frame}

\section{Time plots}
\label{sec:time_plots}
Usando la técnica descripta con las variaciones que consideramos en la anterior sección, generamos dos series de tiempo para cada tarea. Como antes mencionamos, la ventana elegida es de 16 segundos con un step de 8 segundos lo cual da un overlap del 50\%.

Dada una ventana, puede ocurrir que alguno de los interlocutores no haya hablado, o su interacción haya sido demasiado breve como para medir sus variables \ap. Como ya mencionamos en la sección \ref{sec:tama_modifications}, y a diferencia de \cite{KOU2008.2}, construimos las series sin ese punto, y sin interpolarlo tampoco.

De estas tareas, sólo nos quedamos con aquellas que tengan al menos 5 puntos definidos para cada serie, de manera que tenga sentido poder calcular la correlación cruzada más adelante. Con esto, no sólo nos interesa la duración de la charla, sino cierta calidad de las series generadas. En la tabla \ref{tab:time_series} pueden verse las tareas que tuvimos en consideración, junto a su duración.

\begin{table}
\centering

\adjustbox{max width=\textwidth}{\begin{tabular}{|l|llllllllllll|}
\toprule
Task & S-01 & S-02 & S-03 & S-04 & S-05 & S-06 & S-07 & S-08 & S-09 & S-10 & S-11 & S-12 \\
\hline
01 &         -- &         -- &    149.8   &         -- &         -- &         -- &         -- &         -- &     54.5   &    106.0   &         -- &     56.1   \\
02 &         -- &         -- &         -- &         -- &         -- &         -- &         -- &         -- &     41.7   &     63.8   &         -- &         -- \\
03 &         -- &     51.7   &         -- &     80.7   &     77.9   &     69.2   &     68.4   &     49.6   &         -- &    122.2   &     81.0   &         -- \\
04 &         -- &    187.2   &     93.3   &     76.1   &     79.9   &     99.2   &     84.3   &         -- &     58.0   &    129.6   &     67.9   &     95.2   \\
05 &         -- &         -- &         -- &     86.3   &         -- &    126.7   &    145.8   &     90.7   &     45.7   &    134.2   &         -- &         -- \\
06 &         -- &         -- &         -- &         -- &         -- &    148.2   &     50.6   &     60.2   &     46.1   &     66.7   &     46.7   &     40.2   \\
07 &         -- &     66.0   &         -- &    117.7   &         -- &     72.4   &         -- &     87.7   &     85.9   &    110.6   &     65.7   &         -- \\
08 &         -- &    458.8   &     98.6   &    203.8   &         -- &    188.7   &     59.9   &     48.1   &         -- &    157.4   &         -- &     81.1   \\
09 &         -- &         -- &         -- &     75.5   &    134.2   &     83.0   &    108.7   &         -- &     62.1   &    404.0   &     41.0   &     92.5   \\
10 &     50.1   &    231.3   &    162.8   &    242.5   &         -- &    122.4   &     71.1   &     74.7   &         -- &    356.0   &     69.8   &     92.7   \\
11 &         -- &     74.4   &         -- &     98.6   &     70.1   &         -- &     58.9   &         -- &     72.9   &    104.0   &     59.4   &    101.9   \\
12 &     61.3   &     90.1   &    129.1   &    182.9   &         -- &    130.3   &     75.8   &     57.6   &         -- &    101.6   &         -- &     64.8   \\
13 &     55.1   &    124.0   &    108.1   &    144.1   &    114.7   &         -- &         -- &     83.8   &     94.0   &    174.0   &     84.8   &     91.5   \\
14 &         -- &     75.3   &         -- &         -- &    107.3   &         -- &     52.5   &    144.3   &     75.5   &    108.4   &     91.6   &     98.4   \\
\bottomrule
\end{tabular}
}

\caption{Tabla de tareas seleccionadas y sus duraciones}
\label{tab:time_series}
\end{table}


Como primer paso siempre recomendado en el análisis de series de tiempo \cite{CHATFIELD}, graficamos los time plots conjunto de cada par de series, a la vez que sus autocorrelogramas (ver apéndice \ref{sec:time_series}). En la figura \ref{fig:time_plot} podemos observar un ejemplo de esto.

A priori, las series tienen aspecto de series autoregresivas de orden uno. Es decir, series que son de la forma $X_t = \alpha X_{t-1} + e_t + c$, con $e_t$ ruido blanco, $\alpha$ y $c$ constantes. Este hecho es esperable  por la construcción misma del método TAMA, ya que la ventana de cada punto tiene un solapamiento con la ventana anterior. Más aún, uno esperaría que $\alpha \sim 0.5$ ya que nuestras ventanas tienen ese índice de overlap. Los autocorrelogramas de las series, por otro lado, tienen en su mayoría un valor significativo en $k = 1$, el valor del $\alpha$ de la autoregresión.


\begin{figure}
\centering
\includegraphics[width=15cm]{images/time_plot_with_autocorrelation.png}
\caption{Time-plot generado por el método TAMA, junto a su autocorrelograma}
\label{fig:time_plot}
\end{figure}

El hecho de que los autocorrelogramas desciendan rápidamente a cero es un indicio de que las series de tiempo construidas son estacionarias. Esto nos habilita a efectuar el análisis bivariado de las series.

\section{Análisis bivariado}
\label{sec:analisis_bivariado}

\newcommand{\squarederr}[1]{
    \sum\limits_{t=1}^n \varnorm{#1}^2
}

\newcommand{\crosscorr}[2]{
  \frac{\sum\limits_{t=|k|+1}^n \varnorm{#1} (#2_{t-k} - \mu_{#2})}{
    \sqrt{\squarederr{#1} \squarederr{#2}}
  } \\
}

\newcommand{\corrdenom}{\sqrt{\squarederr{A}\squarederr{B}}}

En \cite{KOU2008.2} se continúa el trabajo en series de tiempo, y se efectúan análisis tanto para cada serie por separado como para las dos en conjunto, lo cual se llama ``análisis bivariado'' en la terminología de series de tiempo. En este análisis pretendemos analizar ambas series como parte de un sistema y ver cómo se influyen y retroalimentan mutuamente.

Una posible medida del \entrainment se podría obtener midiendo cuánto influye una serie sobre otra, considerándolas a ambas como parte de un sistema donde ambas interactúan. Este \entrainment, entonces, sería direccional: queremos medir cuánto influye el interlocutor $A$ sobre el interlocutor $B$ y viceversa. Puede darse el caso en que ambos tengan fuerte interacción, en tal caso hablamos de \emph{feedback}.

Para medir cuánto se mimetizan las dos series, utilizaremos la función de correlación cruzada (f.c.c) \cite{CHATFIELD}, que mide cuánto se parecen la serie $X$ e $Y$ aplicando un desplazamiento $k$, lo cual nos arroja como resultado un valor entre $-1$ y $1$ (similar al coeficiente de correlación de la estadística clásica). Podemos aproximar la c.c.f. mediante la fórmula de la correlación cruzada muestral.

\begin{equation}
  \label{cross_correlation_definition}
  r_{AB}(k) =
  \left\{
    \begin{array}{ll}
      \frac{\sum\limits_{t=k+1}^n \varnorm{A} (B_{t-k} - \mu_{B})}{\corrdenom} \\ & \mbox{si } k \geq 0 \\
      \frac{\sum\limits_{t=-k+1}^n \varnorm{B} (A_{t+k} - \mu_{A})}{\corrdenom} \\  & \mbox{si } k < 0
    \end{array}
  \right.
\end{equation}

Podemos ver que, si $k \geq 0$, lo que hacemos es, a grandes rasgos, calcular la correlación de Pearson entre $A$ y $B$, pero tomando los $n-k$ últimos valores de $A$ y los $n-k$ primeros de $B$. Si $k < 0$, lo hacemos entre $A$ y $B$, pero desplazando en sentidos inversos. Viéndolo de otra forma, si $k \geq 0$, estamos midiendo cuánto influye $B$ sobre $A$ contemplando un desplazamiento de $k$ puntos; si $k \leq 0$ medimos la influencia de $A$ sobre $B$ a misma distancia. La utilización de estos desplazamientos está explicada en \cite{gravano2015backward}, donde se menciona que la influencia de los hablantes no es necesariamente inmediata sino que puede tener algunos segundos de demora para tomar lugar.


Para cada conversación, se estima entonces el correlograma cruzado, considerando desplazamientos tanto positivos como negativos. Hecho esto, en el estudio \cite{KOU2008.2} sólo analizan la significancia de los resultados de la correlación cruzada, enumerando aquellos lags en los cuales esto ocurrió. En la sección \ref{sec:method_entrainment} comentaremos cómo utilizamos la técnica descripta para la medición del entrainment direccional.

\begin{figure}
\centering
\includegraphics[width=\textwidth]{images/time_plot_with_cross_correlation.png}
\caption{Time-plot producido por TAMA, junto a su autocorrelación y correlación cruzada}
\end{figure}

\section{Armado de Tabla}

Para condensar todos nuestros datos, armamos una tabla por cada variable a/p. Esta table contiene información definida para cada interlocutor, tarea y sesión de nuestro corpus.

\begin{itemize}
  \item session: número de sesión
  \item task: número de tarea
  \item speaker: 0 si corresponde al interlocutor A; B en otro caso
  \item count: La cantidad de puntos definidos que tiene la serie
  \item entrainment: Si $speaker=0$, es $A\rightarrow B$; $B \rightarrow A$ en otro caso
  \item best\_lag: el lag del cross-correlogram donde se logra el \emph{entrainment}
  \item tama\_mean: el promedio de la variable
\end{itemize}

Además, agregamos las variables sociales (relativas al interlocutor) para cada fila:

\begin{itemize}
  \item conversation\_awkward
  \item flow\_naturally
  \item hard\_time\_understanding\_each\_other
  \item contributes\_to\_successful\_completion
  \item making\_self\_clear
  \item engaged\_in\_game
  \item planning\_what\_to\_say
  \item gives\_encouragement
  \item difficult\_for\_partner\_to\_speak
  \item bored\_with\_game
  \item dislikes\_partner
\end{itemize}

En el corpus original, cada variable estaba replicada por cada interlocutor, y por sí o por no, de manera que teníamos:

\begin{enumerate}
  \item $conversation\_awkward\_A_yes$
  \item $conversation\_awkward$
  \item $conversation\_awkward$
  \item $conversation\_awkward$
\end{itemize}



Ésto nos da una tabla de 210 filas, y 21 columnas.


\begin{figure}
\centering
\includegraphics[width=10cm]{images/linear_regression.jpg}
\caption{Ejemplo de Regresión Lineal}
\end{figure}

\section{Análisis de regresión}

Llegado a este punto, dada una variable a/p, nos interesaría evaluar la relación entre el entrainment y las distintas variables sociales. Con esto en mente, planteamos un modelo de regresión lineal donde nuestra variable explicativa será la mimetización, y la variable \emph{dependiente} será la variable social.

En base a ésto, podremos observar cuál es la variación conjunta de ellas. Es esperable que, al aumentar la mimetización, aumenten ciertas variables sociales (por ejemplo, la compenetración en el juego) y que otras desciendan (el aburrimiento).


\subsection{Modelo clásico de Regresión Lineal}

En el modelo clásico de regresión lineal, tenemos un conjunto de valores fijos $X_1, X_2, \ldots, X_n$, que son llamadas variables independientes. Asociado a cada uno de estos valores fijos, tenemos variables aleatorias $Y_1, \ldots, Y_n$. Asumimos, además, que nuestras variables son de la forma

\begin{equation}
  Y_i = E[Y|X_i] + u_i
\end{equation}

donde $u_i$ es la perturbación estocástica de la variable.

Asumiendo que $E[Y|X_i]$ es una función lineal de $X_i$; es decir, que existen $\beta_1, \beta_2 \in \mathbb{R}$ que cumplen

\begin{equation}
  E[Y|X_i] = \beta_1 + \beta_2 X_i
\end{equation}

obtenemos que

\begin{equation}
  Y_i = \beta_1 + \beta_2 X_i + u_i
\end{equation}

Nuestro objetivo es poder entonces conseguir estimadores $\widehat{\beta_1}, \widehat{\beta_2}$ que nos permitan analizar y predecir el comportamiento conjunto de estas variables.

\subsection{Nuestro modelo}

Sea entonces una variable acústica/prosódica (por ejemplo, el pitch o la intensidad), y una variable social de las que acabamos de enumerar en \ref{sec:panel_data}. Sean $E_1, \ldots, E_n$ los valores de entrainment para el set de datos que definimos en \ref{sec:panel_data}, y sean $V_1, V_2, \ldots V_n$ los valores de la variable social de cada conversación.

Sobre éstas variables es que planteamos nuestro modelo de regresión lineal clásica: queremos ver qué relación hay tomando como variable ``fija'' al entrainment, y como variable dependiente a la variable social. Queremos hallar, entonces $\widehat{\beta_1}, \widehat{\beta_2} \in \mathbb{R}$

\begin{equation}
  V_i \simeq \widehat{\beta_1} + \widehat{\beta_2} E_i
\end{equation}


Para ello, calcularemos los estimadores $\widehat{\beta_1}, \widehat{\beta_2} \in \mathbb{R}$ mediante el método \emph{QR} (insertar referencia aquí) que nos provee el lenguaje R. A su vez, luego de ésto efectuaremos un análisis de significancia sobre $\beta_2$ para verificar que sean distintos de 0.


Uno esperaría que un alto \emph{entrainment} se relacione con un alto valor de ciertas variables sociales \cite{BRE1996}, por ejemplo la compenetración con el juego, el ayudar a terminarlo. Esto significa esperar que el valor de $\widehat{\beta_2}$; y se relacione con bajos valores de otras, como el aburrimiento, o el rechazo percibido hacia el compañero.


\subsection{Modelo agrupado o \emph{pooled}}

En el modelo agrupado o \emph{pooled}, no distinguimos entre datos provenientes de distintos ``grupos'' \cite{gujarati1999} y sobre éstos calculamos la regresión lineal, agrupando todos los datos disponibles.

Un problema que surge con este tipo de regresión es que niega todo tipo de \emph{heterogeneidad} de los datos: estos pueden provenir de interlocutores más o menos empáticos, o cuya interacción en el juego se vio influída por factores no medidos en el experimento. Todo ésto es descartado, aún cuando puede afectar seriamente  el resultado obtenido.

AGREGAR GRAFICO DE EJEMPLO PARA ESTO

\subsection{Modelo de Efectos Fijos dentro de cada grupo}


El modelo de efectos fijos agrega el concepto de heterogeneidad permitiendo que cada sujeto tenga su propio valor de ordenada al origen. En términos formales, reformulemos nuestro modelo de la sección anterior:

\begin{equation}
  V_{it} = \beta_1 + \beta_2 * E_{it} + u_{it}
\end{equation}

donde $i$ es la cantidad de sujetos (en nuestro caso, 12 sesiones por 2 interlocutores = 24), $t$ es la variable de tiempo (en nuestro caso, las tareas de cada sesión). El modelo de efectos fijos nos permite


DEFINIR SUJETO



\chapter{Resultados}
\section{Modelo clásico}

\begin{figure}[t!]
\includegraphics[width=15cm]{images/regression_F0_MEAN_1.pdf}
\caption{Gráfico de los pares entrainment-variable a/p, junto a la regresión lineal obtenida \label{regresion_clasica} para \emph{F0\_MEAN}}
\end{figure}

En el modelo clásico dio resultados con baja significancia. En \ref{regresion_clasica} puede verse el gráfico de \emph{F0\_MEAN} y 4 variables sociales y en \ref{regresion_clasica_tabla} pueden verse los valores de las estimaciones de $\estslope$ junto a sus p-valores.

\begin{figure}[l]
% "ENG_MEAN"
% latex table generated in R 3.2.2 by xtable 1.8-0 package
% Thu Jan  7 03:01:56 2016
\begin{tabular}{rrrrr}
  \hline
 & Estimate & Std. Error & t value & Pr($>$$|$t$|$) \\
  \hline
bored\_with\_game & 10.65 & -0.44 & 0.00 & 0.66 \\
  difficult\_for\_partner\_to\_speak & 10.56 & -0.54 & 0.00 & 0.59 \\
  contributes\_to\_successful\_completion & 59.62 & -1.15 & 0.00 & 0.25 \\
  engaged\_in\_game & 73.14 & 0.55 & 0.00 & 0.59 \\
  gives\_encouragement & 47.49 & -0.04 & 0.00 & 0.97 \\
  making\_self\_clear & 52.97 & -0.99 & 0.00 & 0.33 \\
  planning\_what\_to\_say & 32.02 & -1.81 & 0.00 & 0.07 \\
  dislikes\_partner & 9.61 & -0.94 & 0.00 & 0.35 \\
   \hline
\end{tabular}

% "F0_MEAN"
% latex table generated in R 3.2.2 by xtable 1.8-0 package
% Thu Jan  7 03:01:56 2016
\begin{tabular}{rrrrr}
  \hline
 & Estimate & Std. Error & t value & Pr($>$$|$t$|$) \\
  \hline
bored\_with\_game & 10.63 & -0.18 & 0.00 & 0.86 \\
  difficult\_for\_partner\_to\_speak & 10.68 & -0.97 & 0.00 & 0.33 \\
  contributes\_to\_successful\_completion & 59.38 & 0.89 & 0.00 & 0.37 \\
  engaged\_in\_game & 73.41 & 0.77 & 0.00 & 0.44 \\
  gives\_encouragement & 47.59 & 1.09 & 0.00 & 0.28 \\
  making\_self\_clear & 52.91 & -0.32 & 0.00 & 0.75 \\
  planning\_what\_to\_say & 31.49 & 0.39 & 0.00 & 0.70 \\
  dislikes\_partner & 9.81 & -1.74 & 0.00 & 0.08 \\
   \hline
\end{tabular}

% "F0_MAX"
% latex table generated in R 3.2.2 by xtable 1.8-0 package
% Thu Jan  7 03:01:56 2016
\begin{tabular}{rrrrr}
  \hline
 & Estimate & Std. Error & t value & Pr($>$$|$t$|$) \\
  \hline
bored\_with\_game & 11.08 & 2.46 & 0.00 & 0.01 \\
  difficult\_for\_partner\_to\_speak & 10.65 & 0.52 & 0.00 & 0.60 \\
  contributes\_to\_successful\_completion & 60.08 & -0.98 & 0.00 & 0.33 \\
  engaged\_in\_game & 74.20 & -0.57 & 0.00 & 0.57 \\
  gives\_encouragement & 48.17 & -1.04 & 0.00 & 0.30 \\
  making\_self\_clear & 53.86 & -2.32 & 0.00 & 0.02 \\
  planning\_what\_to\_say & 31.86 & -1.05 & 0.00 & 0.30 \\
  dislikes\_partner & 9.65 & 0.97 & 0.00 & 0.33 \\
   \hline
\end{tabular}

% "NOISE_TO_HARMONICS_RATIO"
% latex table generated in R 3.2.2 by xtable 1.8-0 package
% Thu Jan  7 03:01:56 2016
\begin{tabular}{rrrrr}
  \hline
 & Estimate & Std. Error & t value & Pr($>$$|$t$|$) \\
  \hline
bored\_with\_game & 10.75 & -0.44 & 0.00 & 0.66 \\
  difficult\_for\_partner\_to\_speak & 10.64 & 0.16 & 0.00 & 0.87 \\
  contributes\_to\_successful\_completion & 60.34 & -0.75 & 0.00 & 0.45 \\
  engaged\_in\_game & 74.49 & -0.16 & 0.00 & 0.87 \\
  gives\_encouragement & 48.39 & -0.79 & 0.00 & 0.43 \\
  making\_self\_clear & 53.60 & -0.11 & 0.00 & 0.91 \\
  planning\_what\_to\_say & 31.99 & -0.02 & 0.00 & 0.99 \\
  dislikes\_partner & 9.62 & -1.39 & 0.00 & 0.17 \\
   \hline
\end{tabular}




\caption{Tablas con los resultados de la regresión clásica para ENG\_MEAN, ENG\_MAX, F0\_MEAN y F0\_MAX. En la segunda columna se cita el valor de $\estslope$, la desviación estándar calculada, el t-valor obtenido y la significancia}\label{regresion_clasica_tabla}
\end{figure}



\section{Modelo de Efectos Fijos}

\newcommand{\slopeestim}[1] { $\estslope \sim #1$ }

El modelo de efectos fijos sobre el valor absoluto del \emph{entrainment} dio valores sustancialmente más apreciables. \ENGMAX, \FOMEAN y \NOISETOHARMONICS poseen valores altamente significativos ( p-valor menor a 0.05) para la regresión con efectos fijos para al menos 2 variables sociales.

En la tabla \ref{regresion_efectos_fijos_tabla} podemos ver estos valores con las variables sociales significativas resaltadas.


\begin{figure}

\begin{tabular}{rrrrr}
  \hline
 \ENGMAX & Estimate & Std. Error & t value & Pr($>$$|$t$|$) \\
  \hline
contributes\_to\_successful\_completion & 0.0497 & 0.4262 & 0.1165 & 0.9074 \\
  \myhighlight making\_self\_clear & 1.6581 & 0.3864 & 4.2909 & 0.0001 \\
  engaged\_in\_game & 0.3307 & 0.2576 & 1.2840 & 0.2008 \\
  planning\_what\_to\_say & 0.5005 & 0.5327 & 0.9395 & 0.3487 \\
  gives\_encouragement & 0.4264 & 0.3792 & 1.1246 & 0.2622 \\
  \myhighlight difficult\_for\_partner\_to\_speak & -0.7200 & 0.2858 & -2.5190 & 0.0126 \\
  bored\_with\_game & 0.2163 & 0.2560 & 0.8450 & 0.3992 \\
  dislikes\_partner & -0.4318 & 0.3443 & -1.2541 & 0.2114 \\
   \hline
\end{tabular}

\begin{tabular}{rrrrr}
  \hline
 \FOMEAN & Estimate & Std. Error & t value & Pr($>$$|$t$|$) \\
  \hline
 \myhighlight contributes\_to\_successful\_completion & 1.0274 & 0.3025 & 3.3962 & 0.0008 \\
  \myhighlight making\_self\_clear & 0.8307 & 0.3934 & 2.1115 & 0.0361 \\
  \myhighlight engaged\_in\_game & 0.8850 & 0.2750 & 3.2182 & 0.0015 \\
  planning\_what\_to\_say & 0.7167 & 0.5400 & 1.3273 & 0.1860 \\
  gives\_encouragement & 0.0075 & 0.3941 & 0.0190 & 0.9848 \\
  difficult\_for\_partner\_to\_speak & -0.5975 & 0.3928 & -1.5209 & 0.1300 \\
  \myhighlight bored\_with\_game & -0.7586 & 0.2481 & -3.0572 & 0.0026 \\
  dislikes\_partner & 0.0371 & 0.3800 & 0.0977 & 0.9223 \\
   \hline
\end{tabular}

\begin{tabular}{rrrrr}
  \hline
 \NOISETOHARMONICS & Estimate & Std. Error & t value & Pr($>$$|$t$|$) \\
  \hline
\myhighlight contributes\_to\_successful\_completion & 0.7041 & 0.3404 & 2.0686 & 0.0400 \\
  \myhighlight making\_self\_clear & 1.3344 & 0.3537 & 3.7725 & 0.0002 \\
  engaged\_in\_game & 0.0954 & 0.3462 & 0.2756 & 0.7832 \\
  planning\_what\_to\_say & -0.1874 & 0.4177 & -0.4485 & 0.6543 \\
  gives\_encouragement & 0.7234 & 0.4782 & 1.5127 & 0.1321 \\
  difficult\_for\_partner\_to\_speak & -0.1941 & 0.3436 & -0.5648 & 0.5729 \\
  \myhighlight bored\_with\_game & 0.5876 & 0.3028 & 1.9404 & 0.0539 \\
  dislikes\_partner & 0.3582 & 0.3330 & 1.0755 & 0.2835 \\
   \hline
\end{tabular}

\caption{Tablas con los resultados de la regresión de efectos fijos para \ENGMAX, \FOMEAN y \NOISETOHARMONICS. En la segunda columna se cita el valor de $\estslope$, la desviación estándar calculada, el t-valor obtenido y la significancia. Las columnas resaltadas corresponden a aquellas significantes}\label{regresion_efectos_fijos_tabla}
\end{figure}

%%%% BIBLIOGRAFIA
\backmatter
\bibliographystyle{alpha}

\bibliography{tesis}
\end{document}
