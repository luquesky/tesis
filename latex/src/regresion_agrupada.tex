En esta sección, mostraremos el primer experimento realizado. Este consistió en aplicar un modelo de regresión lineal de cada variable social sobre el \entrainment, sin desagregar los datos por sesión y hablante.

Una variación que usamos en el presente experimento (y en el posterior) es utilizar como variable dependiente el \emph{valor absoluto} del \entrainment, en base a estudios que sugieren que los interlocutores pueden también \emph{diferenciarse} como un rasgo positivo en la conversación.

\section{Nuestro modelo de regresión}

Dada una variable acústico-prosódica (por ejemplo, el pitch o la intensidad), queremos investigar la relación entre entrainment y las distintas variables sociales medidas. Sea $V$ una variable social de las enumeradas en la tabla \ref{tab:panel_data}. Sean $E_1, \ldots, E_n$ los valores de entrainment para el set de datos que definimos en la sección \ref{sec:panel_data}, y sean $V_1, V_2, \ldots V_n$ los valores de la variable social de cada conversación.

Sobre estas variables es que planteamos nuestro modelo de regresión lineal clásica, para analizar qué relación hay tomando como variable ``fija'' al \entrainment, y como variable dependiente a la variable social. El problema, entonces, es hallar estimadores $\widehat{\beta_1}, \widehat{\beta_2} \in \mathbb{R}$ de modo que

\begin{equation}
  V_i \simeq \estintercept + \estslope E_i
  \label{eq:my_model}
\end{equation}


Para ello, calculamos los estimadores mediante el método \emph{QR} que nos provee el lenguaje R. A su vez, luego de esto efectuamos un análisis de significancia sobre $\estslope$ para verificar que sea estadísticamente distinto de 0.

Uno esperaría que un alto \emph{entrainment} se relacione con un alto valor de ciertas variables sociales, por ejemplo la compenetración con el juego o el ayudar a terminarlo. En términos de la ecuación \ref{eq:my_model}, esperamos que $\estslope \geq 0$. De manera inversa, cuando las variables sociales tienen un carácter negativo de la conversación, esperamos que $\estslope \leq 0$.


El modelo de regresión que usamos en este primer experimento se denomina agrupado o \emph{pooled} ya que no distinguimos entre datos provenientes de distintos ``grupos'' \cite{gujarati1999} y calculamos la regresión lineal agrupando todos los datos disponibles agrupados.

Un problema que surge con este tipo de regresión es que niega todo tipo de \emph{heterogeneidad} de los datos: estos pueden provenir de interlocutores más o menos empáticos, o cuya interacción en el juego se vio influída por factores no medidos en el experimento. Todo esto es descartado, aún cuando puede afectar seriamente  el resultado obtenido.

En el siguiente capítulo ahondaremos un poco más en cómo definimos los grupos en nuestro trabajo.

\section{Resultados sobre \entrainment}

Los resultados de este experimento no fueron interesantes ya que dieron muy pocos valores de $\estslope$ significativos. En la figura \ref{fig:regresion_clasica} puede verse el gráfico de regresión lineal del \entrainment contra distintas variables sociales, tomando como variable acústico-prosódica a \FOMEAN. En las figuras \ref{fig:regresion_clasica_tabla_1} y \ref{fig:regresion_clasica_tabla_2} podemos observar la tabla de las regresiones, con las estimaciones obtenidas y sus valores de significancia para todas las variables sociales. Se observa que no sólo las pendientes tienen un muy bajo valor absoluto, sino que además ni siquiera tienen los signos que esperábamos en un principio.

En base a lo arrojado por este análisis de regresión, intentamos introducir variaciones del experimento. La primera, es cambiar la variable explicativa por el valor absoluto del entrainment.


\begin{figure}[b!]
\includegraphics[width=15cm]{images/regression_F0_MEAN_1.pdf}
\caption{Gráfico de los pares entrainment-variable a/p, junto a la regresión lineal obtenida para \emph{F0\_MEAN}}
\label{fig:regresion_clasica}
\end{figure}


%
% Primera tabla
%

\begin{figure}
\adjustbox{max width=\textwidth}{
\begin{tabular}{rrrrr}
  \hline
\ENGMAX & $\estslope$ & Std. Error & t value & Pr($>$$|$t$|$) \\ 
  \hline
contributes\_to\_successful\_completion & -0.0558 & 0.1400 & -3.985368E-01 & 0.6906 \\ 
  making\_self\_clear & 0.1454 & 0.1475 & 9.854566E-01 & 0.3255 \\ 
  engaged\_in\_game & 0.0647 & 0.1178 & 5.494021E-01 & 0.5833 \\ 
  planning\_what\_to\_say & 0.0864 & 0.1689 & 5.115176E-01 & 0.6095 \\ 
  gives\_encouragement & -0.0699 & 0.1486 & -4.706984E-01 & 0.6383 \\ 
  difficult\_for\_partner\_to\_speak & -0.0053 & 0.1304 & -4.057895E-02 & 0.9677 \\ 
  bored\_with\_game & 0.0083 & 0.1326 & 6.230464E-02 & 0.9504 \\ 
  dislikes\_partner & -0.0937 & 0.1129 & -8.305546E-01 & 0.4072 \\ 
  \hline
\ENGMEAN & $\estslope$ & Std. Error & t value & Pr($>$$|$t$|$) \\ 
  \hline
  contributes\_to\_successful\_completion & -0.1541 & 0.1395 & -1.104898E+00 & 0.2705 \\ 
  making\_self\_clear & -0.1212 & 0.1475 & -8.214940E-01 & 0.4123 \\ 
  engaged\_in\_game & 0.0967 & 0.1176 & 8.221595E-01 & 0.4119 \\ 
  \softhl planning\_what\_to\_say & -0.2808 & 0.1677 & -1.673747E+00 & 0.0957 \\ 
  gives\_encouragement & 0.0092 & 0.1485 & 6.184983E-02 & 0.9507 \\ 
  difficult\_for\_partner\_to\_speak & -0.0579 & 0.1302 & -4.443657E-01 & 0.6572 \\ 
  bored\_with\_game & -0.0797 & 0.1324 & -6.019681E-01 & 0.5479 \\ 
  dislikes\_partner & -0.0937 & 0.1127 & -8.311282E-01 & 0.4069 \\ 
  \hline
\FOMEAN & $\estslope$ & Std. Error & t value & Pr($>$$|$t$|$) \\ 
  \hline
contributes\_to\_successful\_completion & 0.1295 & 0.1405 & 9.218693E-01 & 0.3577 \\ 
  making\_self\_clear & -0.0394 & 0.1486 & -2.648899E-01 & 0.7914 \\ 
  engaged\_in\_game & 0.1147 & 0.1182 & 9.700134E-01 & 0.3332 \\ 
  planning\_what\_to\_say & 0.0855 & 0.1698 & 5.034510E-01 & 0.6152 \\ 
  gives\_encouragement & 0.2108 & 0.1488 & 1.416825E+00 & 0.1580 \\ 
  difficult\_for\_partner\_to\_speak & -0.0886 & 0.1310 & -6.762104E-01 & 0.4997 \\ 
  bored\_with\_game & -0.0518 & 0.1333 & -3.884336E-01 & 0.6981 \\ 
  \hl dislikes\_partner & -0.2228 & 0.1126 & -1.978537E+00 & 0.0492 \\ 
   \hline
\FOMAX & $\estslope$ & Std. Error & t value & Pr($>$$|$t$|$) \\ 
  \hline
contributes\_to\_successful\_completion & -0.1848 & 0.1387 & -1.332296E+00 & 0.1842 \\ 
  \hl making\_self\_clear & -0.3568 & 0.1450 & -2.460911E+00 & 0.0147 \\ 
  engaged\_in\_game & -0.1325 & 0.1169 & -1.133286E+00 & 0.2584 \\ 
  planning\_what\_to\_say & -0.1801 & 0.1676 & -1.074476E+00 & 0.2839 \\ 
  gives\_encouragement & -0.2067 & 0.1472 & -1.404358E+00 & 0.1617 \\ 
  difficult\_for\_partner\_to\_speak & 0.0583 & 0.1296 & 4.497919E-01 & 0.6533 \\ 
  \hl bored\_with\_game & 0.3085 & 0.1301 & 2.370678E+00 & 0.0187 \\ 
  dislikes\_partner & 0.0996 & 0.1122 & 8.876897E-01 & 0.3757 \\ 
   \hline
\end{tabular}}

\caption{Tablas con los resultados de la regresión pooled sobre el \entrainment para \ENGMAX, \ENGMEAN, \FOMEAN y \FOMAX. En la segunda columna se cita el valor de $\estslope$, la desviación estándar calculada, el t-valor obtenido y la significancia. Las columnas resaltadas corresponden a aquellas significantes, con diferentes matices de gris según $p < 0.10$, $p < 0.5$, o $p < 0.01$}

\label{fig:regresion_clasica_tabla_1}
\end{figure}

%
% Second table
%
%
\begin{figure}
\adjustbox{max width=\textwidth}{
\begin{tabular}{rrrrr}
  \hline
\NOISETOHARMONICS & $\estslope$ & Std. Error & t value & Pr($>$$|$t$|$) \\ 
  \hline
contributes\_to\_successful\_completion & -0.0531 & 0.1378 & -3.854086E-01 & 0.7003 \\ 
  making\_self\_clear & 0.0235 & 0.1456 & 1.611797E-01 & 0.8721 \\ 
  engaged\_in\_game & 0.0028 & 0.1161 & 2.384333E-02 & 0.9810 \\ 
  planning\_what\_to\_say & 0.0359 & 0.1664 & 2.154899E-01 & 0.8296 \\ 
  gives\_encouragement & -0.0687 & 0.1463 & -4.697588E-01 & 0.6390 \\ 
  difficult\_for\_partner\_to\_speak & 0.0323 & 0.1284 & 2.519307E-01 & 0.8013 \\ 
  bored\_with\_game & -0.0936 & 0.1304 & -7.178558E-01 & 0.4737 \\ 
  dislikes\_partner & -0.1472 & 0.1108 & -1.328068E+00 & 0.1856 \\ 
  \hline
\PHONAVG & $\estslope$ & Std. Error & t value & Pr($>$$|$t$|$) \\ 
  \hline
contributes\_to\_successful\_completion & -0.0593 & 0.1431 & -4.143729E-01 & 0.6790 \\ 
  making\_self\_clear & -0.0093 & 0.1512 & -6.156528E-02 & 0.9510 \\ 
  engaged\_in\_game & 0.1173 & 0.1202 & 9.755995E-01 & 0.3304 \\ 
  planning\_what\_to\_say & 0.1062 & 0.1726 & 6.151568E-01 & 0.5391 \\ 
  gives\_encouragement & -0.0158 & 0.1520 & -1.037459E-01 & 0.9175 \\ 
  difficult\_for\_partner\_to\_speak & -0.0170 & 0.1333 & -1.275075E-01 & 0.8987 \\ 
  bored\_with\_game & -0.1324 & 0.1352 & -9.786726E-01 & 0.3289 \\ 
  dislikes\_partner & -0.0502 & 0.1155 & -4.350299E-01 & 0.6640 \\ 
   \hline
\SYLAVG & $\estslope$ & Std. Error & t value & Pr($>$$|$t$|$) \\ 
  \hline
contributes\_to\_successful\_completion & -0.1920 & 0.1426 & -1.347147E+00 & 0.1794 \\ 
  making\_self\_clear & -0.2043 & 0.1505 & -1.356908E+00 & 0.1763 \\ 
  engaged\_in\_game & 0.0054 & 0.1205 & 4.493146E-02 & 0.9642 \\ 
  planning\_what\_to\_say & -0.0520 & 0.1728 & -3.009701E-01 & 0.7637 \\ 
  gives\_encouragement & -0.0407 & 0.1520 & -2.677948E-01 & 0.7891 \\ 
  difficult\_for\_partner\_to\_speak & 0.0909 & 0.1332 & 6.827016E-01 & 0.4956 \\ 
  bored\_with\_game & 0.0178 & 0.1356 & 1.314632E-01 & 0.8955 \\ 
  dislikes\_partner & 0.0460 & 0.1155 & 3.980614E-01 & 0.6910 \\ 
   \hline
\LOCALJITTER & $\estslope$ & Std. Error & t value & Pr($>$$|$t$|$) \\ 
  \hline
contributes\_to\_successful\_completion & -0.1813 & 0.1385 & -1.309210E+00 & 0.1919 \\ 
  making\_self\_clear & 0.0281 & 0.1469 & 1.913441E-01 & 0.8484 \\ 
  engaged\_in\_game & 0.1072 & 0.1168 & 9.176696E-01 & 0.3599 \\ 
  planning\_what\_to\_say & -0.1635 & 0.1675 & -9.759850E-01 & 0.3302 \\ 
  gives\_encouragement & -0.0380 & 0.1476 & -2.575414E-01 & 0.7970 \\ 
  difficult\_for\_partner\_to\_speak & -0.0411 & 0.1295 & -3.175009E-01 & 0.7512 \\ 
  bored\_with\_game & -0.0164 & 0.1317 & -1.247555E-01 & 0.9008 \\ 
  dislikes\_partner & -0.0308 & 0.1122 & -2.747907E-01 & 0.7837 \\ 
   \hline
\LOCALSHIMMER & $\estslope$ & Std. Error & t value & Pr($>$$|$t$|$) \\ 
  \hline
contributes\_to\_successful\_completion & -0.0299 & 0.1407 & -2.122533E-01 & 0.8321 \\ 
  making\_self\_clear & 0.1098 & 0.1484 & 7.400752E-01 & 0.4601 \\ 
  engaged\_in\_game & -0.0214 & 0.1185 & -1.806035E-01 & 0.8569 \\ 
  planning\_what\_to\_say & 0.0283 & 0.1698 & 1.668648E-01 & 0.8676 \\ 
  gives\_encouragement & -0.1702 & 0.1489 & -1.143038E+00 & 0.2543 \\ 
  difficult\_for\_partner\_to\_speak & -0.0035 & 0.1311 & -2.645931E-02 & 0.9789 \\ 
  bored\_with\_game & 0.0431 & 0.1332 & 3.232737E-01 & 0.7468 \\ 
  dislikes\_partner & -0.0299 & 0.1136 & -2.631785E-01 & 0.7927 \\ 
   \hline
\end{tabular}}

\caption{Tablas con los resultados de la regresión agrupada sobre el \entrainment para \NOISETOHARMONICS, \SYLAVG, \PHONAVG, \LOCALSHIMMER y \LOCALJITTER. En la segunda columna se cita el valor de $\estslope$, la desviación estándar calculada, el t-valor obtenido y la significancia. Las columnas resaltadas corresponden a aquellas significantes, con diferentes matices de gris según $p < 0.10$, $p < 0.5$, o $p < 0.01$}

\label{fig:regresion_clasica_tabla_2}
\end{figure}


\section{Valor absoluto de \entrainment}

En la sección \ref{sec:method_entrainment}, definimos el \entrainment como el valor de la correlación cruzada (en un sentido de los lags) con mayor valor absoluto. Esto puede dar, como resultado, valores positivos entre 0 y 1 a los cuales consideramos como \entrainment; o bien valores negativos entre -1 y 0, estos considerados como dis-\entrainment: la divergencia de las variables \ap medidas a través del tiempo.

Este fenómeno de dis-\entrainment o antimimicry \cite{CHAR1999} refiere al proceso por el cual uno de los hablantes no imita al otro sino más bien todo lo contrario, acentúa alguna diferencia. Si bien estudios de larga data como \cite{bourhis1973language} o \cite{dabbs1969similarity} lo emparentan con una connotación negativa, \cite{healey2014divergence} y \cite{levitan2015acoustic} sugieren que puede entenderse este fenómeno como una conducta de adaptación cooperativa. No sólo eso, sino que este fenómeno de mimetización complementaria podría ser incluso más prevalente que la mimetización a secas \cite{levitan2015acoustic}.

En base a esto es que decidimos probar alguna medida que capture positivamente el fenómeno de la anti-mimetización de igual manera que con el \entrainment antes definido. Es decir, esperamos que cuando tengamos o bien \entrainment o \entrainment complementario ocurra que tenemos valores altos de variables sociales de carácter positivo. Mutatis mutandis con las variables sociales de connotación negativa.

Con este fin, en vez de utilizar sólo el valor de \entrainment como variable explicativa, efectuamos el mismo análisis pero utilizando el valor absoluto del \entrainment como tal. Usar esto permite captar y valorar el \entrainment complementario de la misma manera que el ``positivo'' y valorar su relación con las variables sociales medidas. A esta nueva métrica la llamaremos \emph{unsigned entrainment}

\section{Resultados sobre \absentrainment}

Utilizando esta variable explicativa, los resultados son bastante distintos. En las tablas \ref{fig:pooled_abs_entrainment_1} y \ref{fig:pooled_abs_entrainment_2} podemos observar que hubo al menos un resultado significativo para todas las variables \ap, exceptuando \PHONAVG.

Casi todos los resultados significativos y positivos de $\estslope$ son respecto de variables sociales de carácter positivo, como \svclear, \svengaged y \svencourages; la notable excepción es \svdifficult, que tiene un carácter negativo pero a su vez $\estslope > 0$ en varios casos. El único caso significativo donde $\estslope < 0$ es para \svbored, que era algo justamente esperado.

Habiendo reformulado anteriormente nuestra hipótesis, estos resultados dan indicio de que el valor absoluto del \entrainment se relaciona con las variables sociales medidas, de manera positiva para aquellas favorables para la conversación, y de manera inversa para aquellas contrarias. Sin embargo, consideramos que en esta asociación influyen factores no medidos dentro de cada conversación, por lo cual planteamos un segundo experimento que contemple esta \emph{heterogeneidad} para analizar mejor cómo interactúan el \entrainment con los rasgos sociales.


%
% Primera tabla
%

% "ENG_MAX"
\begin{figure}

\begin{tabular}{rrrrr}

  \hline
\ENGMAX & Estimate & Std. Error & t value & Pr($>$$|$t$|$) \\ 
  \hline
contributes\_to\_successful\_completion & -0.1851 & 0.3852 & -4.806290E-01 & 0.6313 \\ 
  \stronghl making\_self\_clear & 1.2502 & 0.3977 & 3.143535E+00 & 0.0019 \\ 
  engaged\_in\_game & 0.3906 & 0.3233 & 1.207901E+00 & 0.2285 \\ 
  planning\_what\_to\_say & 0.1613 & 0.4651 & 3.467699E-01 & 0.7291 \\ 
  gives\_encouragement & 0.6711 & 0.4066 & 1.650711E+00 & 0.1003 \\ 
  difficult\_for\_partner\_to\_speak & -0.4136 & 0.3578 & -1.155966E+00 & 0.2490 \\ 
  bored\_with\_game & 0.0760 & 0.3650 & 2.081799E-01 & 0.8353 \\ 
  dislikes\_partner & -0.4139 & 0.3098 & -1.335942E+00 & 0.1830 \\ 
  \hline
\ENGMEAN & Estimate & Std. Error & t value & Pr($>$$|$t$|$) \\ 
  \hline
contributes\_to\_successful\_completion & 0.4632 & 0.4021 & 1.151825E+00 & 0.2507 \\ 
  making\_self\_clear & 0.6432 & 0.4237 & 1.517942E+00 & 0.1305 \\ 
  \hl engaged\_in\_game & 0.7620 & 0.3355 & 2.271214E+00 & 0.0242 \\ 
  planning\_what\_to\_say & 0.0213 & 0.4869 & 4.365416E-02 & 0.9652 \\ 
  gives\_encouragement & 0.5379 & 0.4267 & 1.260484E+00 & 0.2089 \\ 
  \softhl difficult\_for\_partner\_to\_speak & 0.6644 & 0.3729 & 1.781913E+00 & 0.0762 \\ 
  bored\_with\_game & -0.1525 & 0.3819 & -3.992020E-01 & 0.6902 \\ 
  dislikes\_partner & 0.4595 & 0.3241 & 1.417936E+00 & 0.1577 \\ 
   \hline
\FOMEAN & Estimate & Std. Error & t value & Pr($>$$|$t$|$) \\ 
  \hline
  contributes\_to\_successful\_completion & 0.2325 & 0.3918 & 5.933090E-01 & 0.5536 \\ 
  making\_self\_clear & 0.1139 & 0.4141 & 2.749784E-01 & 0.7836 \\ 
  \hl engaged\_in\_game & 0.8316 & 0.3251 & 2.558367E+00 & 0.0112 \\ 
  planning\_what\_to\_say & -0.0011 & 0.4733 & -2.364724E-03 & 0.9981 \\ 
  gives\_encouragement & 0.4261 & 0.4153 & 1.025888E+00 & 0.3061 \\ 
  difficult\_for\_partner\_to\_speak & 0.0088 & 0.3652 & 2.411215E-02 & 0.9808 \\ 
  \hl bored\_with\_game & -0.8747 & 0.3664 & -2.387296E+00 & 0.0179 \\ 
  dislikes\_partner & -0.2077 & 0.3162 & -6.569472E-01 & 0.5119 \\ 
   \hline
\FOMAX & Estimate & Std. Error & t value & Pr($>$$|$t$|$) \\ 
  \hline
  contributes\_to\_successful\_completion & 0.6593 & 0.4004 & 1.646479E+00 & 0.1012 \\ 
  making\_self\_clear & 0.5479 & 0.4240 & 1.292389E+00 & 0.1977 \\ 
  \softhl engaged\_in\_game & 0.5883 & 0.3369 & 1.746412E+00 & 0.0822 \\ 
  planning\_what\_to\_say & 0.1313 & 0.4864 & 2.700229E-01 & 0.7874 \\ 
  gives\_encouragement & 0.4879 & 0.4266 & 1.143779E+00 & 0.2540 \\ 
  difficult\_for\_partner\_to\_speak & 0.0605 & 0.3753 & 1.610613E-01 & 0.8722 \\ 
  bored\_with\_game & -0.3913 & 0.3807 & -1.027693E+00 & 0.3053 \\ 
  dislikes\_partner & 0.1286 & 0.3252 & 3.953927E-01 & 0.6930 \\ 
   \hline
\end{tabular}

\caption{Tablas con los resultados de la regresión pooled sobre el absolute value \entrainment para \ENGMAX, \ENGMEAN, \FOMEAN y \FOMAX. En la segunda columna se cita el valor de $\estslope$, la desviación estándar calculada, el t-valor obtenido y la significancia. Las columnas resaltadas corresponden a aquellas significantes, con diferentes matices de gris según $p < 0.10$, $p < 0.5$, o $p < 0.01$}

\label{fig:pooled_abs_entrainment_1}
\end{figure}

%
% Second table
%
%
\begin{figure}
\begin{tabular}{rrrrr}
  \hline
\NOISETOHARMONICS & $\estslope$ & Std. Error & t value & Pr($>$$|$t$|$) \\ 
  \hline
  contributes\_to\_successful\_completion & 0.3012 & 0.3668 & 8.212853E-01 & 0.4124 \\ 
  \softhl making\_self\_clear & 0.6912 & 0.3850 & 1.795341E+00 & 0.0741 \\ 
  engaged\_in\_game & 0.3573 & 0.3083 & 1.159121E+00 & 0.2477 \\ 
  planning\_what\_to\_say & -0.6433 & 0.4411 & -1.458137E+00 & 0.1463 \\ 
  \hl gives\_encouragement & 0.9573 & 0.3843 & 2.490692E+00 & 0.0135 \\ 
  difficult\_for\_partner\_to\_speak & 0.2473 & 0.3417 & 7.237456E-01 & 0.4700 \\ 
  bored\_with\_game & 0.1127 & 0.3478 & 3.239352E-01 & 0.7463 \\ 
  dislikes\_partner & 0.1021 & 0.2964 & 3.445479E-01 & 0.7308 \\ 
  \hline
\PHONAVG & $\estslope$ & Std. Error & t value & Pr($>$$|$t$|$) \\ 
  \hline
  contributes\_to\_successful\_completion & 0.4244 & 0.4041 & 1.050214E+00 & 0.2948 \\ 
  making\_self\_clear & 0.4874 & 0.4266 & 1.142560E+00 & 0.2545 \\ 
  engaged\_in\_game & 0.3743 & 0.3401 & 1.100468E+00 & 0.2724 \\ 
  planning\_what\_to\_say & 0.0675 & 0.4891 & 1.379776E-01 & 0.8904 \\ 
  gives\_encouragement & 0.3814 & 0.4294 & 8.882870E-01 & 0.3754 \\ 
  difficult\_for\_partner\_to\_speak & -0.2884 & 0.3768 & -7.652857E-01 & 0.4450 \\ 
  bored\_with\_game & -0.1753 & 0.3836 & -4.570691E-01 & 0.6481 \\ 
  dislikes\_partner & -0.0253 & 0.3271 & -7.746968E-02 & 0.9383 \\ 
  \hline
\SYLAVG & $\estslope$ & Std. Error & t value & Pr($>$$|$t$|$) \\ 
  \hline
  contributes\_to\_successful\_completion & 0.2603 & 0.3843 & 6.774699E-01 & 0.4989 \\ 
  making\_self\_clear & 0.4636 & 0.4050 & 1.144625E+00 & 0.2537 \\ 
  \hl engaged\_in\_game & 0.6655 & 0.3205 & 2.076166E+00 & 0.0391 \\ 
  planning\_what\_to\_say & 0.2054 & 0.4641 & 4.425111E-01 & 0.6586 \\ 
  \softhl gives\_encouragement & 0.7216 & 0.4054 & 1.780238E+00 & 0.0765 \\ 
  \hl difficult\_for\_partner\_to\_speak & 0.7099 & 0.3548 & 2.000706E+00 & 0.0467 \\ 
  \hl bored\_with\_game & -0.7740 & 0.3603 & -2.148100E+00 & 0.0329 \\ 
  dislikes\_partner & -0.2765 & 0.3099 & -8.921180E-01 & 0.3734 \\ 
  \hline
\LOCALJITTER & $\estslope$ & Std. Error & t value & Pr($>$$|$t$|$) \\ 
  \hline
  contributes\_to\_successful\_completion & 0.2577 & 0.3726 & 6.916180E-01 & 0.4899 \\ 
  making\_self\_clear & 0.2344 & 0.3936 & 5.955030E-01 & 0.5522 \\ 
  \softhl engaged\_in\_game & 0.5409 & 0.3118 & 1.734767E+00 & 0.0843 \\ 
  planning\_what\_to\_say & -0.3620 & 0.4496 & -8.053115E-01 & 0.4216 \\ 
  gives\_encouragement & 0.3403 & 0.3954 & 8.608149E-01 & 0.3903 \\ 
  difficult\_for\_partner\_to\_speak & -0.1179 & 0.3473 & -3.395874E-01 & 0.7345 \\ 
  bored\_with\_game & -0.0256 & 0.3533 & -7.251562E-02 & 0.9423 \\ 
  dislikes\_partner & -0.1089 & 0.3010 & -3.618293E-01 & 0.7178 \\ 
  \hline
\LOCALSHIMMER & $\estslope$ & Std. Error & t value & Pr($>$$|$t$|$) \\ 
  \hline
  contributes\_to\_successful\_completion & 0.2636 & 0.3712 & 7.101116E-01 & 0.4784 \\ 
  making\_self\_clear & -0.0986 & 0.3925 & -2.512623E-01 & 0.8019 \\ 
  engaged\_in\_game & 0.3814 & 0.3118 & 1.223352E+00 & 0.2226 \\ 
  planning\_what\_to\_say & -0.7361 & 0.4457 & -1.651521E+00 & 0.1001 \\ 
  \softhl gives\_encouragement & 0.6756 & 0.3918 & 1.724198E+00 & 0.0862 \\ 
  difficult\_for\_partner\_to\_speak & 0.4077 & 0.3450 & 1.181785E+00 & 0.2386 \\ 
  bored\_with\_game & -0.5326 & 0.3501 & -1.521345E+00 & 0.1297 \\ 
  dislikes\_partner & -0.2428 & 0.2996 & -8.104076E-01 & 0.4186 \\ 
   \hline
\end{tabular}

\caption{Tablas con los resultados de la regresión agrupada sobre absolute value entrainment para \NOISETOHARMONICS, \SYLAVG, \PHONAVG, \LOCALSHIMMER y \LOCALJITTER. En la segunda columna se cita el valor de $\estslope$, la desviación estándar calculada, el t-valor obtenido y la significancia. Las columnas resaltadas corresponden a aquellas significantes, con diferentes matices de gris según $p < 0.10$, $p < 0.5$, o $p < 0.01$}

\label{fig:pooled_abs_entrainment_2}

\end{figure}



