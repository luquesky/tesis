\section{Series de Tiempo}

\subsection*{Definición Informal}
En términos informales, una serie de tiempo es un conjunto de datos recolectados secuencialmente en el tiempo. Este tipo de datos se dan en varios campos de estudio, mayormente en economía, ciencias de la atmósfera, y otras.

Ejemplos de series de tiempo:

\begin{itemize}
    \item Volumen de lluvias en sucesivos días de un año
    \item Precio de acciones en diferentes meses
    \item Cantidad de habitantes de una ciudad año a año
\end{itemize}

\begin{figure}
\centering
\includegraphics[width=15cm]{images/desocupacion.jpg}
\caption{Gráfico de serie de tiempo de la evolución del desempleo en Argentina \label{desocupacion}}
\end{figure}


\subsection*{¿Para qué queremos series de tiempo?}

Hay varios motivos por los cuales uno querría efectuar un análisis de una serie de tiempo.

\emph{1) Descripción} Usualmente, lo primero que se hace al obtener la serie de tiempo es graficarla y obtener las características más notorias de ésta. Por ejemplo, en \ref{desocupacion} puede notarse que hay una tendencia decreciente del $2003$ hasta el $2012$. En otras (como en el volumen de lluvias) podrá observarse cierta estacionalidad en la serie.

Si bien ésto no requiere técnicas avanzadas de análisis, es el primer paso fundamental para comprender una serie de tiempo.


\emph{2) Explicación} Cuando analizamos dos o más series de tiempo, podemos querer ver cómo se comportan en conjunto. Una variación en una serie de tiempo puede producir un cambio en otra. Por ejemplo, podemos intentar buscar como varían en conjunto la temperatura diaria con la cantidad de mL de lluvia caídos.

\emph{3) Predicción} Dada una serie de tiempo, podemos querer intentar predecir un valor futuro.

\emph{4) Control} Dado un proceso del que se mide cierto parámetro de calidad, podemos querer ajustar variables de entrada para mantenerla en ciertos valores.

En nuestro caso, nos es de interés 1 y 2.


\subsection*{Definición formal}

Una proceso estocástico es una colección de variables aleatorias $\{X_t \}_{t \in T}$ donde $T$ es un conjunto de puntos de tiempo. En nuestro caso, nos interesa $T = \mathbb{N}$, de manera que el proceso será de la forma $X_1, X_2, \ldots $. Podemos entender un proceso estocástico como un conjunto de variables ordenadas por el tiempo.

Llamamos serie de tiempo a una observación de este proceso estocástico. Usualmente sólo tendremos esta instancia, a diferencia de otros problemas estadísticos donde tendremos muchas observaciones.

