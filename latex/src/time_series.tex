\section{Series de Tiempo}

\subsection*{Definición Informal}
En términos informales, una serie de tiempo es un conjunto de datos recolectados secuencialmente en el tiempo. Este tipo de datos se dan en varios campos de estudio, mayormente en economía, ciencias de la atmósfera, y otras.

Ejemplos de series de tiempo:

\begin{itemize}
    \item Volumen de lluvias en sucesivos días de un año
    \item Precio de acciones en diferentes meses
    \item Cantidad de habitantes de una ciudad año a año
\end{itemize}

\begin{figure}
\centering
\includegraphics[width=15cm]{images/desocupacion.jpg}
\caption{Gráfico de serie de tiempo de la evolución del desempleo en Argentina \label{desocupacion}}
\end{figure}


\subsection*{¿Para qué queremos series de tiempo?}

Hay varios motivos por los cuales uno querría efectuar un análisis de una serie de tiempo.

\emph{1) Descripción} Usualmente, lo primero que se hace al obtener la serie de tiempo es graficarla y obtener las características más notorias de ésta. Por ejemplo, en \ref{desocupacion} puede notarse que hay una tendencia decreciente del $2003$ hasta el $2012$. En otras (como en el volumen de lluvias) podrá observarse cierta estacionalidad en la serie.

Si bien ésto no requiere técnicas avanzadas de análisis, es el primer paso fundamental para comprender una serie de tiempo.

