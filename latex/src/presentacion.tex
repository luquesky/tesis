\documentclass{beamer}
\usepackage[spanish]{babel}
\usepackage[T1]{fontenc}
\usepackage[utf8]{inputenc}
\usepackage{lmodern}

\uselanguage{spanish}
\languagepath{spanish}

\def\tituloTesis{Métricas de mimetización acústico-prosódica en hablantes y su relación con rasgos sociales de diálogos}

\title{\tituloTesis}
\author{Juan Manuel Pérez}
\usepackage{amssymb}
\usepackage{graphicx}
\usepackage{amsthm}
\usepackage{amsmath}
\usepackage{mathtools}



\usetheme{Madrid}

\begin{document}


\frame{\titlepage}


\section{Introducción}

\begin{frame}
  \frametitle{Lo que voy a contar hoy}
  \centering{}
  \begin{itemize}
    \item Sistemas de diálogo Humano-Computadora
    \item Mimetización o entrainment
    \item Métricas de mimetización
    \item Time Aligned Moving Average
    \item
  \end{itemize}

\end{frame}


\begin{frame}
  \frametitle{Sistemas de diálogo Humano-Computadora}

  \begin{enumerate}
    \item Sistemas actuales
    \item Bien en la parte lingüística de la comunicación
    \item Mal en la parte superestructural, intercambio de emociones y prosodia
  \end{enumerate}
\end{frame}


\begin{frame}
  \frametitle{Ejemplos}
  \begin{enumerate}
    \item HAL
    \item Sistemas de llamadas
    \item otros
  \end{enumerate}
\end{frame}


\begin{frame}
  \frametitle{Prosodia}
  \framesubtitle{Definir acá qué es la prosodia aproximadamente}

\end{frame}



\begin{frame}
  \frametitle{Entrainment}

  \begin{enumerate}
    \item Fenómeno ubícuo en la comunicación
    \item Ocurre a varios niveles
    \item Largamente estudiado en psicología de comportamiento (referencias)
    \item (ya está, la próxima conversación que tengan afuera de acá van a chequear ésto)
  \end{enumerate}
\end{frame}

\begin{frame}
  \frametitle{¿Y cómo lo medimos?}
  La definición de entrainment hasta acá vista es heurística! ¿Cómo definimos una medida para esto?

  Vamos a explorar una métrica definida en trabajos anteriores, pulirla un poco, y verificar que efectivamente capture ciertas características del entrainment.

  ¿Cómo? Usando un corpus con anotaciones sociales


\end{frame}


\section{Antecedentes}

\begin{frame}
  \frametitle{Otras Métricas}
\end{frame}

\begin{frame}
  \frametitle{Otras Métricas}
  \subtitle{Problemas}
\end{frame}



\begin{frame}
  \frametitle{Método TAMA}
\end{frame}


\begin{frame}
  \frametitle{Método TAMA}
  \subtitle{Ventajas}
\end{frame}


\section{Materiales y Métodos}

\subsection{Corpus}
\begin{frame}
  \frametitle{Columbia Games Corpus}
  \framesubtitle{Descripción}

\end{frame}


\begin{frame}
  \frametitle{Columbia Games Corpus}
  \framesubtitle{Juegos de Objeto}

\end{frame}


\begin{frame}
  \frametitle{Columbia Games Corpus}
  \framesubtitle{Anotaciones sociales}

\end{frame}




\end{document}
