\documentclass{beamer}
\usepackage[spanish]{babel}
\usepackage[T1]{fontenc}
\usepackage[utf8]{inputenc}
\usepackage{lmodern}
\usepackage{amssymb}
\usepackage{graphicx}
\usepackage{amsthm}
\usepackage{amsmath}
\usepackage{mathtools}


\uselanguage{spanish}
\languagepath{spanish}


\def\tituloTesis{Métricas de mimetización acústico-prosódica en hablantes y su relación con rasgos sociales de diálogos}

\title{\tituloTesis}
\author{Juan Manuel Pérez}

\usetheme{Madrid}

\begin{document}


\frame{\titlepage}


\section{Abstract}

\begin{frame}
  \frametitle{¿Cómo dijo?}


\begin{itemize}[<+->]
  \item Sistemas de diálogo humano-computadora son cada vez más frecuentes, y sus aplicaciones comprenden una amplia gama de rubros
  \item Bien en la dimensión lingüística, mal en todo lo superestructural: emociones, actitudes, intenciones.
  \item Mimetización: Fenómeno insconsciente que se manifiesta a través de la adaptación de los hablantes. Fuertemente emparentada con el sentimiento de empatía.
  \item Objetivo del trabajo: Explorar y refinar una métrica de la mimetización acústico-prosódica, y validar que capture ciertas percepciones sociales en un corpus de diálogos en inglés.
\end{itemize}

\end{frame}


\section{Introducción}

\begin{frame}
  \frametitle{Temario de la tesis}
  \centering{}
  \begin{itemize}
    \item Sistemas de diálogo Humano-Computadora
    \item Mimetización o entrainment
    \item Métricas de mimetización
    \item Time Aligned Moving Average
    \item
  \end{itemize}

\end{frame}


\begin{frame}
  \frametitle{Sistemas de diálogo Humano-Computadora}

  \begin{enumerate}[<+->]
    \item Sistemas actuales
    \item Bien en la parte lingüística de la comunicación: entender y transmitir mensajes estructuralmente correctos.
    \item Mal en la parte superestructural: intercambio de emociones, actitudes, etc.
    \item El presente trabajo trata de hacer un (pequeño) aporte sobre el análisis de la ``naturalidad'' de las conversaciones.
  \end{enumerate}
\end{frame}


\begin{frame}
  \begin{columns}
    \column{0.5\textwidth}
    \frametitle{Ejemplos de ``falta de naturalidad''}
    \begin{enumerate}
      \item Sistemas de llamadas comerciales
      \item Siri, Google Now
      \item Otros?
    \end{enumerate}
    \column{0.5\textwidth}
    \begin{figure}
      \includegraphics[scale=0.25]{images/hal.jpg}
    \end{figure}
  \end{columns}
\end{frame}


\begin{frame}
  \frametitle{Prosodia}
  \framesubtitle{Definir acá qué es la prosodia aproximadamente}

\end{frame}



\begin{frame}
  \frametitle{Entrainment}

  \begin{enumerate}
    \item Fenómeno ubícuo en la comunicación
    \item Ocurre a varios niveles
    \item Largamente estudiado en psicología de comportamiento (referencias)
    \item (ya está, la próxima conversación que tengan afuera de acá van a chequear ésto)
  \end{enumerate}
\end{frame}

\begin{frame}
  \frametitle{¿Y cómo lo medimos?}
  La definición de entrainment hasta acá vista es heurística! ¿Cómo definimos una medida para esto?

  Vamos a explorar una métrica definida en trabajos anteriores, pulirla un poco, y verificar que efectivamente capture ciertas características del entrainment.

  ¿Cómo? Usando un corpus con anotaciones sociales


\end{frame}


\section{Antecedentes}

\begin{frame}
  \frametitle{Otras Métricas}
\end{frame}

\begin{frame}
  \frametitle{Otras Métricas}
  \subtitle{Problemas}
\end{frame}



\begin{frame}
  \frametitle{Método TAMA}
\end{frame}


\begin{frame}
  \frametitle{Método TAMA}
  \subtitle{Ventajas}
\end{frame}


\section{Materiales y Métodos}

\subsection{Corpus}
\begin{frame}
  \frametitle{Columbia Games Corpus}
  \framesubtitle{Descripción}

\end{frame}


\begin{frame}
  \frametitle{Columbia Games Corpus}
  \framesubtitle{Juegos de Objeto}

\end{frame}


\begin{frame}
  \frametitle{Columbia Games Corpus}
  \framesubtitle{Anotaciones sociales}

\end{frame}




\end{document}
