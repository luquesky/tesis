\subsection{Corpus}
\begin{frame}
  \frametitle{Columbia Games Corpus}
  \framesubtitle{Descripción}
  \begin{columns}
    \column{0.35\textwidth}
      \begin{figure}
        \includegraphics[width=\textwidth]{images/columbia_games_color.jpg}
      \end{figure}

    \column{0.65\textwidth}

    \begin{itemize}
      \item Corpus de conversaciones diádicas en Inglés Americano Estándar
      \item 12 sesiones con 14 tareas/juegos cada una.
      \item En cada sesión, se sentó a dos participantes en una cabina profesional de grabación, y una cortina opaca colgando entre ellos para evitar la comunicación visual.
      \item Los participantes contaron con computadoras a través de las cuales interactuaban mediante juegos.
    \end{itemize}
  \end{columns}

\end{frame}


\begin{frame}
  \frametitle{Columbia Games Corpus}
  \framesubtitle{Juegos de objeto}
  \begin{columns}
    \column{0.35\textwidth}
      \begin{figure}
        \includegraphics[width=\textwidth]{images/columbia_games_color.jpg}
      \end{figure}

    \column{0.65\textwidth}

    \begin{itemize}
      \item Dos roles: Descriptor y Seguidor
      \item En la pantalla, se ven entre 5 y 7 objetos en posiciones aleatorias
      \item El Descriptor ve uno más, titilante, del cual debe describir su posición
      \item El Seguidor debe mover la representación del objeto a la misma posición de la pantalla
      \item Finalmente, se puntúa de 1 a 100 el posicionamiento del objeto
    \end{itemize}
  \end{columns}
\end{frame}





\begin{frame}
  \frametitle{Columbia Games Corpus}
  \framesubtitle{Anotaciones sociales}

  Cinco anotadores escucharon el audio correspondiente a una tarea del juego y respondieron a varias preguntas sobre los sujetos:


\begin{table}
\adjustbox{max width=0.8\textwidth}{
\begin{tabular}{|l|l|}
  \hline

  \textbf{Nombre} & \textbf{Pregunta} \\
  \hline

  \svcontributes &  ¿el sujeto contribuye para el éxito del equipo?  \\ \hline
  \svengaged &  ¿el sujeto parece comprometido con el juego? \\ \hline
  \svclear &  ¿el sujeto se expresa correctamente? \\ \hline
  \svplanning &  ¿el sujeto piensa lo que va a decir? \\ \hline
  \svencourages &  ¿el sujeto alienta a su compañero? \\ \hline
  \svdifficult &  ¿el sujeto le hace difícil hablar a su compañero? \\ \hline
  \svbored &  ¿el sujeto está aburrido con el juego? \\ \hline
  \svdislikes &  ¿al sujeto no le agrada su compañero? \\ \hline
\end{tabular}
}
\end{table}

De cada una de estas preguntas obtenemos un puntaje de 0 a 5, para cada hablante de cada tarea.
\end{frame}


\begin{frame}
\frametitle{Extracción de features acústico-prosódicas}

Usando el software Praat \footnote{http://www.fon.hum.uva.nl/praat/} se extrajeron las variables acústico-prosódicas para cada segmento de habla

\begin{table}
\adjustbox{max width=0.8\textwidth}{
\centering
\begin{tabular} {|c|c|}
  \hline
  Variable & Descripción \\
  \hline
  \FOMEAN & Valor medio de la frecuencia fundamental \\\hline
  \FOMAX  & Valor máximo de la frecuencia fundamental \\\hline
  \ENGMEAN & Valor medio de la intensidad \\\hline
  \ENGMAX & Valor máximo de la intensidad \\\hline
  \NOISETOHARMONICS & Noise-to-harmonics ratio \\\hline
  \LOCALSHIMMER & Shimmer medido \\\hline
  \LOCALJITTER  & Jitter medido \\\hline
  \SYLAVG & Cantidad de sílabas por segundo \\\hline
  \PHONAVG & Cantidad de fonemas por segundo \\\hline
\end{tabular}
}
\end{table}

\end{frame}
