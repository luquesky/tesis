\section{Panel de datos}
\label{sec:panel_data}

Luego de construir las series de tiempo para cada una de las conversaciones que seleccionamos anteriormente, pasamos a construir una gran tabla que se utilizó en los análisis de regresión detallados en la siguiente sección. Para condensar todos nuestros datos, armamos una tabla por cada variable acústico-prosódica que contiene información definida para cada interlocutor y tarea de nuestro corpus.

Cada fila de esta tabla representa los datos de un hablante dentro de una tarea. Este hecho lo usamos fuertemente a la hora de definir los grupos en nuestro modelo de Efectos Fijos. En la tabla \ref{tab:panel_data} se describen las columnas generadas.

\begin{table}[b!]
\centering
\adjustbox{max width=\textwidth}{
\begin{tabular}{|c|c|}
  \hline
  \hl \emph{Campo} & \emph{Descripción} \\\hline
  session & número de sesión del corpus (1-12)\\\hline
  speaker & 0 si corresponde al interlocutor A; B en otro caso \\\hline
  task & número de tarea en la sesión (1-14) \\\hline
  count & La cantidad de puntos definidos que tiene la serie \\\hline
  entrainment & Si $speaker=0$, es $\fwentrainment{AB}$; $\fwentrainment{BA}$ en otro caso \\\hline
  best\_lag & el lag del cross-correlogram donde se logra el \emph{entrainment} \\\hline

  engaged\_in\_game & ¿el sujeto parece comprometido con el juego? \\\hline
  difficult\_for\_partner\_to\_speak & ¿al interlocutor se le dificulta hablar? \\\hline
  contributes\_to\_successful\_completion & ¿el sujeto contribuye para el éxito del equipo? \\\hline
  gives\_encouragement & ¿el sujeto alienta a su compañero?\\\hline
  making\_self\_clear & ¿el sujeto se expresa con claridad?\\\hline

  planning\_what\_to\_say & ¿el sujeto piensa lo que va a decir? \\\hline
  bored\_with\_game & ¿el sujeto se muestra aburrido? \\\hline
  dislikes\_partner &  ¿al sujeto no le agrada su compañero? \\\hline
\end{tabular}
}
\caption{Columnas de la tabla generada para ser utilizada en los análisis de regresión lineal}
\label{tab:panel_data}
\end{table}



La tabla generada tuvo una dimensión de 210 x 14, siendo 210 la cantidad de tareas (contadas dos veces por cada hablante) y 14 las columnas mencionadas en la tabla \ref{tab:panel_data}. Una forma de ver esta tabla es que, para cada sesión y hablante, tenemos una serie de tiempo sobre las tareas   siendo los datos el grado del \entrainment y las variables sociales. En la jerga econométrica, llamamos a este tipo de datos \emph{de panel}\cite{gujarati1999}: un conjunto de mediciones temporales sobre un mismo sujeto a lo largo del tiempo. En este caso el sujeto es un hablante en una sesión, el tiempo son las tareas, y las mediciones son los valores medidos de \entrainment y las diferentes variables sociales.

En la tabla \ref{tab:panel_data_example} tenemos una sección de la tabla. Los sujetos que tenemos en este ejemplo son 3: $speaker = 0$ y $session=1$, $speaker = 1$ y $session=1$, y $speaker = 0$ y $session=2$. También tenemos cinco series de tiempo para cada sujeto: \entrainment, \emph{bored}, \emph{engaged}, \emph{encourages} y \emph{clear}. Vale la pena remarcar que estas series de tiempo, al igual que las que consideramos en la construcción de TAMA, pueden tener datos faltantes ya que, como fue descripto en la sección \ref{sec:time_plots}, no tomamos todas las tareas de todas las sesiones sino aquellas que tienen cierta calidad de diálogo.


\begin{figure}
\centering
\adjustbox{max width=\textwidth}{
\begin{tabular}{lrrrrrrrrr}
\toprule
session &  speaker &  task &  entrainment &  bored\_with\_game &  engaged\_in\_game &  gives\_encouragement &  making\_self\_clear &  planning\_what\_to\_say \\
\midrule
1 &        0 &    10 &     0.581475 &                0 &                5 &                    5 &                  5\\
1 &        0 &    12 &    -0.569677 &                1 &                5 &                    5 &                  5\\
1 &        0 &    13 &     0.533701 &                2 &                4 &                    5 &                  4\\
1 &        1 &    10 &    -0.917101 &                0 &                5 &                    2 &                  3\\
1 &        1 &    12 &     0.467112 &                0 &                5 &                    4 &                  2\\
1 &        1 &    13 &    -0.602364 &                0 &                5 &                    4 &                  3\\
2 &        0 &     3 &     0.520696 &                0 &                4 &                    5 &                  5\\
2 &        0 &     4 &    -0.241060 &                0 &                5 &                    4 &                  4\\
2 &        0 &     7 &     0.743719 &                0 &                5 &                    4 &                  5\\
2 &        0 &     8 &     0.147362 &                0 &                5 &                    4 &                  2\\
\bottomrule
\end{tabular}
}


\caption{Extracto de la tabla generada para \FOMEAN}
\label{tab:panel_data_example}
\end{figure}
