
\begin{table}[t]
\begin{figure}[ht]
\centering
% psl is "Positive Slope"
\newcommand{\psl} { $+$ }
% nsl stands for "Negative SLope"
\newcommand{\nsl} { $-$ }


\begin{tabular}{| c | c | c | c | c | c |}
  \hline
 & ENG\_MAX & ENG\_MEAN & F0\_MEAN & F0\_MAX & NOISERATIO  \\
  \hline
contributes  &      &  & \psl &  & \psl \\ \hline
  clear     & \psl &  & \psl &  & \psl \\ \hline
  engaged    &      &  & \psl &  &      \\ \hline
  planning   &      &  &      &  &      \\ \hline
  encourages &      &  &      &  &      \\ \hline
  difficult  & \nsl &  &      &  &      \\ \hline
  bored      &      &  & \nsl &  &      \\ \hline
  dislikes   &      &  &      &  &      \\ \hline
   \hline
\end{tabular}

\adjustbox{max width=\textwidth}{
\begin{tabular}{| c | c | c | c | c | c | c |}
  \hline
& PHON\_AVG & PHON\_COUNT & SHIMMER & SYL\_AVG & SYL\_COUNT & VCD2TOT \\
  \hline
contributes  &      &  &  &  &      &  \\ \hline
  clear     & \psl &  &  &  & \psl &  \\ \hline
  engaged    &      &  &  &  &      &  \\ \hline
  planning   &      &  &  &  &      &  \\ \hline
  encourages &      &  &  &  &      &  \\ \hline
  difficult  &      &  &  &  &      &  \\ \hline
  bored      &      &  &  &  &      &  \\ \hline
  dislikes   &      &  &  &  &      &  \\ \hline
  \hline
\end{tabular}
}

\caption{Tabla que representa los resultados significantes del experimento. En una de las entradas, tenemos los nombres abreviados de las variables sociales, y en la otra las variables a/p. El símbolo \psl representa valor significante y positivo de la pendiente de la regresión de efectos fijos, mientras que \nsl representa significante y negativo }

\label{sign_table}

\end{figure}

\caption{Resultados de la regresión de efectos fijos. El símbolo \psl representa valor significante y positivo de la pendiente de la regresión de efectos fijos, mientras que \nsl representa significante y negativo. \psl representa $p < 0.10$, \ppsl $p < 0.5$, y \pppsl $p < 0.01$. Análogamente para \nsl, \nnsl, y \nnnsl }
\label{sign_table}
\end{table}

\newcommand{\slopeestim}[1] { $\estslope \sim #1$ }

La primer métrica que definimos, $\fwentrainment{AB}^{(1)}$, no obtuvo resultados significativos para las pendientes, por lo cual las omitimos del presente análisis. Por otro lado, la métrica $\fwentrainment{AB}^{(2)}$ dio valores sustancialmente apreciables. La Tabla \ref{sign_table} presenta una versión resumida de los resultados para la regresión de las variables sociales sobre esta métrica, para cada una de las variables \ap, marcando con \psl (aproximadamente significativo), \ppsl si $p < 0.05$ y \pppsl si $p < 0.01$.

Podemos observar que para las primeras cinco variables \ap, y las tres primeras variables sociales tenemos un conjunto de pendientes positivas significativas (o aproximadamente). Estas variables (\svcontributes, \svclear, y \svengaged) representan percepciones positivas, que era lo que esperábamos ver: a mayor valor de \entrainment, mayor valor de estas variables sociales positivas. Esta tendencia es principalmente notoria para las variables \ap relacionadas al tono y a la intensidad. Para aquellas variables sociales de connotación negativa también se encontraron pendientes menores a cero, aunque en este caso tan sólo dos son significativas.

A su vez, puede destacarse que esta medida del entrainment es consistente con otras métricas definidas en otros trabajos, como las construídas en \cite{gravano2015backward} sobre anotaciones discretas de los patrones entonacionales usando la convención ToBI\cite{pitrelli1994evaluation}.
