
\newcommand{\slopeestim}[1] { $\estslope \sim #1$ }


Este modelo, utilizando como variable independiente al valor absoluto del \entrainment, dio valores sustancialmente apreciables. Casi todas las variables \ap poseen al menos un valor significativo de $\estslope$, destacándose \ENGMEAN, \NOISETOHARMONICS y \FOMEAN  con 3, 3 y 4 valores significativos respectivamente. Una versión simplificada tabla la podemos ver en la tabla \ref{sign_table} que grafica mediante tabla de doble entrada aquellos pares de variables \ap y variables sociales con coeficientes significativos y su signo.

Con respecto a las variables sociales, podemos observar que:

\begin{itemize}
  \item \svcontributes se relaciona positivamente con el \absentrainment cuando la variable acústico-prosódica medida es \FOMEAN o bien \NOISETOHARMONICS. Esto significa que, cuando sube el valor absoluto del \entrainment, esta variable positiva también lo hace con buena probabilidad. Esto es un efecto esperable: cuando hay mimetización, hay colaboración para el éxito en el juego.
  \item \svclear, otra variable que refleja una visión positiva del juego, también se relaciona positivamente con el \absentrainment para las variables \FOMEAN, \NOISETOHARMONICS, \ENGMAX como a su vez para \PHONAVG
  \item \svengaged, de la misma manera que las dos anteriores, relaciona positivamente pero sólo con \FOMEAN
  \item \svdifficult, se relaciona de la manera esperada con el \absentrainment cuando la variable acústico prosódica es \ENGMAX; esto es, con $\estslope < 0$. Esto tiene sentido, ya que a mayor mimetización de los interlocutores, la dificultad de estos para hablar debería disminuir. Por otro lado, $\estslope > 0$ cuando la variable acústico-prosódica es \ENGMEAN, lo cual no era un resultado esperado, pero bien puede ser parte del error estadístico.
  \item La variable \svbored se comporta de idéntica manera, sólo que con \FOMEAN.
  \item \svplanning y \svencourages, otras variables positivas, no presentan valores significativos.
  \item \svdislikes no presenta valores significativos
\end{itemize}


En resumen, encontramos fuerte evidencia empírica en favor de la hipótesis de que el valor absoluto del \entrainment se relaciona de manera positiva con atributos sociales de características positivas, mientras que lo hace de manera inversa con los que tienen connotaciones negativas.

Un hecho a destacar es que esta medida del entrainment es consistente con otras métricas definidas en otros trabajos, como las construídas en \cite{gravano2015backward} sobre anotaciones discretas de los patrones entonacionales usando la convención ToBI\cite{pitrelli1994evaluation}.

\begin{table}[t]
\begin{figure}[ht]
\centering
% psl is "Positive Slope"
\newcommand{\psl} { $+$ }
\newcommand{\ppsl} { $++$ }
\newcommand{\pppsl} { $+++$ }

% nsl stands for "Negative SLope"
\newcommand{\nsl} { $-$ }
\newcommand{\nnsl} { $--$ }
\newcommand{\nsl} { $---$ }


\begin{tabular}{| c | c | c | c | c | c |}
  \hline
               &\ENGMAX  & \ENGMEAN  & \FOMEAN  & \FOMAX  & NOISERATIO  \\
  \hline
  contributes  &         &           & \psl     &         & \psl        \\ \hline
  clear        & \pppsl  &           & \psl     &         & \psl        \\ \hline
  engaged      &         &           & \psl     &         &             \\ \hline
  planning     &         &           &          &         &             \\ \hline
  encourages   &         &           &          &         &             \\ \hline
  difficult    & \nsl    &           &          &         &             \\ \hline
  bored        &         &           & \nsl     &         &             \\ \hline
  dislikes     &         &           &          &         &             \\ \hline
  \hline
& PHON\_AVG & PHON\_COUNT & SHIMMER & SYL\_AVG & SYL\_COUNT \\
  \hline
contributes  &      &  &  &  &        \\ \hline
  clear      & \psl &  &  &  & \psl   \\ \hline
  engaged    &      &  &  &  &        \\ \hline
  planning   &      &  &  &  &        \\ \hline
  encourages &      &  &  &  &        \\ \hline
  difficult  &      &  &  &  &        \\ \hline
  bored      &      &  &  &  &        \\ \hline
  dislikes   &      &  &  &  &        \\ \hline
  \hline
\end{tabular}


\caption{Tabla que representa los resultados significantes del experimento. En una de las entradas, tenemos los nombres abreviados de las variables sociales, y en la otra las variables a/p. El símbolo \psl representa valor significante y positivo de la pendiente de la regresión de efectos fijos, mientras que \nsl representa significante y negativo }

\label{sign_table}

\end{figure}

\caption{Tabla que representa los resultados significantes del análisis. En una de las entradas, tenemos los nombres abreviados de las variables sociales, y en la otra las variables a/p. El símbolo \psl representa valor significante y positivo de la pendiente de la regresión de efectos fijos, mientras que \nsl representa significante y negativo. \psl representa $p < 0.10$, \ppsl $p < 0.5$, y \pppsl $p < 0.01$. Análogamente para \nsl, \nnsl, y \nnnsl }
\label{sign_table}
\end{table}
