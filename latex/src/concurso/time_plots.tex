\section{Time plots}
\label{sec:time_plots}
Usando la técnica descripta con las variaciones que consideramos en la anterior sección, generamos dos series de tiempo para cada tarea, una por cada hablante. 

Dada una ventana, puede ocurrir que alguno de los interlocutores no haya hablado, o su interacción haya sido demasiado breve como para medir sus variables \ap. Como ya mencionamos en la Sección \ref{sec:tama_modifications}, y a diferencia de \cite{KOU2008.2}, construimos las series sin ese punto, y sin interpolarlo tampoco.

De estas tareas, sólo nos quedamos con aquellas que tengan al menos 5 puntos definidos para cada serie, de manera que tenga sentido poder calcular la correlación cruzada más adelante. Con esto, no sólo nos interesa la duración de la charla, sino cierta calidad de las series generadas. En la Tabla \ref{tab:time_series} pueden verse las tareas que tuvimos en consideración, junto a su duración.

\begin{table}[t]
\centering

\adjustbox{max width=\textwidth}{\begin{tabular}{lllllllllllll}
\toprule
Task & S-01 & S-02 & S-03 & S-04 & S-05 & S-06 & S-07 & S-08 & S-09 & S-10 & S-11 & S-12 \\
\midrule
01 &         -- &         -- &    149.888 &         -- &         -- &         -- &         -- &         -- &     54.514 &    106.096 &         -- &     56.135 \\
02 &         -- &         -- &         -- &         -- &         -- &         -- &         -- &         -- &     41.711 &     63.837 &         -- &         -- \\
03 &         -- &     51.762 &         -- &     80.737 &     77.977 &     69.260 &     68.489 &     49.607 &         -- &    122.272 &     81.037 &         -- \\
04 &         -- &    187.201 &     93.333 &     76.131 &     79.946 &     99.240 &     84.342 &         -- &     58.020 &    129.621 &     67.977 &     95.292 \\
05 &         -- &         -- &         -- &     86.336 &         -- &    126.759 &    145.849 &     90.742 &     45.773 &    134.206 &         -- &         -- \\
06 &         -- &         -- &         -- &         -- &         -- &    148.218 &     50.672 &     60.281 &     46.165 &     66.762 &     46.773 &     40.200 \\
07 &         -- &     66.024 &         -- &    117.762 &         -- &     72.410 &         -- &     87.702 &     85.900 &    110.675 &     65.758 &         -- \\
08 &         -- &    458.885 &     98.681 &    203.867 &         -- &    188.708 &     59.933 &     48.144 &         -- &    157.442 &         -- &     81.165 \\
09 &         -- &         -- &         -- &     75.551 &    134.247 &     83.045 &    108.786 &         -- &     62.128 &    404.014 &     41.097 &     92.555 \\
10 &     50.131 &    231.392 &    162.895 &    242.588 &         -- &    122.408 &     71.198 &     74.775 &         -- &    356.079 &     69.834 &     92.769 \\
11 &         -- &     74.400 &         -- &     98.634 &     70.189 &         -- &     58.911 &         -- &     72.947 &    104.036 &     59.495 &    101.970 \\
12 &     61.331 &     90.100 &    129.129 &    182.917 &         -- &    130.375 &     75.891 &     57.656 &         -- &    101.661 &         -- &     64.842 \\
13 &     55.146 &    124.095 &    108.196 &    144.193 &    114.720 &         -- &         -- &     83.828 &     94.087 &    174.009 &     84.824 &     91.525 \\
14 &         -- &     75.334 &         -- &         -- &    107.356 &         -- &     52.583 &    144.378 &     75.589 &    108.456 &     91.648 &     98.487 \\
\bottomrule
\end{tabular}
}

\caption{Tabla de tareas seleccionadas y sus duraciones}
\label{tab:time_series}
\end{table}


Como primer paso siempre recomendado en el análisis de series de tiempo \cite{CHATFIELD}, graficamos los time plots conjunto de cada par de series, a la vez que sus autocorrelogramas. 

A priori, las series tienen aspecto de series autoregresivas de orden uno. Es decir, series que son de la forma $X_t = \alpha X_{t-1} + e_t + c$, con $e_t$ ruido blanco, $\alpha$ y $c$ constantes. Este hecho es esperable  por la construcción misma del método TAMA, ya que la ventana de cada punto tiene un solapamiento con la ventana anterior. Más aún, uno esperaría que $\alpha \sim 0.5$ ya que nuestras ventanas tienen ese índice de overlap. Los autocorrelogramas de las series, por otro lado, tienen en su mayoría un valor significativo en $k = 1$, el valor del $\alpha$ de la autoregresión.

El hecho de que los autocorrelogramas desciendan rápidamente a cero es un indicio de que las series de tiempo construidas son estacionarias. Esto nos habilita a efectuar el análisis bivariado de las series.
