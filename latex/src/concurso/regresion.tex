\section{Análisis de regresión}

Llegado a este punto, dada una variable acústico-prosódica, nos interesó evaluar la relación entre el entrainment o mimetización sobre dicha variable y las distintas variables sociales. Con esto en mente, planteamos un modelo de regresión lineal tomando como nuestra variable \emph{explicativa} la mimetización, y la variable \emph{dependiente} será la variable social elegida. Este análisis de regresión nos permitió observar cuál es la variación conjunta de ellas.

Nuestra hipotesis consistió en que la mimetización (por ejemplo, en la intensidad o pitch) se relacionaría de manera directa con ciertas variables sociales de connotación positiva (por ejemplo, la compenetración en el juego) y que se relacionaría de manera inversa con aquellas de carácter negativo (el aburrimiento o el desagrado por su compañero), siguiendo la línea de trabajos previos \cite{gravano2015backward}.

El tipo de regresión que utilizamos en este trabajo es el de efectos fijos. Este modelo \cite[chap 16]{gujarati1999} ayuda a controlar la heterogeneidad no observada cuando ésta es constante en el tiempo para cada sujeto del sistema. Asumimos que estos factores son inherentes a la conversación entre el hablante y su interlocutor, y por este motivo, definimos los sujetos (en el lenguaje del modelo estadístico) como cada uno de los hablantes y sus respectivas sesiones. No nos importa si el mismo sujeto se repite en otra sesión: cada hablante de una sesión es un sujeto distinto para el modelo de efectos fijos.
