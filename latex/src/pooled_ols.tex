\subsection{Modelo agrupado o \emph{pooled}}

En el modelo agrupado o \emph{pooled}, no distinguimos entre datos provenientes de distintos ``grupos'' \cite{gujarati1999} y sobre éstos calculamos la regresión lineal, agrupando todos los datos disponibles.

Un problema que surge con este tipo de regresión es que niega todo tipo de \emph{heterogeneidad} de los datos: estos pueden provenir de interlocutores más o menos empáticos, o cuya interacción en el juego se vio influída por factores no medidos en el experimento. Todo ésto es descartado, aún cuando puede afectar seriamente  el resultado obtenido.

AGREGAR GRAFICO DE EJEMPLO PARA ESTO

\begin{figure}[b!]
\includegraphics[width=15cm]{images/regression_F0_MEAN_1.pdf}
\caption{Gráfico de los pares entrainment-variable a/p, junto a la regresión lineal obtenida \label{regresion_clasica} para \emph{F0\_MEAN}}
\end{figure}

En el modelo clásico dio resultados con baja significancia. En \ref{regresion_clasica} puede verse el gráfico de \emph{F0\_MEAN} y 4 variables sociales y en \ref{regresion_clasica_tabla} pueden verse los valores de las estimaciones de $\estslope$ junto a sus p-valores.

% "ENG_MEAN"
% latex table generated in R 3.2.2 by xtable 1.8-0 package
% Thu Jan  7 03:01:56 2016
\begin{tabular}{rrrrr}
  \hline
 & Estimate & Std. Error & t value & Pr($>$$|$t$|$) \\
  \hline
bored\_with\_game & 10.65 & -0.44 & 0.00 & 0.66 \\
  difficult\_for\_partner\_to\_speak & 10.56 & -0.54 & 0.00 & 0.59 \\
  contributes\_to\_successful\_completion & 59.62 & -1.15 & 0.00 & 0.25 \\
  engaged\_in\_game & 73.14 & 0.55 & 0.00 & 0.59 \\
  gives\_encouragement & 47.49 & -0.04 & 0.00 & 0.97 \\
  making\_self\_clear & 52.97 & -0.99 & 0.00 & 0.33 \\
  planning\_what\_to\_say & 32.02 & -1.81 & 0.00 & 0.07 \\
  dislikes\_partner & 9.61 & -0.94 & 0.00 & 0.35 \\
   \hline
\end{tabular}

% "F0_MEAN"
% latex table generated in R 3.2.2 by xtable 1.8-0 package
% Thu Jan  7 03:01:56 2016
\begin{tabular}{rrrrr}
  \hline
 & Estimate & Std. Error & t value & Pr($>$$|$t$|$) \\
  \hline
bored\_with\_game & 10.63 & -0.18 & 0.00 & 0.86 \\
  difficult\_for\_partner\_to\_speak & 10.68 & -0.97 & 0.00 & 0.33 \\
  contributes\_to\_successful\_completion & 59.38 & 0.89 & 0.00 & 0.37 \\
  engaged\_in\_game & 73.41 & 0.77 & 0.00 & 0.44 \\
  gives\_encouragement & 47.59 & 1.09 & 0.00 & 0.28 \\
  making\_self\_clear & 52.91 & -0.32 & 0.00 & 0.75 \\
  planning\_what\_to\_say & 31.49 & 0.39 & 0.00 & 0.70 \\
  dislikes\_partner & 9.81 & -1.74 & 0.00 & 0.08 \\
   \hline
\end{tabular}

% "F0_MAX"
% latex table generated in R 3.2.2 by xtable 1.8-0 package
% Thu Jan  7 03:01:56 2016
\begin{tabular}{rrrrr}
  \hline
 & Estimate & Std. Error & t value & Pr($>$$|$t$|$) \\
  \hline
bored\_with\_game & 11.08 & 2.46 & 0.00 & 0.01 \\
  difficult\_for\_partner\_to\_speak & 10.65 & 0.52 & 0.00 & 0.60 \\
  contributes\_to\_successful\_completion & 60.08 & -0.98 & 0.00 & 0.33 \\
  engaged\_in\_game & 74.20 & -0.57 & 0.00 & 0.57 \\
  gives\_encouragement & 48.17 & -1.04 & 0.00 & 0.30 \\
  making\_self\_clear & 53.86 & -2.32 & 0.00 & 0.02 \\
  planning\_what\_to\_say & 31.86 & -1.05 & 0.00 & 0.30 \\
  dislikes\_partner & 9.65 & 0.97 & 0.00 & 0.33 \\
   \hline
\end{tabular}

% "NOISE_TO_HARMONICS_RATIO"
% latex table generated in R 3.2.2 by xtable 1.8-0 package
% Thu Jan  7 03:01:56 2016
\begin{tabular}{rrrrr}
  \hline
 & Estimate & Std. Error & t value & Pr($>$$|$t$|$) \\
  \hline
bored\_with\_game & 10.75 & -0.44 & 0.00 & 0.66 \\
  difficult\_for\_partner\_to\_speak & 10.64 & 0.16 & 0.00 & 0.87 \\
  contributes\_to\_successful\_completion & 60.34 & -0.75 & 0.00 & 0.45 \\
  engaged\_in\_game & 74.49 & -0.16 & 0.00 & 0.87 \\
  gives\_encouragement & 48.39 & -0.79 & 0.00 & 0.43 \\
  making\_self\_clear & 53.60 & -0.11 & 0.00 & 0.91 \\
  planning\_what\_to\_say & 31.99 & -0.02 & 0.00 & 0.99 \\
  dislikes\_partner & 9.62 & -1.39 & 0.00 & 0.17 \\
   \hline
\end{tabular}




