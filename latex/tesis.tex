\documentclass[11pt,a4paper,twoside]{tesis}
% SI NO PENSAS IMPRIMIRLO EN FORMATO LIBRO PODES USAR
%\documentclass[11pt,a4paper]{tesis}

\usepackage{graphicx}
\usepackage[utf8]{inputenc}
\usepackage[spanish]{babel}
\usepackage[left=3cm,right=3cm,bottom=3.5cm,top=3.5cm]{geometry}

\begin{document}

%%%% CARATULA
% Comentar y descomentar según corresponda
\def\titulo{Licenciado }

\def\autor{Juan Manuel Pérez}
\def\tituloTesis{Mimetización entre interlocutores}
\def\runtitulo{Medición de la mimetización entre interlocutores utilizando series de tiempo}
\def\runtitle{Measuring entrainment between speakers using time series}
\def\director{Agustín Gravano}
\def\codirector{Ramiro Gálvez}
\def\lugar{Buenos Aires, 2015}
\input{caratula}

%%%% ABSTRACTS, AGRADECIMIENTOS Y DEDICATORIA
\frontmatter
\pagestyle{empty}
\chapter*{\runtitulo}

Los sistemas de diálogo humano-computadora son cada vez más frecuentes, y sus aplicaciones comprenden una amplia gama de rubros: desde aplicaciones móviles, motores de búsqueda, juegos, hasta tecnologías de asistencia para ancianos y discapacitados. Si bien es cierto que estos sistemas logran captar buena parte de la dimensión lingüística de la comunicación humana, tienen un déficit importante a la hora de procesar y transmitir el aspecto superestructural de la comunicación oral, que radica en el intercambio de afecto, emociones, actitudes y otras intenciones de los participantes.

El \emph{entrainment} (mimetización) es un fenómeno inconsciente que se manifiesta a través de la adaptación de posturas, forma de hablar, gestos faciales y otros comportamientos entre dos o más interactores. A su vez, la ocurrencia de esta mimetización está fuertemente emparentada con el sentimiento de empatía y compenetración entre los participantes. En nuestro caso, nos es de interés el \emph{entrainment} sobre las variables acústico-prosódicas, como el tono, intensidad, y otras.

En el presente trabajo, nos proponemos explorar y refinar una métrica del entrainment acústico-prosódico definida en trabajos previos. Analizamos la relación entre los valores obtenidos y las percepciones sociales que terceros tienen sobre las conversaciones, en un corpus de diálogos orientados a tareas en inglés.

\bigskip

\noindent\textbf{Palabras claves:} Procesamiento del Habla, Series de Tiempo, Entrainment, Regresión Lineal


\cleardoublepage
\input{abs_en.tex}

\cleardoublepage
\input{agradecimientos.tex} % OPCIONAL: comentar si no se quiere

\cleardoublepage
\hfill \textit{A mis amigos.}

\hfill \textit{A mis compañeros de la Facultad, con los que tanto remamos.}

\hfill \textit{A mi familia, que me acompañaron.}  % OPCIONAL: comentar si no se quiere

\cleardoublepage
\tableofcontents

\mainmatter
\pagestyle{headings}

%%%% ACA VA EL CONTENIDO DE LA TESIS

\chapter{Introducción}

\section{Introducción}

Los sistemas de diálogo humano-computadora son cada vez más frecuentes, y sus aplicaciones comprenden una amplia gama de rubros: desde aplicaciones móviles, motores de búsqueda, juegos, o tecnologías de asistencia para ancianos y discapacitados.

Si bien es cierto que estos sistemas logran captar la dimensión lingüística de la comunicación humana, tienen un déficit importante a la hora de procesar y transmitir el aspecto superestructural de la comunicación, que radica en el intercambio de afecto, emociones, actitudes y otras intenciones de los participantes. La habilidad de los participantes de poder expresar, comprender, y reaccionar de acuerdo a estas señales sociales es necesaria para el entendimiento mutuo y una comunicación exitosa.

Un aspecto particular de la comunicación es el fenómeno de \emph{entrainment}(arrastre, mimetización, efecto camaleón), que comprende la adaptación inconsciente de las variables acústicas/prosódicas(a/p) (por ejemplo, el tono de la voz, la velocidad del habla, etc) de manera dinámica en el transcurso de una o varias interacciones. Este fenómeno ha sido introducido por \emph{Brennan et al}\cite{BRE1996} en 1996, y se ha observado que la convergencia de los participantes en estas variables ocurre en conjunto con una interacción más fluída y un mayor sentimiento de simpatía por sus interlocutores \cite{CHAR1999}.

Poder medir esta mimetización de los interlocutores no es una tarea fácil, sin embargo. En primer lugar, un diálogo no es una sucesión de turnos, sino que es una serie de tiempo dinámica, llena de interrupciones. Más aún, la mimetización no tiene un carácter instantáneo, sino que se sucede a lo largo de la interacción entre los participantes. Estos factores dificultan ostensiblemente poder modelar este fenómeno.

%%%% BIBLIOGRAFIA
\backmatter
\begin{thebibliography}{9}
\bibitem{BRE1996}
    Brennan:
    \emph{Lexical entrainment in spontaneous dialog},
    1996
\bibitem{CHAR1999}
    Chartrand, Bargh:
    \emph{The chameleon effect: The perception-behavior link and social interaction},
    1999
\bibitem{DEL2013}
    De Looze, Scherer, Vaughan, Campbell:
    \emph{Investigating automatic measurements of prosodic accommodation and its dynamics in social interaction},
    2013
\bibitem{KOU2008}
    Kousidis et al:
    \emph{Towards measuring continuous acoustic feature convergence in unconstrained spoken dialogues},
    2008
\bibitem{KOU2009}
    Kousidis, Dorran, McDonnell \& Coyle:
    \emph{Time Series Analysis of Acoustic Feature Convergence in Human Dialogues},
    2009
\bibitem{LEV2012}
    Levitan, Gravano, Willson, Benus, Hirschberg \& Nenkova,
    \emph{Acoustic-Prosodic entrainment and social behavior},
    2012
\bibitem{CHATFIELD}
    Chatfield C.,
    \emph{The analysis of time series: an introduction, Third Edition}
    1984
\end{thebibliography}
\end{document}
